



   Cho một tam giác đều có cạnh bằng N. Ba đỉnh của tam giác ban đầu được đánh dấu bằng 3 số nguyên dương đôi một khác nhau. KVD chia tam giác đã cho thành N\textasciicircum2 tam giác đều nhỏ có cạnh 1 và gán vào đỉnh của các tam giác mới 1 số thực sao cho nếu ABC và BCD là 2 trong các tam giác nhỏ thì tổng các số đánh trên 2 đỉnh A,D bằng tổng các số đánh trên hai đỉnh B,C.  

   Tuy nhiên, sau cả buổi gán số như vậy KVD muốn tính tổng các số đã đánh. Nhưng do quá mệt mỏi nên KVD muốn nhờ các bạn lập trình tính tổng các số đã được gán.  

\subsubsection{   Input  }
\begin{itemize}
	\item     Gồm một dòng duy nhất chứa 4 số N,a,b,c(a,b,c là 3 số được đánh vào 3 đỉnh của tam giác ban đầu).   
\end{itemize}

\subsubsection{   Output  }
\begin{itemize}
	\item     In ra số duy nhất là phần nguyên tổng của các số được đánh.   
\end{itemize}

\subsubsection{   Giới hạn  }
\begin{itemize}
	\item     0$<$ N  $\le$ 1000000000.   
	\item     0=$<$ a,b,c  $\le$ 1000000.   
\end{itemize}

\subsubsection{   Ví dụ  }
\begin{verbatim}
Input
5 1 2 3
Output
42
\end{verbatim}
