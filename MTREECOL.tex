







   Bob muốn tô màu một cây N nút. Một nút được tô khi và chỉ khi nút cha của nó được tô và chỉ được tô từng nút một. Mỗi nút tô mất 1 đơn vị thời gian (thời điểm bắt đầu là 0).  

   Nếu Fi là thời điểm kết thúc tô nút i thì chi phí cho tô nút i là Ci x Fi (Ci cho trước).  

   Ví dụ, nếu tô cây sau theo thứ tự 1, 3, 5, 2, 4 và chi phí Ci cho từng nút là 1, 2, 1, 2 và 4 thì tổng chi phí là 33 và nó là nhỏ nhất.  

\href{http://tinypic.com}{}


\includegraphics{http://i40.tinypic.com/whxkld.jpg}



   Cần tính chi phí nhỏ nhất này với một cây bất kỳ.  

\subsubsection{   Input  }

   Gồm nhiều bộ test. Dòng đầu chứa hai số nguyên N và R (1  $\le$  N  $\le$  1000, 1  $\le$  R  $\le$  N), với N là số nút và R là chỉ số của nút gốc. Dòng thứ hai chứa N số nguyên C1, C2, .., CN (1  $\le$  Ci  $\le$  500). N-1 dòng tiếp theo mỗi dòng chứa hai số nguyên V1, V2 là cạnh nối hai nút V1 và V2, V1 là cha của V2.  

   Kết thúc là bộ test có N=R=0 và không cần xử lý.  
\begin{verbatim}

\\Sample Input
\\5 1 
\\1 2 1 2 4 
\\1 2 
\\1 3 
\\2 4 
\\3 5 
\\0 0 
\\\end{verbatim}

\subsubsection{   Output  }

   Với mỗi bộ test, in ra chi phí nhỏ nhất cần trả.  
\begin{verbatim}

\\Sample output
\\33
\\\end{verbatim}

