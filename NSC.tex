








\includegraphics{../../../content/candies.jpg}

   Tết Trung Thu sắp đến và chị Nuga đã mua rất nhiều kẹo để chia cho các em của mình. Tổng số Nuga có M chiếc kẹo và cần   \textbf{    chia hết   }   cho N em. Để cho tiện, ta đánh số các em từ 1 đến N.  

   Nuga biết rằng em thứ i chỉ vui khi nhận được ít nhất A   $_    i   $   cái kẹo. Nhưng bố mẹ của em ấy cũng không muốn con mình ăn quá B   $_    i   $   chiếc (để tránh sâu răng). Như thế, Nuga phải chia cho em thứ i số kẹo là X   $_    i   $   thỏa mãn A   $_    i   $   ≤ X   $_    i   $   ≤ B   $_    ­i   $   .  

   Bạn hãy giúp Nuga tính xem có bao nhiêu cách chia M chiếc kẹo cho N em để thỏa mãn tất cả các yêu cầu trên.  

\subsubsection{   Dữ liệu  }

   - Dòng đầu ghi 2 số nguyên M, N.  

   - Dòng thứ hai gồm N số nguyên A   $_    1   $   , A   $_    2   $   , …, A   $_    n   $   .  

   - Dòng thứ ba gồm N số nguyên B   $_    1   $   , B   $_    2   $   , …, B   $_    n   $   .  

\subsubsection{\textbf{    Giới hạn   }}

   - Trong 30\% tổng số test, 1 ≤ N ≤ 5 và 0 ≤ A   $_    i   $   ≤ B   $_    i   $   ≤ M ≤ 20.  

   - Trong các test còn lại, 1 ≤ N ≤ 16 và 0 ≤ A   $_    i   $   ≤ B   $_    i   $   ≤ M ≤ 10   $^    9   $   .  

\subsubsection{   Kết quả  }

   Một dòng duy nhất chứa số cách chia thỏa mãn.  

\subsubsection{   Ví dụ  }
\begin{verbatim}
\textbf{Dữ liệu:}
\\6 3
\\0 0 0
\\3 2 4
\\
\\\textbf{Kết quả:}
\\9\end{verbatim}



