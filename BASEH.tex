



   Cho một dãy số a[1],a[2],a[3],...,a[n] và hai số K,H được xác định như sau:  
\begin{itemize}
	\item     a[1]=1;   
	\item     Nếu K chẵn thì a[K]=H*a[K/2].   
	\item     Nếu K lẻ thì a[K]=H*a[(K-1)/2]+1.   
\end{itemize}

   Các bạn hãy lập trình tính số thứ K của dãy viết trong hệ cơ số H.  

\subsubsection{   Input  }
\begin{itemize}
	\item     Gồm một dòng duy nhất chứa 2 số K,H.   
\end{itemize}

\subsubsection{   Output  }
\begin{itemize}
	\item     In ra số duy nhất là kết quả bài toán.   
\end{itemize}

\subsubsection{   Giới hạn  }
\begin{itemize}
	\item     0$<$ K  $\le$ 1000000000.   
	\item     0=$<$ H  $\le$ 2008.   
\end{itemize}

\subsubsection{   Ví dụ  }
\begin{verbatim}
Input
7 110
Output
111
\end{verbatim}
