



   Một xâu nhị phân được gọi là bậc k nếu nó bắt đầu bằng 0, kết thúc bằng 1 và chứa một xâu con không phải tiền tố hay hậu tố có bậc k−1, đồng thời k là số nguyên lớn nhất thỏa mãn tính chất đó. Xâu rỗng được hiểu là có bậc 0.  

   Cho xâu nhị phân S độ dài N, xác định số xâu con bậc k của S.  

\subsubsection{   Input  }

   Dòng 1: số nguyên T là số test. Mỗi test được cho trên hai dòng:  

   Dòng 1: hai số nguyên N, k  (2 ≤ N ≤ 10   $^    5   $   ; 1 ≤ k*2 ≤ N)  

   Dòng 2: xâu nhị phân S độ dài N.  

   Tổng độ dài xâu của các test không vượt quá 10   $^    5   $   .  

\subsubsection{   Output  }

   Dòng 1…T: dòng i ghi số nguyên kết quả của test i.  

\subsubsection{   Example  }
\begin{verbatim}
\textbf{Input:}

1

3 1

011 \textbf{Output:}
2\end{verbatim}