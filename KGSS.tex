

 

You are given a sequence A[1], A[2], ..., A[N] ( 0 ≤ A[i] ≤ 10\textasciicircum8 , 2 ≤ N ≤ 10\textasciicircum5 ). There are two types of operations and they are defined as follows:
\begin{itemize}
	\item \textbf{Update: }
\begin{itemize}
	\item \textbf{​}This will be indicated in the input by a 'U' followed by space and then two integers i and x.
	\item U i x , 1 ≤ i ≤ N, and x, 0 ≤ x ≤ 10\textasciicircum8.
	\item This operation sets the value of A[i] to x.
\end{itemize}
	\item \textbf{Query:}
\begin{itemize}
	\item This will be indicated in the input by a 'Q' followed by a single space and then two integers i and j.
	\item Q x y , 1 ≤ x $<$ y ≤ N.
	\item For Query, you must find i and j such that x ≤ i, j ≤ y and i != j, such that the sum A[i]+A[j] is maximized. Print the sum A[i]+A[j].
\end{itemize}
\end{itemize}

\subsubsection{Input}

The first line of input consists of an integer N representing the length of the sequence. Next line consists of N space separated integers A[i]. Next line contains an integer Q , Q ≤ 10\textasciicircum5, representing the number of operations. Next Q lines contain the operations.

\subsubsection{Output}

Output the maximum sum mentioned above, in a separate line, for each Query.

\subsubsection{Example}
\begin{verbatim}
Input:
5
1 2 3 4 5
6
Q 2 4
Q 2 5
U 1 6
Q 1 5
U 1 7
Q 1 5

Output:
7
9
11
12

\end{verbatim}

Warning: large Input/Output data, be careful with certain languages