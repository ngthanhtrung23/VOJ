





   Cho xâu nhị phân S có độ dài n (N$<$=5000).Ta định nghĩa phép S(k) là phép chuyển k kí tự cuối lên đầu xâu .Ví dụ S=1000101 thì S(2)=0110001.  

   Gọi A(S) là tập các xâu nhị phân là kết quả của phép S(p) OR S(q) với 0$<$=p,q$<$=n-1  

   Yêu cầu : Cho trước xâu S và T là 2 xâu nhị phân có độ dài n .Hãy kiểm tra xem T có thuộc tập A(S)  

   Input :Dòng đầu chứa xâu T,dòng thứ 2 chứa xâu S.  

   Output : Đưa ra ‘Yes’ hoặc ‘No’ tương ứng với có hoặc không  

\textbf{     Sample    }

       Input :     

   11111  

   10101  

       Output :     

   No  



       Input :     

   11110  

   10101  

       Output :     

   Yes  



