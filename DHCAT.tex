

Cải tiến từ 1 bài thi UVA.

Đồng hồ cát có dạng 2 tam giác đều chung đỉnh, gồm 2n-1 dòng.

Lần lượt mỗi dòng có: n số, n-1 số, ....,1, 2..., n số (n $<$ 21).

Mỗi ô ở dòng trên chỉ có thể di xuống ô phải dưới ( R ) hoặc ô trái dưới ( L ).

\textbf{Yêu cầu:}

Tìm đường đi có trọng số nhỏ nhất. Đường đi được mô tả là ô xuất phát ở hàng 1 (các ô được đánh số từ 0 tới n-1) và chuỗi LR mô tả đường đi.

Cho số S  $\le$  5000, đếm số đường đi có trọng số S và mô tả đường đi có thứ tự từ điển nhỏ nhất ứng với S.

\textbf{Input}

Dòng đầu tiên là 2 số n và S.

2n-1 dòng tiếp theo số a mô tả đồng hồ cát. ( a $\le$ 5000)

\textbf{Output }

Gồm 4 dòng :

Trọng số nhỏ nhất từ hàng 1  tới hàng cuối .

Mô tả đường đi ứng với trọng số nhỏ nhất ( nếu có nhiều đường đi, in ra đường đi có thứ tự từ điển nhỏ nhất )

Số đường đi có trọng số là S. Nếu không có đường thì in ra -1.

Đường đi ứng với trọng số S thỏa mãn.

\textbf{Example}
\begin{verbatim}
\textbf{Input:}
3 17
1 2 3
4 5
6
5 4
3 2 1\end{verbatim}
\begin{verbatim}
\textbf{Output:}
16
0 RRRR
2
0 RRRL\end{verbatim}
