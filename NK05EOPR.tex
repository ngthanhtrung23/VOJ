



   Cho một dãy số là một hoán vị của 12 số tự nhiên đầu tiên (từ 0 đến 11). Giả sử số 0 ở vị trí thứ i trong dãy số (vị trí được đánh số từ 0 đến 11, từ trái sang phải) thì bạn có thể đổi chỗ số 0 với số ở vị trí thứ j nếu thỏa mãn cả hai điều kiện sau:  
\begin{itemize}
	\item     | i – j | = d    $_     k    $    , với k=1..3 và (d    $_     1    $    ,d    $_     2    $    ,d    $_     3    $    ,d    $_     4    $    )=(1;3;6;12)    


	\item     [i/d    $_     k+1    $    ]=[j/d    $_     k+1    $    ], với [] là hàm phần nguyên   
\end{itemize}

   Bạn hãy tìm số phép đổi chỗ ít nhất để có thể sắp xếp dãy số theo thứ tự tăng dần  

\subsubsection{   Dữ liệu vào  }

   Dòng đầu tiên là một số nguyên t cho biết số lượng test (t $\le$ 20)  

   Mỗi bộ test bao gồm một dòng là dãy bao gồm các số từ 0 đến 11, mỗi số ngăn cách bởi một khoảng trắng.  

   Biết rằng mỗi dãy số cho trước luôn luôn có thể sắp xếp tăng dần bằng phép đổi chỗ đã quy định  

\subsubsection{   Kết qủa  }

   Với mỗi bộ test, in ra số phép đổi chỗ ít nhất để sắp xếp dãy số đã cho theo thứ tự tăng dần  

\subsubsection{   Ví dụ  }
\begin{verbatim}
Dữ liệu mẫu
2
1 10 2 3 0 5 7 4 8 6 9 11
6 4 1 0 3 5 9 7 2 10 11 8

Kết qủa
8
9
\end{verbatim}
