



   Cho N điểm phân biệt trên mặt phẳng toạ độ . Toạ độ của điểm i là ( Xi , Yi ) trong đó Xi , Yi là các số nguyên ( -10000 ≤ Xi , Yi ≤ 10000 ) .   
\\   Ta định nghĩa khoảng cách giữa 2 điểm (X1,Y1) , (X2,Y2) là khoảng cách Manhattan được tính = | X1 – X2 | + | Y1 – Y2 | .   
\\   Hàm Q(X,Y) := | X – X1 | + | X – X2 | + … + | X – Xn | + | Y – Y1 | + … |Y – Yn | .   
\\   ( Trong đó X , Y là 2 số nguyên thoả mãn -10000 ≤ X , Y ≤ 10000 và Xi ≠ X hoặc Yi ≠ Y với mọi i = 1 .. n  ) .   
\\   Hãy tìm tập tất cả các điểm nguyên (X,Y) để hàm Q(X,Y) có giá trị nhỏ nhất .  

\subsubsection{   Input  }

   Dòng 1 : số nguyên dương T là số bộ test ( T ≤ 20 ) .   
\\   Các nhóm dòng sau mô tả 1 bộ test . 1 bộ test sẽ có format như sau :   
\\   Dòng 1 : số nguyên dương N (  N ≤ 10000 ) .   
\\   N dòng tiếp theo , dòng thứ i gồm 2 số nguyên là toạ độ của điểm thứ i .  

\subsubsection{   Output  }

   Với mỗi bộ test ghi 1 dòng gồm 2 số nguyên dương S , K tương ứng là giá trị nhỏ nhất của hàm Q(X,Y) và số lượng điểm thoả mãn yêu cầu .   
\\

\subsubsection{   Example  }
\begin{verbatim}
Input:
1
2
0 1
1 0

Output:
2 2
\end{verbatim}