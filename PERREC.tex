

Cho 1 bảng kích thước N * N được chia thành các ô vuông đơn vị. Mỗi ô vuông có thể có màu đen hoặc trắng. Bây giờ, định nghĩa 1 \emph{ hình chữ nhật tốt } là 1 hình chữ nhật có các cạnh song song với cạnh của bảng và chỉ chứa các ô vuông màu trắng. 1 hình chữ nhật được gọi là \emph{ hoàn hảo, } nếu nó là 1 hình chữ nhật tốt, và không tồn tại 1 hình chữ nhật tốt nào khác chứa nó (tức không thể mở rộng hình chữ nhật này sang trái, phải, trên hay dưới).

\textbf{Yêu cầu}: Xác định số hình chữ nhật hoàn hảo của bảng đã cho.

Lưu ý:

Để giảm kích thước của input, bảng sẽ được tô màu theo quy tắc sau:
\begin{itemize}
	\item Ban đầu bảng chỉ chứa các ô vuông màu trắng
	\item Sinh 2 dãy số X và Y độ dài m theo quy tắc
\begin{itemize}
	\item X[0] = x0 mod N, Y[0] = y0 mod N
	\item X[i] = (X[i – 1] * a + b) mod N, Y[i] = (Y[i – 1] * c + d) mod N với 1  $\le$  i $<$ m,
	\item trong đó x0, y0, a, b, c, d, m là các số được cho trước, và P mod Q kí hiệu là phần dư của phép chia P cho Q
\end{itemize}
	\item Tô đen các ô có tọa độ (X[0],Y[0]), (X[1],Y[1]),…, (X[m – 1],Y[m – 1]). (Tọa độ của bảng được đánh số từ 0 đến N – 1 theo thứ tự từ trái qua phải, và từ trên xuống dưới)
\end{itemize}

\textbf{Input:}
\begin{itemize}
	\item 1 dòng duy nhất gồm 8 số nguyên N,m,x0,a,b,y0,c,d như mô tả trong đề bài
\end{itemize}

\textbf{Output:}
\begin{itemize}
	\item 1 dòng duy nhất ghi ra số lượng hình chữ nhật hoàn hảo thu được
\end{itemize}

\textbf{Giới hạn:}
\begin{itemize}
	\item 0 $<$ N  $\le$  2000
	\item 1  $\le$  m  $\le$  4000000
	\item 0  $\le$  a,b,c,d,x0,y0  $\le$  2000
\end{itemize}

\textbf{Example:}
\begin{verbatim}
Input 1
5 1 2 0 0 2 0 0

Output 1
4

Input 2
4 4 0 1 1 0 1 1

Output 2
6

Input 3
10 20 4 76 2 6 2 43

Output 3
12\end{verbatim}
