

Buôn lợn hay Đi chơi với lợn

FJ muốn đi từ A đến B. Trong khi đi, FJ đi qua n thành phố và bán lợn ở đó để kiếm lời. (FJ chuyển từ BÒ sang LỢN).
\\Giá thịt lợn ở thành phố j là pj \$/kq. Khoảng cách từ A đến j là dj km.
\\Lợn nặng nhẹ khác nhau. Vận chuyển 1 kq lợn mất t \$ cho 1 km quãng đường. 

Giúp FJ bán lợn sao cho nhiều lãi nhất. Mỗi thành phố, FJ chỉ bán 1 con.

\href{http://tinypic.com}{
\includegraphics{http://i40.tinypic.com/2unx73t.jpg}}
\\Bessa - Lợn đầu đàn của FJ!!!! :)))))

\subsubsection{Input}

Dòng đầu tiên là hai số n,t (1 ≤ n ≤ 1000), (1 ≤ t ≤ 10^9).
\\Dòng thứ hai là n số nguyên wi (1 ≤ wi ≤ 10^9) — trọng lượng từng con lợn.
\\Dòng thứ ba là n số nguyên dj (1 ≤ dj ≤ 10^9) — khoảng cách từ A đến thành phố i.
\\Dòng thứ tư là n số nguyên pj (1 ≤ pj ≤ 10^9) — giá 1kg thịt lợn ở thành phố i.

\subsubsection{Output}

In ra n số, số thứ i là chỉ số của con lợn được bán ở thành phố i.
\\Lợn được đánh số từ 1 đến n.
\begin{verbatim}
SAMPLE INPUT
3 1
10 20 15
10 20 30
50 70 60

SAMPLE OUTPUT
3 2 1\end{verbatim}
