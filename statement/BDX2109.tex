
\begin{verbatim}



\textbf{SỐ BẤT ĐỐI XỨNG}

 

Một chuỗi được gọi là \textbf{đối xứng} nếu đọc xuôi hay đọc ngược thì nội dung của nó không thay đổi, ví dụ: 12321.

 

Một số nguyên được gọi là \textbf{bất đối xứng} nếu nó không chứa bất kỳ một chuỗi đối xứng nào có độ dài lớn hơn 1 làm chuỗi con (các chữ số liên tiếp).

 

Ví dụ: 96276 là số bất đối xứng, còn 1732398 không phải là số bất đối xứng vì nó chứa một chuỗi con đối xứng “323”

 

Cho hai số nguyên L và R. Hãy tính \textbf{số lượng số bất đối xứng} nằm trong khoảng từ L đến R (tính cả 2 đầu).

 

\textbf{Input}

Dòng đầu chứa số nguyên T là số bộ test. Tiếp theo là T dòng, mỗi dòng gồm 2 số nguyên dương L và R.

 

\textbf{Output}

Gồm T dòng, mỗi dòng ghi một số nguyên duy nhất là số lượng số \textbf{b}\textbf{ất đối xứng} nằm trong khoảng từ L đến R (tính cả 2 đầu).

 

\textbf{Constraint}

T ≤ 100

0 ≤ L ≤ R ≤ 10$^18$

Time limit: \textbf{5s}

 

\textbf{Example}
\begin{tabular}\hline 


Input & 

Output  
\hline


2

123 321

123456789 987654321 & 

153

167386971  
\hline

\end{tabular}

 \end{verbatim}
