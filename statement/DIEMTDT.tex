

 
Kĩ năng tìm kiếm là kĩ năng cơ bản nhưng cũng cần phải biết nên cô không thể không
giao bài tập cho Nam có yêu cầu tương tự. Ở trong trường hợp này, cô giao cho Nam bài
tập về đoạn thằng và điểm và yêu cầu đếm số điểm trong một đoạn thằng.
Cho N điểm trên trục Ox có tọa độ xi và Q đoạn thằng, Nam cần biết trong mỗi đoạn
thằng có chứa bao nhiêm điểm trong N điểm trên. Một điểm p nằm trong đoạn [ai,bi]
nếu ai ≤ p ≤ bi. Ví dụ, nếu có các điểm là 1, 4, 6, 8, 10 và đoạn [ai,bi] là [0,5] thì có hai
điểm nằm trong đoạn này.
Giới hạn : 1 ≤ N ≤ 10\textasciicircum5, 1 ≤ Q ≤ 10\textasciicircum5, 1 ≤ ai $<$ bi ≤ 10\textasciicircum9, 1 ≤ xi ≤ 10\textasciicircum9.
Dữ liệu vào ghi trong file DIEMTDT.INP. Dòng đầu ghi số bộ test T (≤ 5). Dòng sau ghi
hai số N và Q. Dòng tiếp theo ghi N số nguyên là tọa độ của N điểm theo thứ tự tăng dần.
Tiếp theo là Q dòng mỗi dòng ghi hai số nguyên ai và bi . Dữ liệu thỏa điều kiện đề bài.
Dữ liệu ra ghi trong file DIEMTDT.OUT. Với mỗi bộ test ghi ra Q số, mỗi số một dòng,
là số điểm nằm trong mỗi đoạn thằng ứng với bộ Test đó.
Ví dụ:
DIEMTDT.INP
1
5 3
1 4 6 8 10
0 5
6 10
7 100000


DIEMTDT.OUT

2
3
2
