

    Cho đa đồ thị G gồm N đỉnh và M cạnh hai chiều. Các đỉnh được  đánh số từ 1 đến N, các cạnh được đánh số từ 1 đến M. Một cây khung của  đồ thị là một tập hợp N-1 cạnh sao cho giữa hai đỉnh bất kì đều có đường  đi tới nhau. Trọng số của một cây khung bằng tổng các trọng số của N-1  cạnh thuộc cây khung.

    Bạn hãy tìm cây khung nhỏ nhất và cây khung nhỏ thứ nhì của đồ  thị. Hai đồ thị được coi là khác nhau nếu tồn tại một cạnh thuộc cây  khung này mà không thuộc cây khung kia.

 

 \textbf{INPUT: }

-         Dòng đầu tiên ghi số N và M.

-         Tiếp theo là M dòng, dòng thứ i ghi thông tin về cạnh thứ i  gồm 3 số u, v, w với ý nghĩa có cạnh nối từ u đến v với độ dài là w.

 

\textbf{OUTPUT:}

-         Dòng thứ nhất ghi trọng số của cây khung nhỏ nhất.

-         Dòng thứ hai ghi trọng số của cây khung nhỏ thứ nhì.

    Dữ liệu đảm bảo tồn tại ít nhất hai cây khung.

 
\begin{tabular}\hline 


\textbf{Input} & 

\textbf{Output}  
\hline


4 5

1 2 10

2 3 10

3 4 10

4 1 20

4 2 15 & 

30

35  
\hline

\end{tabular}

 

\textbf{Giới hạn: }

-         Trong 30\% số test, 1 ≤ N ≤ 500, 1 ≤ M ≤ 5000.

-         Trong 30\% số test tiếp theo, 1 ≤ N ≤ 5000, 1 ≤ M ≤ 50000.

-         Trong tất cả các test, 1 ≤ N ≤ 50000, 1 ≤ M ≤ 500000, 1 ≤ w ≤ 10$^9$

 

\textbf{Lưu ý:}

-         Dữ liệu lớn, tránh sử dụng cin/cout để đọc và ghi dữ liệu.

 
