







   Palindrome is a string that has the property of reading the same in either direction (left to right or right to left). You are to determine whether a given string can be expressed as a concatenation of palindromes of even length.  

   Note: A string can be formed by concatenation of any number of even palindrome strings.  

\subsubsection{   Input  }

   First line contains   \textbf{    T   }   (T $<$ 100), the number of test cases.   \textbf{    T   }   lines follow, each containing the string corresponding to that particular test case.  

\textbf{    Note:   }

   There might be a new-line character (i.e. '$\backslash$r' in C++) at the end of each line. Be careful with your languages.  

\subsubsection{   Output  }

   Output consists of   \textbf{    T   }   lines, one corresponding to each test case. You should output   \emph{    YES   }   if the string can be expressed as concatination of even length palindromes and   \emph{    NO   }   otherwise.  

\subsubsection{   Example  }
\begin{verbatim}
\textbf{Input:}
3
madam
aA
aabb

\textbf{Output:}
NO
NO
YES 
\end{verbatim}

\subsubsection{   Constraints  }

   Length of string ≤ $10^{6}$

