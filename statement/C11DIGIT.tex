

Trong đợt hội trại hiện đại của trường Lê Quý Đôn, sau khi vượt qua những trận AOE đầy căng thẳng với các Game thủ chuyên tin thì tiếp theo, TMB phải vượt qua những trận đấu trí đầy khốc liệt với đội tuyển toán của trường, với phần thi hết sức khó khăn: … TÍNH NHẨM. Do đã làm mất máy tính Casio từ lâu nên khả năng tính nhẩm của TMB cũng không tệ. Vượt qua được vòng 1 và vòng 2, những người còn lại trên sàn đấu là những con người có có tốc độ tính toán ngang ngửa siêu máy tính. Vòng 3 kế đến không hề dễ. Do việc tính nhầm những con số bình thường không còn là vấn đề với những người còn lại. Nên việc kế đến là tính toán những biểu thức có giá trị vô cùng lớn. Ban tổ chức cũng cho biết biểu thức đó sẽ có dạng:

E = x\textasciicircum1 + x\textasciicircum2 + x\textasciicircum3 + … + x\textasciicircumk

Do kết quả của E sẽ khá lớn nên để tránh làm mỏi miệng thí sinh, ban tổ chức cũng chi cần các thí sinh cho biết số chữ số của kết quả.

 

\textbf{Yêu cầu: }Hãy giúp TMB vượt qua nốt vòng này nhé :D

\textbf{Giới hạn:}

x $<$= 10\textasciicircum16

k $<$= 10\textasciicircum16

Có 30\% số test với N , K $<$= 10\textasciicircum3

Có 50\% số test với N , K $<$= 10\textasciicircum6

\textbf{Input:}
\begin{itemize}
	\item 

Gồm 1  dòng chứa chứa 2 số nguyên là  x và k
\end{itemize}

\textbf{Output:}
\begin{itemize}
	\item 

Gồm 1 dòng là  kết quả
\end{itemize}

\textbf{Ví dụ:}
\begin{tabular}\hline 


\textbf{Input} & 

\textbf{Output}  
\hline


2 3 & 

2  
\hline

\end{tabular}
