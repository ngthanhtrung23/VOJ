



   Xét số nguyên dương   \textbf{\emph{     x    }}   . Gọi   \textbf{\emph{     S(x)    }}   là hàm tính tổng các chữ số của   \textbf{\emph{     x    }}   (trong dạng biểu diễn cơ số 10). Ví dụ,   \textbf{\emph{     S    }}   (21) = 2+1 = 3.  

   Cho số nguyên dương   \textbf{\emph{     n    }}   . Ta có thể biểu diễn   \textbf{\emph{     n    }}   dưới dạng tổng của   \textbf{\emph{     k    }}   số nguyên   \textbf{\emph{     a    }}$_    1   $   ,   \textbf{\emph{     a    }}$_    2   $   , . . .,   \textbf{\emph{     a     $_      k     $}}   .  

\textbf{\emph{     Yêu cầu    }}   : Cho hai số nguyên dương   \textbf{\emph{     n    }}   và   \textbf{\emph{     m    }}   (   \textbf{\emph{     n    }}   ,   \textbf{\emph{     m    }}   ≤ 10   $^    12   $   ). Hãy xác định   \textbf{\emph{     k    }}   nhỏ nhất, sao cho với nó tồn tại các số   \textbf{\emph{     a    }}$_    1   $   ,   \textbf{\emph{     a    }}$_    2   $   , . . .,   \textbf{\emph{     a     $_      k     $}}   thỏa mãn:  
\begin{itemize}
	\item     a1+a2+ ...+ak= N   
	\item     S(a1)+ S(a2)+...+S(ak)= M   
\end{itemize}

\textbf{\emph{     Dữ liệu    }}   : 2 dòng chứa hai số nguyên   \textbf{\emph{     n    }}   và   \textbf{\emph{     m    }}   .  

\textbf{\emph{     Kết quả    }}   : kết quả đưa ra trên một dòng dưới dạng số nguyên. Nếu không tồn tại cách phân tích thì đưa ra số -1.  

   VD:  

   Input  

   100  

   1  

   Output  

   1  