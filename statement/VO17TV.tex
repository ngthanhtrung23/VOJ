

Ngoại tình là một trong những việc làm bị xã hội gay gắt nhất. Có ba kiểu ngoại tình chủ yếu: Tán vợ hàng xóm, ... vợ bạn và cưới cô em vợ. Có những vụ ngoại tình cưới em vợ đã đi vào thi ca Việt Nam mà giờ đây chúng ta ai cũng biết.

Thúy Vân sau khi kết hôn với "chồng" của chị gái mình, để xóa đi mọi dấu tích của mối tình cũ, đã cố gắng xóa mọi tin nhắn trước kia giữa chồng mình và chị gái. (Ghi chú: trong truyện, Thúy Vân mang họ Vương chứ không phải họ G*** như ngày nay). Chồng cô, dù đã có vợ nhưng vẫn không quên nổi mối tình xưa, lén lút giấu vợ giữ lại N tin nhắn của người yêu cũ. Trong lúc vợ mải mê công tác tại huyện Lạc Rang, TP Bắp Rang, anh ta vẫn lôi ra đọc lại N tin nhắn này.

Có những lời lẽ được trao qua gửi lại rất nhiều lần. Kim Trọng gọi một đoạn tin nhắn là Kiều dị (kỳ diệu), nếu như nó là xâu con liên tiếp của ít nhất K trong số N tin nhắn kia. Anh ta muốn tìm ra đoạn tin nhắn kỳ diệu dài nhất, và khắc cốt ghi tâm đoạn này.

\subsubsection{Input}

Dòng đầu tiên chứa hai số nguyên dương N và K.

N dòng sau, mỗi dòng chứa nội dung của một tin nhắn, là một xâu gồm các chữ cái in hoa trong tiếng Anh

\subsubsection{Output}

Gồm một số nguyên duy nhất là độ dài tin nhắn kỳ diệu dài nhất tìm được.

\subsubsection{Giới hạn}

Trong tất cả các test, 1 $<$= K $<$= N $<$= 50.

Trong 20\% số test, tổng độ dài các tin nhắn không quá 70.

Trong 50\% số test, tổng độ dài các tin nhắn không quá 1000.

Trong 100\% số test, tổng độ dài các tin nhắn không quá 100000.

Trong lúc thi bài của bạn chỉ được chấm với test ví dụ.

\subsubsection{Example}
\begin{verbatim}
\textbf{Input:}

3 2

ABC

BBBBC

CCACC\end{verbatim}
\begin{verbatim}
\textbf{Output:}

2\end{verbatim}