



\subsection{   Đếm K-chữ nhật  }

   Cho một bảng ô vuông kích thước M*N. Mỗi ô vuông chứa một kí tự của bảng chữ cái tiếng Anh ('A' .. 'Z').  

   Một K-chữ nhật của bảng là một hình chữ nhật với các cạnh song song với các cạnh của bảng, và chứa đúng K loại kí tự khác nhau.  

   Ví dụ, với bảng 4*3 sau:  
\begin{verbatim}
CED
CEB
CBC
DDA
\end{verbatim}

   Hình chữ nhật [(1,1), (2,2)] là một 2-chữ nhật của bảng vì nó chứa 2 kí tự khác nhau: C và E.  

   Cho M, N, K và bảng M*N. Tính xem có bao nhiêu K-chữ nhật trong bảng.  

\subsubsection{   Input  }

   Dòng đầu chứa 3 số nguyên M, N và K. (1 ≤ M, N ≤ 100, 1 ≤ K ≤ 26)  

   Tiếp theo là M dòng, mỗi dòng chứa N kí tự của bảng chữ cái tiếng Anh ('A' .. 'Z')  

\subsubsection{   Output  }

   Viết ra một số nguyên là số lượng K-chữ nhật trong bảng đã cho.  

\subsubsection{   Example  }
\begin{verbatim}
Input:
4 3 3
CED
CEB
CBC
DDA


Output:
12

\end{verbatim}