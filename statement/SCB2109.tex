

\textbf{SỐ CÂN BẰNG}

 

Một số nguyên dương được gọi là \textbf{số cân bằng} nếu nó thoả mãn cả 2 điều kiện sau:

 - \textbf{Số lượng} chữ số chẵn bằng \textbf{số lượng} chữ số lẻ.

 - \textbf{Tổng} các chữ số chẵn bằng \textbf{tổng} các chữ số lẻ.

 

Ví dụ: 1982, 11822989 là các số cân bằng.

 

Cho 2 số nguyên dương L và R. Tính \textbf{số lượng số cân bằng} nằm trong khoảng từ L đến R (tính cả 2 đầu).

 

\textbf{Input}

Dòng đầu chứa số nguyên T là số bộ test. Tiếp theo là T dòng, mỗi dòng gồm 2 số nguyên dương L và R .

 

\textbf{Output}

Gồm T dòng, mỗi dòng ghi một số nguyên duy nhất là \textbf{số lượng số cân bằng} nằm trong khoảng từ L đến R (tính cả 2 đầu).

 

\textbf{Constraint}

T ≤ 100

1 ≤ L ≤ R ≤ 10$^12$

Time limit: \textbf{5s}

 

\textbf{Example}
\begin{tabular}\hline 


Input & 

Output  
\hline


2

1   10000

45645   10987656 & 

450

29993  
\hline

\end{tabular}

 

 
