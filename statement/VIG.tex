

VIG là một hòn đảo hình chữ nhật. Chúng ta có thể chia nó ra thành  một lưới ô vuông kích thước n x m. Ô thuộc hàng i và cột j của lưới ô  vuông được gọi là ô (i, j). Các ô ở biên là những ô thuộc hàng 1 hoặc  hàng n hoặc cột 1 hoặc cột m. Từ một ô (i, j) bất kỳ không phải biên,  bạn có thể di chuyển sang 1 trong 4 ô kề cạnh. Tuy nhiên, có một số ô có  vật cản nên không thể di chuyển đến.

Chính vì những đặc tính như  vậy, Carrot quyết định thử nghiệm thuật toán dò đường của mình bằng cách  cho 1 con robot đi trên hòn đảo này. Tuy nhiên, thuật toán có lỗi và  con robot bị lạc ở đâu đó trên VIG. Do hòn đảo quá rộng, việc tìm kiếm  toàn bộ mất rất nhiều thời gian. May mắn thay, bằng một cách kỳ diệu nào  đó, Carrot đã xác định được con robot nằm ở 1 trong những vị trí có  khoảng cách ngắn nhất đến biên là dài nhất.

Khoảng cách ngắn nhất đến biên của ô (i, j) là số ô ít nhất phải đi qua (không tính ô xuất phát) để đến được 1 ô bất kỳ ở biên.

Nhiệm vụ của bạn là xác định xem có bao nhiêu vị trí như vậy và khoảng cách ngắn nhất từ những vị trí này đến biên.

\subsubsection{Input}
\begin{itemize}
	\item Dòng 1: 2 số nguyên n, m là kích thước của hòn đảo
	\item n dòng tiếp theo, mỗi dòng gồm m ký tự mô tả hòn đảo. Mỗi ô được đại diện bởi một trong 2 ký tự sau:

\begin{itemize}
	\item '*': ô có vật cản
	\item '.': ô không có vật cản
\end{itemize}
\end{itemize}

\subsubsection{Output}
\begin{itemize}
	\item Gồm 2 số nguyên dương lần lượt là số vị trí cần tìm và khoảng cách ngắn nhất từ những vị trí này đến biên.
\end{itemize}

\subsubsection{Ví dụ}
\begin{verbatim}
\textbf{Input:}4 5
*.*.*
*...*
*.*.*
*****
\textbf{Output:}3 2
(3 ô cần tìm là ô (3, 2) và (3, 4) và (2, 3))
\end{verbatim}

\subsubsection{Giới hạn}
\begin{itemize}
	\item 

40\% số test ứng với 40\% số điểm có n, m $<$= 100
	\item 

60\% số test còn lại có n, m $<$= 1000
\end{itemize}
\begin{verbatim}
 \end{verbatim}
