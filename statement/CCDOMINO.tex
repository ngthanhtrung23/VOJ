

Dominoes are gaming pieces used in numerous tile games. Each doimno piece contains two marks. Each mark consists of a number of spots (possibly zero). The number of spots depends on the set size. Each mark in a size \textbf{N} domino set can contain between 0 and \textbf{N} spots, inclusive. Two tiles are considered identical if their marks have the same number of spots, irregardles of reading order. For example tile with 2 and 8 spot marks is identical to the tile having 8 and 2 spot marks. A proper domino set contains no duplicate tiles. A \textbf{complete }set of size\textbf{ N} contains all posible tiles with \textbf{N} or less spots and no duplicate tiles. For example, the complete set of size 2 contains 6 tiles:


\includegraphics{../../../PTNKIBIG/content/CCDOMINO.png}

Write a program that will determine the total number of spots on all tiles of a complete size \textbf{N} set.

\subsubsection{Input}

The first and only line of input contains a single integer,\textbf{ N} (1 ≤ \textbf{N} ≤ 1000), the size of the complete set.

\subsubsection{Output}

The first and only line of output should contain a single integer, total number of spots in a complete size \textbf{N} set.

\subsubsection{Example}
\begin{verbatim}
\textbf{Input1:}

2



\textbf{Output1:}



\end{verbatim}
egin{verbatim}
\textbf{Input2:
\\}3\end{verbatim}
egin{verbatim}
\textbf{Output2:
\\}30\end{verbatim}
egin{verbatim}
\textbf{Input3:
\\}15\end{verbatim}
\begin{verbatim}
\textbf{Output3:
\\}2040\textbf{ }\end{verbatim}
