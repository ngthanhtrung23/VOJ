

Sau quy hoạch động, đồ thị, số học ... cô muốn kiểm tra kĩ năng đã học tiếp theo của Nam
xem biết cách sử dụng như thế nào. Tất nhiên tên kĩ năng đó là bí mật. Cô kể rằng là cụ
cố của Nam là một người chăn trâu có tiếng của địa phương, ông có N (0$<$= N$<$= 10\textasciicircum5)
chuồng trâu được xây ở trên đường vào nhà ông. Mỗi chuồng được coi là một điểm có
tọa độ xi ( 0 $<$= xi $<$= 10\textasciicircum9). Ông có rất nhiều trâu, trong đó có C con trâu đặc biệt không
ưa nhau, chỉ cần nhìn thấy nhau là chúng sẽ trở nên hung tợn và có thể húc nhau. Để
tránh việc này, ông muốn cho các con trâu này, mỗi con vào một chuồng, sao cho khoảng
cách giữa hai con càng lớn càng tốt.
Dữ liệu vào trong file TRAU.INP. Dòng đầu tiên ghi hai số N và K. Dòng thứ hai ghi N
số xi theo thứ tự tăng dần.
Dữ liệu ra ghi trong file TRAU.OUT ghi một số duy nhất là khoảng cách lớn nhất có thể
tìm được.
Ví dụ:
TRAU.INP 
5 3
1 2 4 8 9


TRAU.OUT

3
Giải thích: Ở ví dụ 1, Khoảng cách lớn nhất có thể đạt được là 3. Lúc này, cụ cố của Nam
sẽ nhốt trâu vào chuồng ở vị trí 1, 4 và 8 (hoặc 9).
