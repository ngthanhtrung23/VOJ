

Professor Spook is consulting for NASA, which is planning a series of space shuttle flights and must decide which commercial experiments to perform and which instruments to have on board each flight. For each flight NASA considers a set \textbf{E = \{$E_{1}$, $E_{2}$, ..., $E_{m}$\}} of instruments experiments and the commercial sponsor of \textbf{$E_{j}$} has agreed to pay NASA \textbf{$p_{j}$} dollars for the results of the experiments.

The experiments use a set \textbf{I = \{$I_{1}$, $I_{2}$, ..., $I_{n}$\}} of instruments; each experiment \textbf{$E_{j}$} requires some of the instruments from the set. The cost of carrying instruments \textbf{$I_{k}$} is \textbf{$c_{k}$} dollars. And an instrument can be used for multiple experiments.

The professor's job is to determine which experiments to perform and which instruments to carry for a given flight in order to maximize the net revenue, which is the total income from the experiments performed minus the total cost of the instruments carried. Since he is not a programmer, he asked your help.

\section{Input}

Input starts with an integer \textbf{T (}\textbf{≤ 100)}, denoting the number of test cases.

Each case starts with a line containing two integers \textbf{m (1 ≤ m ≤ 1000)} and \textbf{n (1 ≤ n ≤ 1000), }where \textbf{m} denotes the number of experiments and \textbf{n} denotes the number of instruments. The next line contains \textbf{m} space separated integers, where the \textbf{$j^{th}$} integer denotes the commercial sponsor of \textbf{$E_{j}$} paying NASA \textbf{$p_{j}$(1 ≤ $p_{j}$ ≤ 10000)} dollars for the result of the experiment. The next line contains \textbf{n} space separated integers, where the \textbf{$k^{th}$} integer denotes the cost of carrying the \textbf{$k^{th}$} instrument, \textbf{$c_{k}$(1 ≤ $c_{k}$ ≤ 10000)}. Each of the next \textbf{m} lines contains an integer \textbf{$q_{i}$ (1 ≤ $q_{i}$ ≤ n)} followed by \textbf{$q_{i}$} distinct integers each between \textbf{1} and \textbf{n}, separated by spaces. These \textbf{$q_{i}$} integers denote the required instruments for the \textbf{$i^{th}$} experiment.

\section{Output}

For each case, print the case number and the maximum revenue NASA can make using the experiments.
\begin{verbatim}
\textbf{Sample Input}
2
1 1
10
20
1 1
3 5
20 30 40
1 2 30 4 50
3 1 2 3
3 2 3 4
1 5

\textbf{Sample Output}
Case 1: 0
Case 2: 13\end{verbatim}
