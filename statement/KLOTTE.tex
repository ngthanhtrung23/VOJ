

K luôn giúp đỡ những trẻ em cơ nhỡ bằng cách mua vé số. Trời thương K  nên cho K trúng độc đắc thường xuyên. Tuy nhiên thì "bất quá D" (không  phải bất quá tam) K chẳng bao giờ trúng số quá D lần cả. "Nhật kí trúng  số" của K là một dãy bit có độ dài N, 1 tức là trúng, 0 tức là không  trúng. K thấy rằng có rất nhiều dãy bit có tính chất giống như vậy và tự  hỏi: Nếu ta sắp xếp tất cả lại theo thứ tự từ điển, thì dãy liền sau  dãy "nhật kí trúng số" của mình là dãy nào nhỉ?

\subsubsection{Input}
\begin{itemize}
	\item Dòng thứ nhất: Hai số nguyên N (1 ≤ N ≤ 50) và D (0 ≤ D ≤ N).
	\item Dòng thứ hai: Dãy bit thể hiện "nhật ký trúng số" của K.
\end{itemize}

\subsubsection{Output}
\begin{itemize}
	\item Ghi ra dãy bit liền sau "nhật ký trúng số" của K. Nếu không có dãy nào, ghi ra -1.
\end{itemize}

\subsubsection{Example}
\begin{verbatim}
\textbf{Input 1:}
1 1
1

\textbf{Output 1:}
-1
\textbf{

Input 1:}\textbf{
}3 1
010
\textbf{
Output 1:}
100
\end{verbatim}
