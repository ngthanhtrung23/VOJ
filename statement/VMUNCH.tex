



   Bessie rất yêu bãi cỏ của mình và thích thú chạy về chuồng bò vào  giờ vắt sữa buổi tối.  

   Bessie đã chia đồng cỏ của mình là 1 vùng hình chữ nhật thành  các ô vuông nhỏ với R (1  $\le$  R  $\le$  100) hàng và C (1  $\le$  C  $\le$  100) cột,  đồng thời đánh dấu chỗ nào là cỏ và chỗ nào là đá. Bessie đứng ở vị  trí R\_b,C\_b và muốn ăn cỏ theo cách của mình, từng ô vuông một và  trở về chuồng ở ô 1,1 ; bên cạnh đó đường đi này phải là ngắn nhất.  

   Bessie có thể đi từ 1 ô vuông sang 4 ô vuông khác kề cạnh.  

   Dưới đây là một bản đồ ví dụ [với đá ('*'), cỏ ('.'),  chuồng bò ('B'),  và Bessie ('C') ở hàng 5, cột 6] và một bản đồ cho biết hành trình  tối ưu của Bessie, đường đi được dánh dấu bằng chữ ‘m’.  
\begin{verbatim}
           Bản đồ               Đường đi tối ưu
        1 2 3 4 5 6  $<$-cột      1 2 3 4 5 6  $<$-cột
      1 B . . . * .           1 B m m m * .
      2 . . * . . .           2 . . * m m m
      3 . * * . * .           3 . * * . * m
      4 . . * * * .           4 . . * * * m
      5 * . . * . C           5 * . . * . m

Bessie ăn được 9 ô cỏ.
\end{verbatim}

   Cho bản đồ, hãy tính xem có bao nhiêu ô cỏ mà Bessie sẽ ăn được trên  con đường ngắn nhất trở về chuồng (tất nhiên trong chuồng không có  cỏ đâu nên đừng có tính nhé)  

\subsubsection{   Dữ liệu  }
\begin{itemize}
	\item     Dòng 1: 2 số nguyên cách nhau bởi dấu cách: R và C   
	\item     Dòng 2..R+1: Dòng i+1 mô tả dòng i với C ký tự (và không có dấu          cách) như đã nói ở trên.   
\end{itemize}

\subsubsection{   Kết quả  }
\begin{itemize}
	\item     Dòng 1: Một số nguyên là số ô cỏ mà Bessie ăn được trên hành        trình ngắn nhất trở về chuồng.   
\end{itemize}

\subsubsection{   Ví dụ  }
\begin{verbatim}
Dữ liệu
5 6
B...*.
..*...
.**.*.
..***.
*..*.C

Kết quả
9
\end{verbatim}
