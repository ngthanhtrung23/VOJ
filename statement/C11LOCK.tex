

Ổ khóa nhà của yenthanh132 rất đặc biệt. Nó gồm 5 vòng số. Mỗi vòng có đúng n số, mỗi số có một giá trị nhất định. Cụ thể, vòng thứ i sẽ chứa n số: a(i,1), a(i,2), a(i,3),..., a(i, n).

Do vốn tính hay quên nên yenthanh132 không để một mật mã nhất định mà thay vào đó yenthanh132 đã yêu cầu người ta thiết kế ra một ổ khóa đặt biệt như sau: yenthanh132 sẽ chọn ra một số nguyên k. Để mở được ổ khóa, ta cần phải xoay các vòng số, sao cho tổng 5 số hiện trên 5 vòng số này bằng k.

\textbf{Yêu cầu: } Cho các giá trị trên 5 vòng số. Hãy giúp yenthanh132 đếm xem có bao nhiêu cách để mở ổ khóa của anh ta. Giả sử có 2 cách xoay để chọn các số trên vòng số là [a(1,i1), a(2,i2), a(3,i3), a(4,i4), a(5,i5)] và [a(1,j1), a(2,j2), a(3,j3), a(4,j4), a(5,j5)]. Hai cách đó được xem là khác nhau nếu: hoặc i1 ≠ j1, hoặc i2 ≠ j2, hoặc i3 ≠ j3, hoặc i4 ≠ j4, hoặc i5 ≠ j5. (xem ví dụ để hiểu rõ hơn).

\subsubsection{Dữ liệu}
\begin{itemize}
	\item Dòng đầu tiên chứa hai số nguyên n và k.
	\item Tiếp theo là 5 dòng, mỗi dòng n số nguyên, số thứ j trên dòng thứ i+1 là giá trị của a(i,j).
\end{itemize}

\subsubsection{Kết quả}
\begin{itemize}
	\item Một số nguyên duy nhất là số cách để mở ổ khóa của yenthanh132 .
\end{itemize}

\subsubsection{Giới hạn}
\begin{itemize}
	\item 1 ≤ n ≤ 500
	\item Trong 10\% test có n ≤ 20
	\item Trong 30\% test tiếp theo có n ≤ 100
	\item -$10^{9}$ ≤ a(i,j), k ≤ $10^{9}$
\end{itemize}

\subsubsection{Ví dụ}
\begin{itemize}
\end{itemize}
\begin{verbatim}
\textbf{Input 1:}
2 2
-2 -2
1 2
1 2
1 2
1 2 \end{verbatim}
\begin{verbatim}
\textbf{Output 1:}
2
\end{verbatim}
\begin{verbatim}
\textbf{Input 2:
}5 7
-3 -4 -6 5 -2
9 9 2 -3 -3
-3 -8 7 10 7
0 4 1 -4 1
5 2 -7 -9 3
\end{verbatim}
\begin{verbatim}
\textbf{Output 2:
}113\end{verbatim}
