

Sau khi thương lượng, cuối cùng Phú Ông quyết định sẽ cắt cho Bờm 2 khu đất trong quỹ đất của mình để đổi lấy chiếc quạt mo. Quỹ đất của phú ông có dạng R hàng ngang và C cột dọt gồm \textbf{R }* \textbf{C} thửa. Mỗi thửa có một giá trị kinh tế khác nhau (có thể âm).

Phú Ông đồng ý chia cho Bờm 2 khu đất, một khu \textbf{hình chữ nhật} kích thước \textbf{A }* \textbf{B} gồm \textbf{A} hàng ngang và \textbf{B }cột thửa đất nhỏ (lưu ý \textbf{A} * \textbf{B} khác \textbf{B} * \textbf{A}); và 1 khu đất \textbf{“hình thoi”} kích thước \textbf{K} sao cho:
\begin{itemize}
	\item Hai khu đất Bờm chọn không chứa thửa đất nào chung (\textbf{không giao nhau}).
	\item Hai khu đất Bờm chọn phải nằm \textbf{hoàn toàn trong} khu đất của Phú Ông.
\end{itemize}

Biết rằng khu đất hình thoi với tâm ở thửa (\textbf{x}, \textbf{y}) và có kích thước \textbf{K} sẽ là tập hợp các thửa (\textbf{x'}, \textbf{y'}) thoả mãn |\textbf{x’} - \textbf{x}| + |\textbf{y’} - \textbf{y}| ≤ \textbf{K}. Phú ông cho Bờm lựa chọn tuỳ ý, Bờm cũng bắt đầu giao động trước đề nghị này. Bạn hãy giúp Bờm xác định 2 khu đất thoả mãn sẽ có giá trị kinh tế \textbf{lớn nhất} là bao nhiêu.

\subsubsection{Input}

Dòng đầu tiên ghi 5 số \textbf{R}, \textbf{C}, \textbf{A}, \textbf{B}, \textbf{K}.

Tiếp theo là \textbf{R} dòng, mỗi dòng ghi \textbf{C} số nguyên thể hiện giá trị kinh tế của các thửa đất.

\subsubsection{Output}

In ra giá trị kinh tế lớn nhất của 2 khu đất mà Bờm có thể chọn. Nếu không có cách chọn 2 khu đất thoả mãn, in ra "\textbf{no solution}".

\subsubsection{Giới hạn}

Trong tất cả các test, giá trị kinh tế của các thửa đất có \textbf{trị tuyệt đối} không vượt quá 10$^6$.

Subtask 1 (30\% số điểm): \textbf{R}, \textbf{C} ≤ 50

Subtask 2 (40\% số điểm): \textbf{R}, \textbf{C} ≤ 200

Subtask 3 (30\% số điểm): \textbf{R}, \textbf{C} ≤ 1000

\subsubsection{Ví dụ}
\begin{verbatim}
\textbf{Input 1:}

5 5 2 2 1

1 1 1 1 1

1 2 2 2 1

1 2 2 2 1

1 2 2 2 1

1 1 1 1 1

\end{verbatim}
\begin{verbatim}
\textbf{Output 1:}

16

\end{verbatim}
\begin{verbatim}
\textbf{Input 2:}

5 5 2 2 3

1 1 1 1 1

1 2 2 2 1

1 2 2 2 1

1 2 2 2 1

1 1 1 1 12 2

\end2 2
\begin{verbatim}
\textbf{Output 2:}

no solution

\end{verbatim}

\subsubsection{Giải thích}

Ở bộ test đầu tiên Bờm sẽ hời nhất nếu chọn 2 khu đất như sau:

1 \textbf{1} 1 1 1

\textbf{1}\textbf{2}\textbf{2} 2 1

1 \textbf{2}\emph{2 2} 1

1 2 \emph{2 2} 1

1 1 1 1 1

Ở bộ test thứ hai Bờm không thể chọn khu đất "hình thoi" kích thước là 3 nào nằm hoàn toàn ở trong khu đất của Phú Ông.
