

Bảo mới học lập trình nên thuật toán sắp xếp còn chưa vững. Vì vậy thầy giáo Khoa đã cho Bảo bài tập về nhà như sau: Cho dãy A gồm N số thực. Hãy sắp xếp lại dãy A theo thứ tự tăng dần.

\subsubsection{Dữ liệu vào}
\begin{itemize}
	\item Dòng đầu tiên chứ số nguyên dương N. (N $<$= 10\textasciicircum5)
	\item N dòng tiếp theo dòng thứ i chứ phần tử Ai (|Ai| $<$= 10\textasciicircum5)
\end{itemize}

\subsubsection{Dữ liệu ra}
\begin{itemize}
	\item Gồm N dòng mô tả dãy A sau khi được sắp xếp.
\end{itemize}

\subsubsection{Ví dụ}

\textbf{Dữ liệu vào:}

\textbf{}\textbf{8}

\textbf{}

 

8 

\textbf{10946.530106 }

\textbf{16744.196379 }

\textbf{28853.130725 }

\textbf{19876.514697 }

\textbf{23427.387959 }

\textbf{24736.081291 }

\textbf{2603.070486 }

\textbf{1843.131915}\textbf{
}

\textbf{Dữ liệu ra:}

\textbf{
}\textbf{\textbf{1843.131915
}}\textbf{\textbf{2603.070486
}}\textbf{\textbf{10946.530106
}}\textbf{\textbf{16744.196379
}}\textbf{\textbf{19876.514697
}}\textbf{\textbf{23427.387959
}}\textbf{\textbf{24736.081291
}}\textbf{\textbf{28853.130725}}\textbf{
}
