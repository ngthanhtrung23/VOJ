

Trong thế giới hiện tại, bạn gặp rất nhiều cuộc gọi. Bạn cần phải nhớ hết những số đó. Một phương pháp để nhớ dễ hơn là thay các con số bằng các kí tự như bên dưới:

1: ij     

2: abc 

3: def

4: gh

5: kl     

6: mn 

7: prs 

8: tuv 

9: wxy

0: oqz

Bằng cách này mỗi từ hay một nhóm từ chỉ thay thế cho 1 số duy nhất, vì thế bạn có thể nhớ từ này thay vì nhớ số.

Cho một dãy số điện thoại, viết chương trình tìm dãy từ ngắn nhất (tức là số lượng từ tạo nên dãy từ đó là ít nhất), dãy từ đó khi ghép các từ lại với nhau theo thứ tự thì phải phù hợp với dãy số điện thoại đã cho (theo quy tắc đã nêu ở trên) và được tạo nên bởi các từ trong danh sách từ gợi ý được cho.

\textbf{Input:}

- Dòng đầu tiên chứa dãy số điện thoại (tối đa 100 chữ số)

- Dòng tiếp theo là số N (N$<$=50000) là số lượng từ gợi ý.

- N dòng tiếp theo mỗi dòng chứa một từ gồm các chữ cái thường (‘a’ – ‘z’) và dài tối đa 50 kí tự. Tổng độ dài các từ trong danh sách gợi ý là 300000 kí tự.

\textbf{Output:}

- Nếu không có dãy xâu nào phù hợp thì in ra dòng chữ  “No solution.”

- Ngược lại in ra số lượng các từ trong dãy con tim được.

 

\textbf{Ví dụ:}

\textbf{Input:}

7325189087

5

it

your

reality

real

our

 

\textbf{Output:}

2

 

\textbf{Input:}

4294967296

5

it

your

reality

real

our

 

\textbf{Output:}

No solution.