

Thiếu niên thì thích chụp ảnh và xử lý ảnh - Nam cũng vậy. Nhưng Nam học tin học nên
cô giáo chỉ cho nam xử lý ảnh đen trắng hai màu thôi (tại cô thích bit). Cô giao cho Nam
một bảng hình chữ nhật kích thước N x M chỉ gồm hai số 0 và 1, mỗi ô của bảng là 1
pixel của ảnh. 0 ứng với màu đen và 1 ứng với màu trắng, bảng có ít nhất một ô màu
trắng. Khoảng cách giữa hai pixel ở ô (i,j) và ô (p,q) là |i-p| + |j-q|. Cô muốn Nam cho cô
biết, với mỗi ô màu đen, khoảng cách từ nó đến ô màu trắng gần nhất là bao nhiêu.
Giới hạn : 1 ≤ N, M ≤ 200.
Dữ liệu vào ghi trong file BITMAP.INP, dòng đầu ghi hai số N và M. N dòng tiếp theo,
mỗi dòng ghi M số chỉ nhận hai giá trị 0 và 1.
Dữ liệu ra ghi trong file BITMAP.OUT gồm N dòng, mỗi dòng M số. Mỗi số ở hàng i cột
j là khoảng cách từ đó đến ô trắng gần nhất.
Ví dụ:

BITMAP.INP 
3 4
0 0 0 1
0 0 1 1
0 1 1 0

BITMAP.OUT

3 2 1 0
2 1 0 0
1 0 0 1
