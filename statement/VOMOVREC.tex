

Cho \textbf{N} hình chữ nhật trên mặt phẳng Oxy. Các hình chữ nhật này có toạ độ nguyên và có các cạnh song song với trục toạ độ. Ở mỗi lượt, các hình chữ nhật có thể đứng yên hoặc di chuyển theo \textbf{8} hướng:
\begin{itemize}
	\item sang trái \textbf{1} đơn vị
	\item sang phải \textbf{1} đơn vị
	\item lên trên \textbf{1} đơn vị
	\item xuống dưới \textbf{1} đơn vị
	\item sang trái và lên trên \textbf{1} đơn vị
	\item sang trái và xuống dưới \textbf{1} đơn vị
	\item sang phải và lên trên \textbf{1} đơn vị
	\item sang phải và xuống dưới \textbf{1} đơn vị
\end{itemize}

Hãy xác định sau ít nhất bao nhiên lượt thì \textbf{N} hình chữ nhật ban đầu sẽ được di chuyển đến các vị trí mới sao cho phần diện tích giao nhau của tất cả \textbf{N} hình chữ nhật này lớn hơn hoặc bằng 1.

\subsubsection{Dữ liệu vào}

Dòng đầu tiên ghi số \textbf{N} là số lượng các hình chữ nhật.

Tiếp theo là \textbf{N} dòng, mỗi dòng ghi \textbf{4} số nguyên \textbf{x$_1$}, \textbf{y$_1$}, \textbf{x$_2$}, \textbf{y$_2$} thể hiện hình chữ nhật có góc trái dưới là (\textbf{x$_1$}, \textbf{y$_1$}) góc phải trên là (\textbf{x$_2$}, \textbf{y$_2$}).

\subsubsection{Dữ liệu ra}

In ra số lượt di chuyển tối thiểu.

\subsubsection{Giới hạn}

Subtask 1 (30\% số điểm)
\begin{itemize}
	\item 2 ≤ \textbf{N} ≤ 200
	\item |tọa độ| ≤ 100
	\item Chỉ gồm các \textbf{hình vuông đơn vị} với cạnh là 1
\end{itemize}

Subtask 2 (40\% số điểm)
\begin{itemize}
	\item 200 $<$ \textbf{N} ≤ 10$^5$
	\item |toạ độ| ≤ 2 * 10$^9$
	\item Chỉ gồm các \textbf{hình vuông đơn vị} với cạnh là 1
\end{itemize}

Subtask 3 (30\% số điểm)
\begin{itemize}
	\item 200 $<$ \textbf{N} ≤ 10$^5$
	\item |tọa độ| ≤ 2 * 10$^9$
\end{itemize}

\subsubsection{Ví dụ}
\begin{verbatim}
\textbf{Input:}

3

0 0 1 1

0 0 2 3

2 3 4 5\end{verbatim}
\begin{verbatim}
\textbf{Output:}

2\end{verbatim}

\subsubsection{Giải thích}

 
\includegraphics{https://lh4.googleusercontent.com/l_6LZm5gTEbzskqR6Zox4Jbv-SI-ce-IhgZz7jO7Xwz19xjpSZ4f-yOMqwea3K9EvBLx2Au5Lf_jR0yLgq8sQx3Y5-T_E1YGWarlXXNyDP2pBHW36aBsP42IzYdQKAPDxGaue48P}

Sau 2 lượt:
\begin{itemize}
	\item Hình 1: Di chuyển lên trên \textbf{1} đơn vị rồi sau đó di chuyển chéo lên phải \textbf{1} đơn vị.
	\item Hình 2: Đứng yên.
	\item Hình 3: Di chuyển chéo xuống trái \textbf{1} đơn vị.
\end{itemize}