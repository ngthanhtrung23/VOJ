

Truyền thuyết kể lại rằng, sau một trận động đất kinh hoàng, đất nước C11 bấy giờ trở thành một đảo quốc xinh đẹp giữa biển khơi. Để nhanh chóng đo đạc, quản lí lãnh thổ, quy hoạch đất nước ... trong tình hình mới, nhà vua đã cho xây dựng quanh đảo các cột mốc đánh dấu. Kì lạ thay, đất nước C11 mới lại có hình dạng của một hình chữ nhật khổng lồ kích thước m x n. Do đó, nhà vua quyết định sẽ cắm và đánh số các cộc mốc theo chiều kim đồng hồ từ 1 đến 2 x (m + n), chia đảo thành một lưới m x n ô vuông.

Tưởng đâu những tai họa đã trôi qua, thật không may, k cơn dư chấn lại tiếp tục tàn phá đảo. Truyền thuyết cũng có đoạn nói rằng, mỗi cơn dư chấn i đã tạo một rãnh sâu thẳng kéo dài từ cột mốc a$_i$ đến cột mốc b$_i$ và vùng đất trên đảo bị chia cắt. Trường hợp a$_i$ = b$_i$, đảo chỉ bị rung nhẹ và không bị chia cắt. Các trận dư chấn có thể trùng nhau. Cuối cùng, đất nước C11 bị chia thành p đảo khác nhau.

Dựa theo truyền thuyết trên, các nhà khảo cổ học đang cố gắng khám phá về đất nước C11 xưa cùng một nền văn minh đã bị che lấp theo thời gian. Là một người yêu thích môn Lịch sử, bạn hãy giúp các nhà sử học xác định số p trên để họ có thể chọn được những khu vực thích hợp để khai quật, nghiên cứu.

\subsubsection{Dữ liệu}
\begin{itemize}
	\item Dòng 1 chứa ba số nguyên dương m, n, k ≤ 200.
	\item k dòng tiếp theo mỗi dòng chứa hai số nguyên dương a$_i$ và b$_i$.
\end{itemize}

\subsubsection{Kết quả}

Một số nguyên duy nhất là số p tìm được.

\subsubsection{Ví dụ}


\includegraphics{../../../content/tohuuquan:C11PF.jpg}
\begin{verbatim}
\textbf{Input 1:}

4 4
3 8
3 12
14 6
\\13 7



\textbf{Output 1:}

end{verbatim}
egin{verbatim}
extbf{Input 2:}

4 5
\\3 8
\\3 12
\\14 6
\\13 7
\\8 3



\textbf{Output 2:}

9\end{verbatim}

\subsubsection{Giới hạn}
\begin{itemize}
	\item Có khoảng 15\% tổng số test là test loại A: k ≤ 2
	\item Có khoảng 40\% tổng số test là test loại B: k ≤ 10
	\item Có khoảng 15\% tổng số test là test loại C: m, n ≤ 5
	\item Có khoảng 50\% tổng số test là test loại D: m, n ≤ 20
	\item Có 50\% tổng số test thuộc 1 trong 4 loại A, B, C, D kể trên.
\end{itemize}
