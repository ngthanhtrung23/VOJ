

 DNA là thành phần cơbản cấu tạo thành bộgenome của sinh vật. DNA bao gồm 4 loại khác nhau là \{A,C,G,T\}. Đểnghiên cứu các sinh vật ởmức độphân tử, người ta tiến hành giải mã bộgenome của chúng. Đểgiải mã bộgenome của một sinh vật, máy giải mã thếhệmới sẽsinh ra N đoạn cơsở,mỗi đoạn cơsởlà một dãy bao gồm 30 DNA. Các đoạn cơsởsẽ được ghép nối với nhau đểtạo thành một bộgenome hoàn chỉnh. Ta nói một đoạn DNA X được bao phủbởi một đoạn cơsở Ynếu tồn tại một đoạn liên tiếp trong Ytrùng với X. Giảsửklà sốnguyên dương, một đoạn DNA X được gọi là đoạn tin tưởng cấp knếu X được bao phủbởi ít nhất k đoạn cơsở. Yêu cầu:Cho N đoạn cơsởvà sốnguyên dương k, hãy tìm đoạn tin tưởng cấp kcó độdài lớn nhất. Dữliệu: Vào từfile văn bản GENOME.INP có cấu trúc nhưsau:   Dòng đầu chứa hai sốnguyên dương Nvà k(0 $<$ k ≤ N ≤30000);   Mỗi dòng trong Ndòng tiếp theo chứa một đoạn cơsở.Kết quả: Đưa ra file văn bản GENOME.OUT một sốnguyên là độdài của đoạn tin tưởng tìm được. (Ghi số 1 nếu không tồn tại đoạn tin tưởng cấp k). Ví dụ: GENOME.INP GENOME.OUT 4 3 AAAAAAAAATAAAATAAAAAAAAAAAAATGAAAAAAAAAAAAAAAAAAAATAAATGAAAAAAAAAAAAAAAAAAAAAAATGAAAAAAAAAAAAAAAAAAAAAATGAAAAAAAGGGGAAAA15  Lưu ý:50\% sốtest có N ≤1000 tương ứng với 50\% tổng số điểm dành cho bài.

DNA là thành phần cơ bản cấu tạo thành bộgenome của sinh vật. DNA bao gồm 4 loại khác 

nhau là \{A,C,G,T\}. Đểnghiên cứu các sinh vật ở mức độphân tử, người ta tiến hành giải mã 

bộ genome của chúng. 

Đểgiải mã bộ genome của một sinh vật, máy giải mã thế hệmới sẽsinh ra N đoạn cơ sở,mỗi 

đoạn cơ sởlà một dãy bao gồm 30 DNA. Các đoạn cơ sởsẽ được ghép nối với nhau đểtạo 

thành một bộgenome hoàn chỉnh. 

Ta nói một đoạn DNA X được bao phủ bởi một đoạn cơ sở Ynếu tồn tại một đoạn liên tiếp 

trong Ytrùng với X. Giả sử k là sốnguyên dương, một đoạn DNA X được gọi là đoạn tin 

tưởng cấp k nếu X được bao phủ bởi ít nhất k đoạn cơ sở. 

Yêu cầu:Cho N đoạn cơ sởvà sốnguyên dương k, hãy tìm đoạn tin tưởng cấp k có độ dài lớn 

nhất. 

Dữliệu: Vào từ file văn bản GENOME.INP có cấu trúc nhưsau: 

  Dòng đầu chứa hai sốnguyên dương N và k(0 $<$ k ≤ N ≤30000); 

  Mỗi dòng trong Ndòng tiếp theo chứa một đoạn cơ sở.

Kết quả: Đưa ra file văn bản GENOME.OUT một số nguyên là độ dài của đoạn tin tưởng 

tìm được. (Ghi số -1 nếu không tồn tại đoạn tin tưởng cấp k). 

Ví dụ: 

GENOME.INP 

4 3 

AAAAAAAAATAAAATAAAAAAAAAAAAATG

AAAAAAAAAAAAAAAAAAAATAAATGAAAA

AAAAAAAAAAAAAAAAAAATGAAAAAAAAA

AAAAAAAAAAAAATGAAAAAAAGGGGAAAA

GENOME.OUT

15 

 

Lưu ý:50\% sốtest có N ≤1000 tương ứng với 50\% tổng số điểm dành cho bài.

 