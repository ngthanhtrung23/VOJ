

Nhà thường xuyên có trộm nên K phải đổi mã khóa từng ngày. Mã khóa là một dãy bit có độ dài N. Trong một giai đoạn 2$^N$, quy tắt đổi mã khóa của K như sau:

1. Bắt đầu từ ngày 1, mã khóa là dãy bit toàn số 0.

2. Mỗi ngày K sẽ sử dụng một mã khóa khác nhau.

3. Sau mỗi ngày, K chọn ra một số các bit trong mã khóa có giá trị giống nhau và đảo các bit này, 0 thành 1 hoặc 1 thành 0.

4. Trong các dãy bit có thể tạo ra từ quy tắc 3, K sẽ chọn dãy có thứ tự từ điển nhỏ nhất để làm mã khóa cho ngày hôm sau.

Giả sử bạn là tên trộm và biết được các quy luật trên. Nhưng bạn chỉ rảnh rỗi vào ngày T vì lịch ăn trộm đã kín hết rồi. Hãy tìm ra mã khóa để có thể vơ hết của nhà K trong ngày hôm đó.

\subsubsection{Input}
\begin{itemize}
	\item Hai số nguyên N (1 ≤ N ≤ 50) và T (1 ≤ T ≤ 2$^N$).
\end{itemize}

\subsubsection{Output}
\begin{itemize}
	\item Dãy bit gồm các ký tự '0' hoặc '1' viết liền nhau mô tả mã khóa. 
\end{itemize}

\subsubsection{Example}
\begin{verbatim}
\textbf{Input:}
2 4 
\textbf{Output:}
10

Giải thích: Với N = 2, các mã khóa lần lượt được sử dụng là: 00, 01, 11, 10. \end{verbatim}
