

Xét đơn đồ thị vô hướng G = (V, E) có n (1  $\le$  n  $\le$  10000) đỉnh và m (1  $\le$  m  $\le$  50000) cạnh. Người ta định nghĩa một đỉnh gọi là khớp nếu như xoá đỉnh đó sẽ làm tăng số thành phần liên thông của đồ thị. Tương tự như vậy, một cạnh được gọi là cầu nếu xoá cạnh đó sẽ làm tăng số thành phần liên thông của đồ thị.

Vấn đề đặt ra là cần phải đếm tất cả các khớp và cầu của đồ thị G.

\subsubsection{Input}

Dòng đầu: chứa hai số tự nhiên n,m.

M dòng sau mỗi dòng chứa một cặp số (u,v) (u khác v, 1  $\le$  u  $\le$  n, 1  $\le$  v $<$ n) mô tả một cạnh của G.

\subsubsection{Output}

Gồm một dòng duy nhất ghi hai số, số thứ nhất là số khớp, số thứ hai là số cầu của G

\subsubsection{Example
\\
\includegraphics{http://i433.photobucket.com/albums/qq53/canhtoannguyen/NewBitmapImage.jpg}}

 
\begin{verbatim}
\textbf{Input:}
10 12
1 10
10 2
10 3
2 4
4 5
5 2
3 6
6 7
7 3
7 8
8 9
9 7

\textbf{Output:}
4 3\end{verbatim}

 

 

 

 
