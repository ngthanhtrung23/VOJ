

Hồi nhỏ, Mirko thích chơi "Bắn tàu" nhưng bây giờ anh ta chơi trò "Câu cá trên sông" “Sea battle”.

Trò chơi mô tả trên 1 bảng N ô đánh số từ 1 đến N từ trái qua phải. Trên đó sẽ đặt M tàu.  Với mỗi ô sẽ biết số lượng cá mà ở trong ô đó. Mỗi tàu sẽ chiếm 1 số ô liên tiếp và nó phải thả neo vào 1 ô nào đó.  Nghĩa là ta sẽ biết được với mỗi tàu, ô mà tàu đó bắt buộc phải chiếm.

Chỉ có thể có 1 tàu trên mỗi một ô. Lượng cá bắt được là tổng lượng cá nằm trong ô mà tàu này chiếm.  Cần bắt được nhiều cá nhất.

Bạn hãy giúp Mirko đặt tàu.

 

\subsubsection{Input}

 

Dòng đầu là số N, số ô, 1 ≤ N ≤ 100000.

Dòng tiếp theo là N số nguyên mô tả khối lượng cá trong từng ô, mỗi số $>$=1 và  $\le$ 100.

Dòng tiếp theo là số tàu M,  1 ≤ M ≤ N.

M dòng tiếp theo, mỗi dòng gồm 2 số B và D, nghĩa là tàu phải thả neo ở ô B và tàu có độ dài là D ô.

 

\subsubsection{Output}

 

Khối lượng cá lớn nhất bắt được.

 

\subsubsection{Sample}
\begin{verbatim}
brodovi.in 
 
11 
2 5 3 4 7 6 2 1 3 8 5 
2 
8 3 
3 2 
 
brodovi.out 
 
20 

brodovi.in 
 
13 
3 2 4 7 2 1 3 6 1 2 6 4 1 
2 
5 7 
11 4 
 
brodovi.out 
 
38

brodovi.in 
 
11 
1 1 6 4 4 1 1 3 10 1 1 
3 
2 3 
6 4 
10 2 
 
brodovi.out 
 
31 
\end{verbatim}

 
