
\begin{verbatim}


Biết Nam mới học quy hoạch động nên cô giáo cũng muốn biết là kĩ năng của Nam ở môn này như nào; Nam có chịu đọc sách bổ sung kiến thức hay không hay là Nam lại dùng vài vòng for thuần túy để giải. Tất nhiên, như thế còn tốt hơn là việc Nam nộp tờ giấy trắng. Yêu cầu của cô trông khá đơn giản (tại tính cô cũng đơn giản):

 

Cho một dãy N số a$_1$, a$_2­, $..., a­$_N$. Xác định xem giá trị lớn nhất có thể có của biểu thức

a$_i$ - 2*a$_j$ + 3*a$_k$ với 1 ≤ i $<$ j $<$ k ≤ N.

 

Giới hạn : 3 ≤ N ≤ 10^5, |a$_i$| ≤ 10^9, i = 1..N.

 

Dữ liệu vào trong file TONGLN.INP: Dòng đầu là 2 số nguyên dương N.  Dòng thứ hai là N số nguyên a$_i$. Dữ liệu thỏa mãn điều kiện đề bài.

 

Dữ liệu ra ghi trong file TONGLN.OUT gồm một số duy nhất là giá trị lớn nhất tìm được của biểu thức đề bài.

 
\begin{tabular}\hline 


\textbf{\textbf{TONGLN}.INP} & 

\textbf{\textbf{TONGLN}.OUT}  
\hline


\textbf{5  }

\textbf{1 2 3 4 5 } & 

\textbf{12}  
\hline


\textbf{7}

\textbf{7 10 3 4 -5 2 9 } & 

\textbf{47}  
\hline

\end{tabular}

 

Giải thích: Ở ví dụ 1, bộ ba chỉ số để biểu thức lớn nhất có giá trị là 12 là bộ ba (1,2,5). Ở ví dụ 2 là bộ ba chỉ số (2,5,7) và biểu thức có giá trị tương ứng là 47.\end{verbatim}
