

Cho một dãy số A (-10\textasciicircum9 $<$= Ai $<$= 10\textasciicircum9) gồm N phần tử (1 $<$= N $<$= 10\textasciicircum5), và một số nguyên K (1 $<$= K $<$= N*(N-1) / 2)

Xét các tổng các đoạn con (gồm các phần tử liên tiếp) của dãy A.

VD A = 1 4 2, có các đoạn con là (1), (4), (2), (1 4), (4 2), (1 4 2), có các tổng là 1, 4, 2, 5, 6, 7

Tìm tổng lớn thứ K

\subsubsection{Input}

Dòng 1: N và K

Dòng 2: Dãy A

\subsubsection{Output}

Số nguyên duy nhất là tổng lớn thứ K

\subsubsection{Example}
\begin{verbatim}
\textbf{Input:}

3 4

1 4 2



\textbf{Output:}

4

\end{verbatim}
\begin{verbatim}
\textbf{Input:}

4 6

2 -1 2 -1

\textbf{Output:}

1

\end{verbatim}