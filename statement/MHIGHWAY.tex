

 Để nắm tình hình giao thông trên đường cao tốc mới xây dựng người ta đã tiến hành đo đạc thống kêkhoảng các trung bình giữa các phương tiện tham gia giao thông trên toàn tuyến vào giờ cao điểm vànhận được dãy số nguyên d1, d2, . . ., dn, trong đó dilà khoảng cách trung bình giữa các phương tiệngiao thông trên đoạn đường thứ i.Hai đoạn đường ivà jcó tình trạng giao thông giống nhau bao nhiêu thì độ lệch  h= |di– dj| càngnhỏ bấy nhiêu. Hãy tính độ lệch của hai đoạn đường có tình trạng giao thông giống nhau nhất.Dữ liệu:Vào từ file văn bản HIGHWAY.INP:• Dòng đầu tiên chứa số nguyên n(1 $<$n≤ 106),• Dòng thứ 2 chứa nsố nguyên d1, d2, . . ., dn(1 ≤ di≤ 109, i= 1 ÷ n). Các số trên một dòngghi cách nhau một dấu cách.Kết quả:Đưa ra file văn bản HIGHWAY.OUT một số nguyên – độ lệch tìm được.Ví dụ:HIGHWAY.INP HIGHWAY.OUT612 4 6 9 7 141

Để nắm tình hình giao thông trên đường cao tốc mới xây dựng người ta đã tiến hành đo đạc thống kê

khoảng các trung bình giữa các phương tiện tham gia giao thông trên toàn tuyến vào giờ cao điểm và

nhận được dãy số nguyên d1 , d2 , . . ., dn , trong đó di là khoảng cách trung bình giữa các phương tiện

giao thông trên đoạn đường thứ i. Hai đoạn đường ivà jcó tình trạng giao thông giống nhau bao nhiêu thì độ lệch  h= |di– dj| càng

nhỏ bấy nhiêu. Hãy tính độ lệch của hai đoạn đường có tình trạng giao thông giống nhau nhất.

Dữ liệu:Vào từ file văn bản HIGHWAY.INP:

• Dòng đầu tiên chứa số nguyên n(1 $<$n≤ 10\textasciicircum6),

• Dòng thứ 2 chứa n số nguyên d1 , d2, , . . ., dn (1 ≤ di≤ 10\textasciicircum9, i= 1 ÷ n). Các số trên một dòngghi cách nhau một dấu cách.

Kết quả:Đưa ra file văn bản HIGHWAY.OUT một số nguyên – độ lệch tìm được.

Ví dụ:

HIGHWAY.INP 

6

12 4 6 9 7 14

HIGHWAY.OUT

1

 