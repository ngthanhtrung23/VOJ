



   Cho một bàn cờ vuông kích thước N*N, trên đó có một số ô cấm. Bạn cần đặt N quân xe lên đấy sao cho:  
\begin{itemize}
	\item     Trên mỗi ô của bàn cờ, chỉ có tối đa 1 quân xe   
	\item     Không có quân xe nào được đặt lên các ô cấm   
	\item     Không có 2 quân xe nào có thể ăn được nhau (cùng hàng hoặc cùng cột). Chú ý rằng các quân xe có thể đi xuyên qua ô cấm.   
\end{itemize}

   Đếm số cách đặt N quân xe lên bảng, theo modulo 2 (mod 2).  

\subsubsection{   Input  }
\begin{itemize}
	\item     Dòng 1: Số nguyên dương T - số bộ test.   
	\item     Tiếp theo là T bộ test, mỗi bộ test gồm:    
\begin{itemize}
	\item       Dòng 1: Số nguyên dương N     
	\item       Tiếp theo là N dòng, dòng thứ i gồm N số nguyên. Số nguyên thứ j trên dòng i là 0 nếu ô tương ứng là ô cấm và bằng 1 trong trường hợp ngược lại.     
\end{itemize}
\end{itemize}

\subsubsection{   Output  }

   Gồm T dòng, mỗi dòng là một số nguyên duy nhất là số cách đặt quân xe theo modulo 2.  

\subsubsection{   Giới hạn  }
\begin{itemize}
	\item     1 ≤ T ≤ 10   
	\item     1 ≤ N ≤ 250   
	\item     Trong 20\% số test, N ≤ 15   
\end{itemize}

\subsubsection{   Chấm điểm  }
\begin{itemize}
	\item     Bài của bạn sẽ được chấm trên thang điểm 100. Điểm mà bạn nhận được sẽ tương ứng với \% test mà bạn giải đúng.   
	\item     Trong quá trình thi, bài của bạn sẽ chỉ được chấm với 1 test ví dụ có trong đề bài.   
	\item     Khi vòng thi kết thúc, bài của bạn sẽ được chấm với bộ test đầy đủ.   
\end{itemize}
\begin{itemize}
\end{itemize}

\subsubsection{   Example  }
\begin{verbatim}
\textbf{Input:}
2
3
1 0 1
0 1 1
1 1 1
2
1 0
0 1

\textbf{Output:}
1
1
\end{verbatim}

\subsubsection{   Giải thích  }
\begin{itemize}
	\item     Trong test đầu tiên, có 3 cách xếp. Bạn cần in ra 3 mod 2 = 1   
	\item     Trong test thứ hai, có duy nhất 1 cách xếp. Bạn cần in ra 1 mod 2 = 1   
\end{itemize}