

Cho một xâu độ dài N chỉ gồm các kí tự ‘(‘ và ‘)’, các kí tự được đánh số từ 1 đến N theo chiều từ trái qua phải.

Một dãy ngoặc đúng được định nghĩa như sau:
\begin{itemize}
	\item Xâu rỗng là 1 dãy ngoặc đúng.
	\item Nếu A là 1 dãy ngoặc đúng thì (A) là 1 dãy ngoặc đúng.
	\item Nếu A và B là 2 dãy ngoặc đúng thì AB là 1 dãy ngoặc đúng.
\end{itemize}

Cho M truy vấn, mỗi truy vấn thuộc 1 trong 2 loại sau:
\begin{itemize}
	\item 0 i ch: thay đổi kí tự ở vị trí i của xâu kí tự thành kí tự ch.
	\item 1 i j: in ra 1 nếu xâu con từ vị trí i đến vị trí j là một dãy ngoặc đúng, in ra 0 trong trường hợp ngược lại.
\end{itemize}

 

\textbf{Giới hạn:}

      2  $\le$  N  $\le$  100000

      1  $\le$  M  $\le$  200000

      Trong truy vấn loại 1:    1  $\le$  i  $\le$  N;              ch là ‘(‘ hoặc ‘)’

      Trong truy vấn loại 2:    1  $\le$  i  $\le$  j  $\le$  N;

 

\textbf{Input:}
\begin{itemize}
	\item Dòng đầu tiên chứa 2 số N, M
	\item Dòng tiếp theo chứa N kí tự liên tiếp.
	\item M dòng tiếp theo, mỗi dòng chứa 1 truy vấn thuộc 1 trong 2 loại trên.
\end{itemize}

\textbf{Output:}

In ra 0 hoặc 1 tương ứng với mỗi truy vấn loại 2.

 
\begin{verbatim}
\textbf{Ví dụ:}\textbf{Input}
8 7
()))(())
1 1 2
1 3 4
0 3 (
1 1 4
1 5 8
0 6 )
1 5 8

\textbf{Output}
10110\end{verbatim}
