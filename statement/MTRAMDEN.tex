
\begin{verbatim}


Hiên,  một  huyện  miền  núi  phía  tây  Quảng  Nam  cũng  có trám,  tuy  không  nhiều  như ở  Bắc Kạn.  Các  bạn  Sinh  viên  Tình  nguyện  Mùa  hè  xanh  thấy  hột  trám  vương  vãi  quanh  trường khá nhiều, đã nảy ra sáng kiến “trám hóa” sân trường. Có   hạt trám được thu thập về. Sân trường  có  hình  chữ  nhật.  Bằng     đường  cách  đều  nhau  song  song  với  một  cạnh  của  sân trường và   đường cách đều nhau song song với cạnh kia  của sân trường toàn bộ sân được chia thành  các hình chữ  nhật con  giống nhau  ( 1$<$=m$<$=n ). Các hột trám sẽ được  chặt đôi.

 

Sau  khi  ăn  nhân  bên  trong  học  sinh  sẽ  đóng  nửa  hạt  này  xuống  sân  tại  các  điểm  giao  nhau giữa  các  được  kẻ  và  ở  tâm  điểm  các  hình  chữ  nhật  con.  Tại  mỗi  điểm  chỉ  đóng  nửa  hạt trám. Để  không  lãng  phí  số  hạt  trám  đã  thu  nhặt  và  hạt  trám  được  đóng  phân  bố  đều  trên sân các bạn sinh viên quyết định chọn  m và n   sao cho số hạt trám sẽ được dùng hết và hiệu n- m là nhỏ  nhất.

 


\includegraphics{../../../../../../content/simes:MTRAMDEN.jpg}

Yêu cầu:  Cho  số  nguyên k ,  hãy  xác  định   m và  n.  Nếu  không  tồn  tại  m và n   thỏa mãn  thì  đưa ra hai số -1.

Dữ liệu: Vào từ file văn bản CANARIUM.INP:

Dòng đầu tiên chứa số nguyên   T $<$= 20  là số bộ dữ liệu,

Mỗi bộ dữ liệu cho trên một dòng chứa một số nguyên dương  k ( 1$<$=k $<$=10^12).

 

Kết  quả:  Đưa  ra  file  văn  bản  CANARIUM.OUT,  kết  quả  mỗi  bộ  dữ  liệu  đưa  ra  trên  một dòng gồm 2 số nguyên  m và n  (có thể là -1 -1), hai số cách nhau một dấu cách.

 

Ví dụ:

 

CANARIUM.INP 

2

9

6

CANARIUM.OUT

2 3

-1 -1\end{verbatim}