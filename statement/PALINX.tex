



   Một xâu được gọi là đối xứng nếu đọc từ trái qua phải và đọc từ phải qua trái đều giống nhau.   


   Ví dụ xâu "aba", "abba" là xâu đối xứng; còn xâu "abc", "abca" thì không.   


   Bạn được cho N xâu, như vậy sẽ có NxN cặp xâu. Bạn hãy đếm xem trong NxN cặp xâu này, có bao nhiêu cặp mà khi nối xâu thứ hai vào sau xâu thứ nhất sẽ cho ra một xâu đối xứng.  

\subsubsection{   Input  }

   Dòng đầu ghi một số N. N dòng sau mỗi dòng mô tả một xâu, bắt đầu là độ dài của xâu, sau đó là một dấu cách và tiếp theo là nội dung của xâu.(Xâu chỉ gồm các chữ cái latin thường và có độ dài nguyên dương)   


   Dữ liệu vào luôn đảm bảo tổng độ dài các xâu không quá 1000000.  

\subsubsection{   Output  }

   Ghi ra một số duy nhất là số cặp xâu tìm được.  

\subsubsection{   Example  }
\begin{verbatim}
Input:
3
1 a
2 ab
2 ba

Output:
5
\end{verbatim}
