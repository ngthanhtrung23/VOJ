



   Cho dãy A gồm n số nguyên $A_{1}$   , $A_{2}$   ,… , $A_{n}$   và số nguyên dương k. Hỏi có bao nhiêu cách chia dãy A thành k đoạn liên tiếp có tổng bằng nhau (mỗi đoạn có ít nhất 1 phần tử) ?  

\subsubsection{   Input  }

   -          Dòng đầu : hai số nguyên dương n và k, cách nhau một khoảng trắng  

   -          Dòng hai : n số nguyên $A_{1}$   , $A_{2}$   ,… , $A_{n}$   , mỗi số cách nhau một khoảng trắng  

\subsubsection{   Output  }

   -          Số nguyên S là số cách chia thỏa yêu cầu đề bài. Do kết quả có thể rất lớn, bạn chỉ cần in ra S mod 1000000007 ($10^{9}$   + 7)  

\subsubsection{   Giới hạn :  }

   -          1 ≤ k ≤ n ≤ $10^{6}$

   -          |$A_{i}$   | ≤ $10^{9}$   (1 ≤ i ≤ n)  

\subsubsection{   Ví dụ :  }
\begin{verbatim}
\textbf{Input:}

8 3

-2 6 -1 3 -2 4 5 -1\textbf{Output:}
2\end{verbatim}
