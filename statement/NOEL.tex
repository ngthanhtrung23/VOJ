

Ông già Noel cần vào n ngôi nhà để phát quà. Các ngôi nhà này được biểu diễn trên hệ tọa độ Oxy, ngôi nhà thứ i có tọa độ (x$_i$, y$_i$) với x$_i$, y$_i$ là 2 số nguyên dương. Từ ngôi nhà thứ i đến ngôi nhà thứ j, ông già Noel mất một khoảng thời gian bằng |x$_i$ - x$_j$| + |y$_i$ - y$_j$|. Do số lượng ngôi nhà quá lớn, để tránh nhầm lẫn, ông già Noel chỉ đi đến ngôi nhà có hoành độ không nhỏ hơn hoành độ của những ngôi nhà đến trước đó. Nói cách khác, nếu x$_i$, x$_j$ lần lượt là hoành độ của ngôi nhà được đến thăm thứ i và thứ j và i $<$ j thì x$_i$ $<$= x$_j$.

Biết rằng ông già Noel có thể xuất phát từ một ngôi nhà bất kỳ, bạn cần tính thời gian ngắn nhất để có thể phát quà cho n ngôi nhà này.

\subsubsection{Input}
\begin{itemize}
	\item Dòng 1: N - số lượng ngôi nhà mà ông già Noel cần đến phát quà
	\item N dòng tiếp theo, dòng thứ i gồm 2 số nguyên không âm x, y là tọa độ của ngôi nhà thứ i, không có 2 ngôi nhà nào có cùng tọa độ
\end{itemize}

\subsubsection{Output}
\begin{itemize}
	\item Một số duy nhất là thời gian nhỏ nhất để phát quà hết cho N ngôi nhà này
\end{itemize}

\subsubsection{Ví dụ}
\begin{verbatim}
\textbf{Input:}


1 2
3 1
2 3

\textbf{Output:}
5

\\\end{verbatim}

\subsubsection{Giới hạn}
\begin{itemize}
	\item Trong tất cả các test, n $<$= 10$^5$, tọa độ của các điểm $<$= 10$^6$
	\item 80\% số test đầu tiên, không có 2 ngôi nhà nào có cùng hoành độ
\end{itemize}
\begin{verbatim}

\\\end{verbatim}
