



   Nền phẳng của 1 công trình xây dựng được chia thành lưới ô vuông đơn vị kích thước  MxN ô. Trên mỗi ô (i,j) của lưới, người ta dựng 1 cột bê tông hình hộp có đáy là ô (i,j) và chiều cao là h[i,j] đơn vị. Sau khi dựng xong thì trời đổ mưa to và đủ lâu. Nhà thầu xây dựng muốn tính lượng nước đọng lại giữa các cột để có kế hoạch thi công tiếp theo. Giả thiết, nước ko thẩm thấu qua các cột bê tông cũng như ko rò rỉ qua các đường ghép giữa chúng.  

   Nhiệm vụ của bạn là giúp nhà thầu tính toán lượng nước đọng lại giữa các cột.  

\subsubsection{   Input  }

   Dòng đầu tiên ghi 2 số nguyên dương M và N  

   Dòng thứ i trong M dòng tiếp theo, ghi N số nguyên dương h[i,1],h[i,2]...h[i,N].  

\subsubsection{   Output  }

   1 dòng duy nhất chứa số đơn vị khối nước đọng lại.  



\subsubsection{   Example  }
\begin{verbatim}
\textbf{Input:}


5 5


9 9 9 9 9


9 2 2 2 9


9 2 5 2 9


9 2 2 2 9


9 9 9 9 9





\textbf{Output:}


60





\textbf{Giới hạn:     }1 $\le$ M,N $\le$ 100,1 $\le$ H[i,j] $\le$ 1000


\end{verbatim}
