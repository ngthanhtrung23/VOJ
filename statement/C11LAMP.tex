

       Hôm nay trên lớp Lion\_IT được học về chất bán dẫn và rất thích thú với chiếc đèn led. Lion\_IT quyết định sẽ làm 1 hệ thống đèn led với m bóng mỗi bóng có 1 công tắc điều khiển. Khi hoàn thành hệ thống Lion\_IT quyết định thử xem độ ổn định của hệ thống đến đâu bằng cách liên tục đóng mở 1 số công tắt để chuyển dàn đèn từ trạng thái này sang trạng thái khác. Lion\_IT đã viết n  trạng thái mình muốn thử ra giấy và tự hỏi : ” Không biết thời gian ít nhất để mình có thể kiểm tra hết các trạng thái này là bao nhiêu nhỉ “.

       \textbf{Yêu cầu} : Cho biết số lượng bóng của dàn đèn và các trạng thái Lion\_IT muốn thử. Hãy xác định thời gian ít nhất để kiểm tra hết các trạng thái. Biết mỗi lần bật/tắt 1 công tắc mất 1 đơn vị thời gian và ban đầu tất cả các bóng đều tắt.

\subsubsection{Input:}

         Dòng đầu tiên gồm 2 số nguyên dương n, m (1$<$= n $<$=15, 1 $<$= m $<$= 10 000). n dòng sau mỗi dòng gồm m kí tự thể hiện 1 trạng thái Lion\_IT muốn thử mỗi kí tự là 0/1 thể hiện đèn tắt/bật.

\subsubsection{Output:}

         1 dòng duy nhất là kết quả của bài toán.

\subsubsection{Example}
\begin{verbatim}
\textbf{Input:}


3 3

101

010

111
\textbf{Output:}



\\

\textbf{Chú ý} : có 40% số test  1$<$=n$<$=10; 1$<$=m$<$=20\end{verbatim}
