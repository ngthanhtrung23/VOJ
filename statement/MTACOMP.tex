

Các xâu (chuỗi các chữ số) được tạo ra như sau:

Xâu đầu tiên có 1 ký tự: “1”. Các xâu tiếp theo nhận được bằng cách: các ký tự đầu tiên là số thứ tự của xâu đó (không có 0 đứng đầu), sau đó là xâu trước được lặp lại hai lần. Ví dụ 4 xâu đầu tiên:

1

211

3211211

432112113211211

 

Cho 2 số N, K. Hãy xác định chữ số nằm ở vị trí K của xâu thứ N.

 

 

\subsubsection{Input}

2 số nguyên N, K: 1 ≤ N ≤ 100000, 1 ≤ K ≤ 10\textasciicircum15.

\subsubsection{Output}

Đưa ra chữ số cần tìm. Trong trường hợp K lớn hơn độ dài của xâu N, đưa ra -1

\subsubsection{Example}
\begin{verbatim}
\textbf{Input:}

11 4



\textbf{Output:}

0\end{verbatim}