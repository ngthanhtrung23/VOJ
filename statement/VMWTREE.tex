

Bạn được cho một đồ thị liên thông có V đỉnh và V-1 cạnh, với mỗi đỉnh sẽ có một trọng số tương ứng (các đỉnh khác nhau có thể có trọng số bằng nhau). Lần này, w có một số thao tác trên cây muốn nhờ bạn xử lí như sau:
\begin{itemize}
	\item 1 u v: hỏi trọng số bé nhất trong số các đỉnh trên đường đi từ u đến v.
	\item 2 u v: hỏi trọng số lớn nhất trong số các đỉnh trên đường đi từ u đến v.
	\item 3 u v: đảo ngược trọng số các đỉnh trên đường đi từ u đến v.
\end{itemize}

Giả sử các đỉnh trên đường đi từ u đến v lần lượt là: u, u $_ 1 $ , u $_ 2, $ u $_ 3,... $ u $_ k $$_ , $ v.

Trọng số tương ứng là: w $_ u, $ w $_ u1, $ w $_ u2, $ w $_ u3,... $ w $_ uk, $ w $_ v. $

Thì sau khi thực hiện thao tác loại 3 trọng số các đỉnh sẽ trở thành w $_ v, $ w $_ uk,... $ w $_ u2, $ w $_ u1, $ w $_ u $ .

\subsubsection{Input}

Dòng đầu tiên chứa 2 số tự nhiên V và Q - số đỉnh của đồ thị và số truy vấn.

Dòng tiếp theo chứa V số tự nhiên là trọng số của các đỉnh 1, 2, 3,..., V.

V-1 dòng tiếp theo mỗi dòng chứa 2 số tự nhiên u v - có cạnh nối từ u đến v.

Q dòng tiếp theo mỗi dòng là một trong 3 loại truy vấn nêu trên.

\subsubsection{Giới hạn}
\begin{itemize}
	\item Trong tất cả các test: V,Q ≤ 10 $^ 5 $ , trọng số các đỉnh ≤ 10 $^ 9 $ .
	\item Trong 20\% các test: V,Q ≤ 10 $^ 3 $ .
	\item Trong 40\% các test tiếp theo: số lượng truy vấn loại "3 u v" ≤ 100.
\end{itemize}

\subsubsection{Output}

Với mỗi truy vấn loại 1 và 2 in ra một dòng tương ứng chứa kết quả truy vấn đó.

\subsubsection{Example}
\begin{verbatim}
\textbf{\textbf{Input}}
8 4
2 4 1 4 7 5 2 3
1 2
1 3
2 4
2 5
3 6
3 7
3 8
2 6 4
1 4 1
3 7 2
1 6 8
\end{verbatim}
\begin{verbatim}
\textbf{\textbf{Output}}
5
2
2\end{verbatim}