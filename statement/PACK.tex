

 
% [if !mso]> <mce:style><!   v\:* {behavior:url(#default#VML);} o\:* {behavior:url(#default#VML);} w\:* {behavior:url(#default#VML);} .shape {behavior:url(#default#VML);}  


 
% [endif] 

% [if gte mso 9]><xml> <w:worddocument> <w:view>Normal</w:view> <w:zoom>0</w:zoom> <w:punctuationkerning /> <w:validateagainstschemas /> <w:saveifxmlinvalid>false</w:saveifxmlinvalid> <w:ignoremixedcontent>false</w:ignoremixedcontent> <w:alwaysshowplaceholdertext>false</w:alwaysshowplaceholdertext> <w:compatibility> <w:breakwrappedtables /> <w:snaptogridincell /> <w:wraptextwithpunct /> <w:useasianbreakrules /> <w:dontgrowautofit /> </w:compatibility> </w:worddocument> </xml><![endif]

% [if gte mso 9]><xml> <w:latentstyles DefLockedState="false" LatentStyleCount="156"> </w:latentstyles> </xml><![endif]



%    /* Font Definitions */  @font-face 	{font-family:Calibri; 	mso-font-alt:"Century Gothic"; 	mso-font-charset:0; 	mso-generic-font-family:swiss; 	mso-font-pitch:variable; 	mso-font-signature:-1610611985 1073750139 0 0 159 0;}  /* Style Definitions */  p.MsoNormal, li.MsoNormal, div.MsoNormal 	{mso-style-parent:""; 	margin-top:0in; 	margin-right:0in; 	margin-bottom:10.0pt; 	margin-left:0in; 	line-height:115%; 	mso-pagination:widow-orphan; 	font-size:11.0pt; 	font-family:Calibri; 	mso-fareast-font-family:Calibri; 	mso-bidi-font-family:"Times New Roman";} p.ListParagraph, li.ListParagraph, div.ListParagraph 	{mso-style-name:"List Paragraph"; 	margin-top:0in; 	margin-right:0in; 	margin-bottom:10.0pt; 	margin-left:.5in; 	mso-add-space:auto; 	line-height:115%; 	mso-pagination:widow-orphan; 	font-size:11.0pt; 	font-family:Calibri; 	mso-fareast-font-family:Calibri; 	mso-bidi-font-family:"Times New Roman";} p.ListParagraphCxSpFirst, li.ListParagraphCxSpFirst, div.ListParagraphCxSpFirst 	{mso-style-name:"List ParagraphCxSpFirst"; 	mso-style-type:export-only; 	margin-top:0in; 	margin-right:0in; 	margin-bottom:0in; 	margin-left:.5in; 	margin-bottom:.0001pt; 	mso-add-space:auto; 	line-height:115%; 	mso-pagination:widow-orphan; 	font-size:11.0pt; 	font-family:Calibri; 	mso-fareast-font-family:Calibri; 	mso-bidi-font-family:"Times New Roman";} p.ListParagraphCxSpMiddle, li.ListParagraphCxSpMiddle, div.ListParagraphCxSpMiddle 	{mso-style-name:"List ParagraphCxSpMiddle"; 	mso-style-type:export-only; 	margin-top:0in; 	margin-right:0in; 	margin-bottom:0in; 	margin-left:.5in; 	margin-bottom:.0001pt; 	mso-add-space:auto; 	line-height:115%; 	mso-pagination:widow-orphan; 	font-size:11.0pt; 	font-family:Calibri; 	mso-fareast-font-family:Calibri; 	mso-bidi-font-family:"Times New Roman";} p.ListParagraphCxSpLast, li.ListParagraphCxSpLast, div.ListParagraphCxSpLast 	{mso-style-name:"List ParagraphCxSpLast"; 	mso-style-type:export-only; 	margin-top:0in; 	margin-right:0in; 	margin-bottom:10.0pt; 	margin-left:.5in; 	mso-add-space:auto; 	line-height:115%; 	mso-pagination:widow-orphan; 	font-size:11.0pt; 	font-family:Calibri; 	mso-fareast-font-family:Calibri; 	mso-bidi-font-family:"Times New Roman";} @page Section1 	{size:8.5in 11.0in; 	margin:1.0in 1.0in 1.0in 1.0in; 	mso-header-margin:.5in; 	mso-footer-margin:.5in; 	mso-paper-source:0;} div.Section1 	{page:Section1;} 

% [if gte mso 10]> <mce:style><!    /* Style Definitions */  table.MsoNormalTable 	{mso-style-name:"Table Normal"; 	mso-tstyle-rowband-size:0; 	mso-tstyle-colband-size:0; 	mso-style-noshow:yes; 	mso-style-parent:""; 	mso-padding-alt:0in 5.4pt 0in 5.4pt; 	mso-para-margin:0in; 	mso-para-margin-bottom:.0001pt; 	mso-pagination:widow-orphan; 	font-size:10.0pt; 	font-family:"Times New Roman"; 	mso-ansi-language:#0400; 	mso-fareast-language:#0400; 	mso-bidi-language:#0400;}  


 
% [endif] 




Conan là 1 tên trùm bài bạc, anh ta đố các bạn 1 bài sau. Có 1 bộ bài đặt trên bàn, chứa N lá bài chồng lên nhau từng lá một. Mỗi lá có 1 con số nguyên dương được viết lên 1 mặt và ko viết bên mặt còn lại. Trên lá bài trên cùng con số 1 được viết, lá bài trên cùng thứ hai có số 2…, và trên lá bài dưới cùng là số N. Khi bắt đầu tất cả các lá bài được sắp xếp để các mặt có số nằm phía trên. Người sử dụng các lá bài thực hiện M lượt công việc. Trong lượt thứ i thì ông lấy K[i] lá phía trên, giữ chúng lại và lật úp lại và để lên phía trên của bộ bài. Nhiệm vụ của bạn là viết 1 chương trình xác định vị trí và trạng thái của 1 lá bài nhất định (lật úp hay mở) trong bộ bài sau M lượt của người sử dụng các lá bài.



\textbf{+Input:}

\textbf{}

- Dòng đầu tiên của file PACK1.INP chứa 2 số nguyên dương – N (số lượng bài trong bộ bài. N $<$=100000) và M (số lượt xáo bài của người đó, $<$=1000) và phải tách bởi dấu gạch ngang. M dòng tiếp theo sẽ viết như sau: 1 số nguyên dương Ki (1$<$=Ki$<$=N) - số lượng lá bài dung trong mỗi lượt.



- Dòng đầu tiên của file chữ PACK2.INP chứa 1 số nguyên dương S ($<$=10000) - số lượng lá bài, ví trí sau lượt cuối cùng và trang thái cần được xác định. S dòng tiếp theo chứa 1 con số nguyên dương – con số ghi trên lá bài, ví trí sau lượt cuối cùng và trạng thái. 



\textbf{+Output :}



File PACK.OUT phải chứa đúng S dòng. Mỗi dòng phải chứa 1 số nguyên dương. Nếu đến cuối cùng mà lá thứ P có con số mà dòng thứ i+1 của file PACK2.INP có chưa, dòng thứ i của file output phải có con số:

+P nếu lá có con số của nó viết ở mặt trên

-P là ngược lại



\textbf{Chú ý:}



Thứ tự của các lá bài trong bộ bài đang thay đổi (↑= mặt có ghi số lật ngửa, ↓ mặt có ghi số lật úp)




% [if gte vml 1]><v:shapetype  id="_x0000_t75" coordsize="21600,21600" o:spt="75" o:preferrelative="t"  path="m@4@5l@4@11@9@11@9@5xe" filled="f" stroked="f"> <v:stroke joinstyle="miter" /> <v:formulas> <v:f eqn="if lineDrawn pixelLineWidth 0" /> <v:f eqn="sum @0 1 0" /> <v:f eqn="sum 0 0 @1" /> <v:f eqn="prod @2 1 2" /> <v:f eqn="prod @3 21600 pixelWidth" /> <v:f eqn="prod @3 21600 pixelHeight" /> <v:f eqn="sum @0 0 1" /> <v:f eqn="prod @6 1 2" /> <v:f eqn="prod @7 21600 pixelWidth" /> <v:f eqn="sum @8 21600 0" /> <v:f eqn="prod @7 21600 pixelHeight" /> <v:f eqn="sum @10 21600 0" /> </v:formulas> <v:path o:extrusionok="f" gradientshapeok="t" o:connecttype="rect" /> <o:lock v:ext="edit" aspectratio="t" /> </v:shapetype><v:shape id="_x0000_i1025" type="#_x0000_t75" style='width:268.5pt;  height:154.5pt'> <v:imagedata src="file:///C:\DOCUME~1\LOVELY~1\LOCALS~1\Temp\msohtml1\01\clip_image001.jpg" mce_src="file:///C:\DOCUME~1\LOVELY~1\LOCALS~1\Temp\msohtml1\01\clip_image001.jpg"   o:title="pack" /> </v:shape><![endif]

% [if !vml]



\href{http://s259.photobucket.com/albums/hh288/phongtinptnk0710/?action=view¤t=pack.jpg}{
\includegraphics{http://i259.photobucket.com/albums/hh288/phongtinptnk0710/pack.jpg}}

\textbf{Ex:}

\textbf{PACK1.INP}

 8 3
 1
 8
 4

\textbf{PACK2.INP}

 5
 4
 8
 1
 5
 2

\textbf{PACK.OUT}

 -5
 +4
 +8
 +1
 -7
