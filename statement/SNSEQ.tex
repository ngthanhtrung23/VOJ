



   Cho một dãy số $A_{1}$   .. $A_{N}$   theo công thức sau $A_{i}$   = ($A_{i−1}$   + $A_{i+1}$   ) / 2 − 1 với mọi 1 $<$ i $<$ N và $A_{i}$   $>$= 0 với mọi 1  $\le$  i  $\le$  N. Biết N và $A_{1}$   , tìm giá trị nhỏ nhất có thể của $A_{N}$   .  

\subsubsection{   Dữ liệu  }

   Dòng duy nhất ghi số nguyên N và số thực $A_{1}$   (3  $\le$  N  $\le$  1000, 10  $\le$  $A_{1}$    $\le$  1000).  

\subsubsection{   Kết qủa  }

   Ghi giá trị nhỏ nhất của $A_{N}$   có thể có, chính xác đến 2 chữ số sau dấu phẩy.  
\begin{verbatim}
\textbf{Dữ liệu:} 
692 532.81
\textbf{Kết qủa} 
446113.34 
\end{verbatim}
