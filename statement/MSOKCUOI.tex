
\begin{verbatim}
Cho xâu ký tự  S có N ký tự  chứa các chữ  cái hoa từ  „A‟ đến „Z‟ (N ≤ 10000). Nếu hoán vị xâu này thì ta sẽ được các xâu khác nhau.Ví dụ: S = “BABB” ta sẽ có 4 xâu khác nhau ABBB, BABB, BBAB và BBBA.Tuy nhiên, số các xâu được tạo ra này là rất lớn. Hãy đếm số lượng chữ số 0 tận cùng của số lượng các xâu được tạo ra từ xâu đã cho.Input  Dòng đầu ghi số nguyên dương   (         )     dòng tiếp theo, mỗi dòng ghi một xâu kí tự chữ cái hoa.Output  Mỗi dòng chứa một số nguyên là số lượng chữ số 0 tìm được tương ứng.Input2BABBABBCDOutput01
\begin{verbatim}
Cho xâu ký tự  S có N ký tự  chứa các chữ  cái hoa từ  „A‟ đến „Z‟ (N ≤ 10000). Nếu hoán \end{verbatim}
\begin{verbatim}
vị xâu này thì ta sẽ được các xâu khác nhau.\end{verbatim}
\begin{verbatim}
Ví dụ: S = “BABB” ta sẽ có 4 xâu khác nhau ABBB, BABB, BBAB và BBBA.\end{verbatim}
\begin{verbatim}
Tuy nhiên, số các xâu được tạo ra này là rất lớn. Hãy đếm số lượng chữ số 0 tận cùng của \end{verbatim}
\begin{verbatim}
số lượng các xâu được tạo ra từ xâu đã cho.\end{verbatim}
\begin{verbatim}
Input\end{verbatim}
\begin{verbatim}
  Dòng đầu ghi số nguyên dương T  (     T$<$=10    ) \end{verbatim}
\begin{verbatim}
    dòng tiếp theo, mỗi dòng ghi một xâu kí tự chữ cái hoa có N kí tự.\end{verbatim}
\begin{verbatim}
Output\end{verbatim}
\begin{verbatim}
  Mỗi dòng chứa một số nguyên là số lượng chữ số 0 tìm được tương ứng.\end{verbatim}
\begin{verbatim}
Input\end{verbatim}
\begin{verbatim}
2\end{verbatim}
\begin{verbatim}
BABB\end{verbatim}
\begin{verbatim}
ABBCD\end{verbatim}
\begin{verbatim}
Output\end{verbatim}
\begin{verbatim}
0\end{verbatim}
\begin{verbatim}
1\end{verbatim}\end{verbatim}