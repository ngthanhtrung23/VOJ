

RobotCam là cuộc thi lớn được tổ chức thường niên ở hành tinh C11. Sân chơi có thể mô tả như hệ trục tọa độ Oxy.

Luật chơi đc mô tả như sao :

Trên mặt đất đặt n phần quà tại các điểm phân biệt với nhau, các đội thi phải điều khiển các con robot của mình để thu nhặt tất cả các phần quà.

Nhưng các con robot không đc di chuyển tùy ý của mình mà phải tuân thủ 4 quy tắc sau
\begin{enumerate}
	\item Đường đi của robot phải bắt đầu và kết      thúc ở các điểm trong n điểm đã cho
	\item Trong quá trình di chuyển robot không được      đi tới điểm có hoành độ hay tung độ nhỏ hơn vị trí đang đứng
	\item Hai đường đi của 2 robots khác nhau không      được có điểm chung
	\item Đường đi chỉ gồm 1 điểm cũng đc coi là      đường đi hợp lệ
\end{enumerate}

Sau đây là một ví dụ ( sr vẽ tay :D )


\includegraphics{http://ng6.upanh.com/b2.s23.d4/0e498d595ed6eb04ef3f176e205c6c63_38861876.untitled.bmp}

YÊU CẦU : xác định số lượng robot ít nhất cần dùng để nhặt hết n món quà

\subsubsection{Input}

Dòng 1 : N ( n $<$= 10\textasciicircum5 )

N dòng tiếp theo : ghi hoành độ và tung độ món quà thứ i, trị tuyệt đối $<$= 10\textasciicircum9

 

\subsubsection{Output}

Số lượng robot ít nhất cần dùng

\subsubsection{Example}
\begin{verbatim}
\textbf{Input:}

6

1 1

2 1

1 2

4 2

5 3

4 4\textbf{Output:}

2

\end{verbatim}