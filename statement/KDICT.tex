

Một quyển từ điển của bộ lạc NK được tìm thấy. Thật kì lạ là chỉ với hai ký tự 'a' và 'z', họ tạo ra rất nhiều từ khác nhau. Ví dụ: "aaazz" nghĩa là "thi học kì", "aaa" nghĩa là "kiểm tra lại". Quyển từ điển ngay lập tức được phổ biến trong giới học thuật và người ta nhận ra rằng ngôn ngữ NK rất hay và mang tính biểu cảm rất cao. Tuy nhiên thì có một bất tiện là một từ của ngôn ngữ có thể rất dài. Vì thế, người ta mã hóa một từ bằng 3 thông số N (số ký tự 'a'), M (số ký tự 'z') và T (thứ tự của từ đó trong quyển từ điển nếu chỉ xét tập các xâu có N ký tự 'a' và M ký tự 'z'). Nhưng như vậy cũng đòi hỏi phải có một chương trình giải mã, và nhiệm vụ của bạn là tạo ra nó.

\subsubsection{Input}
\begin{itemize}
	\item Ba số nguyên N, M (đều không quá 100) và T (không quá 10$^9$). 
\end{itemize}

\subsubsection{Output}
\begin{itemize}
	\item N + M ký tự mô tả từ được giải mã. Nếu không giải mã được, in ra -1. 
\end{itemize}

\subsubsection{Example}
\begin{verbatim}
\textbf{Input 1:}
2 2 2

\textbf{Output 1:}
azaz



\textbf{Input 2:}
2 2 6

\textbf{Ouput 2:}
zzaa
\end{verbatim}
