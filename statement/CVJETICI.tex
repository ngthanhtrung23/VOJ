



   On a faraway planet, strange plants with two stems can be found. Every plant on the planet can be described by three numbers: the x-coordinates of the stems L and R, and the height H at which the stems are connect. The image depicts a plant with L=2, R=5 and H=4.  


\includegraphics{http://vn.spoj.pl/content/cv1.jpg}

   Every day a new plant grows on the planet. The plant that grows on day 1 is of height 1, and every subsequent plant is one higher than the previous one.  

   When a stem of a new plant intersects the horizontal segment of another plant, a small flower grows (if one wasn't there already). If segments merely touch in a point, a flower will not grow there. The following images are a visualization of the first example.  


\includegraphics{http://vn.spoj.pl/content/cv2.jpg}
\includegraphics{http://vn.spoj.pl/content/cv3.jpg}
\includegraphics{http://vn.spoj.pl/content/cv4.jpg}
\includegraphics{http://vn.spoj.pl/content/cv5.jpg}

   Write a program that, given the coordinates of all plants, calculates the number of new flower every day.  

\subsubsection{   Input  }

   The first line contains an integer N (1 ≤ N ≤ 100 000), the number of days.  

   Each of the following N lines contains two integers L and R (1 ≤ L $<$ R ≤ 100 000), the coordinates of the stems of a plant.  

\subsubsection{   Output  }

   Output N lines, the number of new flowers after each plant grows.  

\subsubsection{   Example  }
\begin{verbatim}
Input
4
1 4
3 7
1 6
2 6

Output
0
1
1
2


Input
5
1 3
3 5
3 9
2 4
3 8

Output
0
0
0
3
2
\end{verbatim}