

Trong giờ học thống kê và ghi nhớ của trường tiểu học SuperKids, cô giáo đưa cho các bé một bảng hình chữ nhật kích thước m × n được chia làm lưới ô vuông đơn vị. Các hàng của bảng được đánh số từ 1 tới m và các cột được đánh số từ 1 tới n. Ô nằm trên giao của hàng i và cột j gọi là ô (i, j) và trên ô đó ghi số nguyên aij.

Nhiệm vụ của các bé là phải tìm một hình chữ nhật có diện tích lớn nhất thỏa mãn các yêu cầu sau:
\begin{itemize}
	\item Cạnh hình chữ nhật song song với cạnh bảng và hình chữ nhật chứa trọn một số ô của bảng.
	\item Các số ghi trong các ô thuộc hình chữ nhật được chọn phải hoàn toàn phân biệt (không có số nào xuất hiện nhiều hơn 1 lần)
\end{itemize}

Các bé đã thực hiện rất nhanh yêu cầu nhưng cô giáo vẫn loay hoay chưa tìm ra đáp án để chấm cho các bé, hãy giúp cô giáo tìm đáp án trên bảng đã cho.

\subsubsection{Input}
\begin{itemize}
	\item Dòng 1 chứa hai số nguyên dương m, n ≤ 400
	\item m dòng tiếp theo, dòng thứ i chứa n số nguyên dương, số thứ j là aij ≤ 10$^6$
	\item Các số trên một dòng của input file được ghi cách nhau bởi dấu cách
\end{itemize}

\subsubsection{Output}
\begin{itemize}
	\item Một số nguyên duy nhất là diện tích hình chữ nhật được chọn theo phương án tìm được.
\end{itemize}

\subsubsection{Example}
\begin{verbatim}
\textbf{Input:}
\begin{tabular}

3 3  


1 3 1  


4 5 6  


2 6 1
\end{tabular}\textbf{Output:}

6\end{verbatim}

\textbf{ Chú ý: }
\begin{itemize}
	\item 2/7 điểm ứng với các tests có n ≤ 20 
	\item 2/7 điểm ứng với các tests có 20 $<$ n ≤ 100
	\item 3/7 điểm ứng với các tests có 100 $<$ n ≤ 400
\end{itemize}
