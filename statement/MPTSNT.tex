

Nam đã làm quen với khái niệm số nguyên tố và đã học cách sinh ra các số nguyên tố bằng sàng số nguyên tố. Cô giáo muốn kiểm tra kĩ năng viết code cơ bản và kiến thức toán học cơ bản của Nam bằng cách yêu cầu Nam in ra các ước nguyên tố có bậc lớn hơn không và bậc của chúng của biểu thức  C(k,n) = n!/(k!*(n-k)!)

 

Giới hạn : 1 ≤ N ≤ 10\textasciicircum5, 0 ≤ K ≤ N.

 

Dữ liệu vào trong file PTSNT.INP gồm 2 số N K trên một dòng cách nhau ít nhất bởi một dấu cách.

 

Dữ liệu ra ghi trong file PTSNT.OUT gồm các ước nguyên tố p của C(K,N) có số mũ lớn hơn 0 và số mũ tương ứng q. Mỗi cặp nguyên tố p và số mũ q được viết trên một dòng và cách nhau bởi ít nhất một dấu cách. Các số p được viết theo thứ tự tăng dần.

 

\textbf{Ví dụ:}

\textbf{ }
\begin{tabular}\hline 


\textbf{PTSNT.INP} & 

\textbf{PTSNT.OUT}  
\hline


\textbf{5 3} & 

\textbf{2 1}

\textbf{5 1}  
\hline


\textbf{10 4} & 

\textbf{2 1}

\textbf{3 1}

\textbf{5 1}

\textbf{7 1}  
\hline


\textbf{100 1} & 

\textbf{2 2}

\textbf{5 2}  
\hline

\end{tabular}

\textbf{ }

Giải thích: với ví dụ thứ nhất N = 5 và K = 3, C(K,N) = 10 = 2\textasciicircum1 * 5\textasciicircum1. Đáp số là 2 1 và 5 1 ở hai dòng khác nhau. 
