
\begin{verbatim}



\textbf{SỐ SERI}

 

Số \textbf{seri} được sinh ra lần lượt theo thứ tự \textbf{tăng dần} (1, 2, 3, …). Tuy nhiên có một số xâu \textbf{không được} phép xuất hiện như \textbf{một chuỗi con liên tiếp} trong số seri. Do đó, dãy số seri sẽ không còn là các số nguyên liên tiếp tăng dần nữa.

 

Ví dụ, nếu chuỗi con ‘23’ không được xuất hiện trong số seri thì dãy số seri sẽ có dạng như sau :

1, 2, 3, …, 21, 22, 24, 25, …, 121, 122, 124, 125, … 228, 229, 240, 241, 242, …

(thiếu các số 23, 123, 230, 231, …)

 

Dãy số seri được đánh số lần lượt, bắt đầu từ 1. Cho một \textbf{tập các xâu} không được xuất hiện trong số seri, và chỉ số i, hãy \textbf{in ra số seri thứ i} trong dãy.

 

\textbf{Input}

Dòng đầu chứa số nguyên \textbf{T} (≤ 100) là số bộ test.

 

Mỗi test bắt đầu bằng một  dòng chứa số nguyên \textbf{K} (1 ≤ K ≤ 10) thể hiện số lượng xâu không được xuất hiện. Theo sau là \textbf{K} xâu, mỗi xâu có \textbf{độ dài} trong khoảng từ \textbf{1 đến 10}, và chỉ chứa các chữ số từ \textbf{0-9} (có thể có các chữ số 0 ở đầu xâu).

 

Dòng thứ hai của test chứa số nguyên \textbf{N} (1 ≤ N ≤ 100) là số lượng truy vấn. Theo sau là N truy vấn, mỗi truy vấn là một số nguyên \textbf{i} thể hiện vị trí của số seri cần tìm.

 

\textbf{Output}

Với mỗi bộ test, in ra \textbf{N số seri} tương ứng với các vị trí trong truy vấn. Các số seri đảm bảo nằm trong khoảng số nguyên \textbf{64 bit} có dấu.

 

\textbf{Ví dụ}
\begin{tabular}\hline 


Input & 

Output  
\hline


4

1 4

2 3 5

1 2

3 4 5 6

2 4 13

3 4 13 888

3 012 345 6789

2 12345 67890 & 

3 6

5 6 7

5 16 1230

12391 68273  
\hline

\end{tabular}

 \end{verbatim}
