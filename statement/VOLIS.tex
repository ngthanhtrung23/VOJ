



   Cho 1 dãy số nguyên a[1], a[2], …, a[N]. Bạn có thể cộng hoặc trừ mỗi phần tử đi một lượng tối đa là D (dãy số sau khi cộng/trừ có thể xuất hiện số âm hoặc số thực). Hãy tìm độ dài của dây con không giảm dài nhất của dãy số, nếu cộng/trừ các phân tử một cách tối ưu.  

Lưu ý: dãy con của dãy số là dãy số a[x[1]], a[x[2]], ..., a[x[k]] (k là một số nguyên) sao cho 1 ≤ x[1] $<$ x[2] $<$ ... $<$ x[k] ≤ N.

\subsubsection{   Input  }
\begin{itemize}
	\item     Dòng đầu tiên chứa 2 số nguyên N và D.   
	\item     Dòng thứ 2 chứa N số nguyên a[1], a[2], …, a[N], các số được cách nhau bởi ít nhất 1 dấu cách.   
\end{itemize}



\subsubsection{   Output  }
\begin{itemize}
	\item     Một dòng duy nhất là độ dài dãy con không giảm dài nhất trong phương án tối ưu.   
\end{itemize}

\subsubsection{   Giới hạn  }
\begin{itemize}
	\item     0 $<$ N ≤ 1000.   
	\item     0 ≤ D ≤ 10^9.   
	\item     0 ≤ a[i] ≤ 10^9.   
	\item     Trong 30\% số test, D = 0.   
\end{itemize}

\subsubsection{   Example  }
\begin{verbatim}
\textbf{Input:}

4 1
6 4 3 2

\textbf{Output:}

3\emph{Giải thích: Có thể biến đổi dãy thành 6 3 3 3.
\\}
\\\end{verbatim}
