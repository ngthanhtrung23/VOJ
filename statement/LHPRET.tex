

Để tránh mùa đông khắc nghiệt, nhiều động vật chọn giải pháp ngủ đông trong hang. Khi mùa xuân tới, chúng tỉnh dậy và ra khỏi hang kiếm ăn.

Rết cũng thuộc loài ngủ đông. Hang của nó giống như một mê cung và có thể biểu diễn dưới dạng lưới M*N ô vuông ( M hàng, N cột, các hàng được đánh số từ trên xuống dưới, các cột được đánh số từ trái sang phải, bắt đầu từ 1). Một số ô có đá, những ô còn lại là ô trống và tạo thành một miền liên thông. Cửa ra của hang ở ô (1, 1). Rết chỉ có thể bò ở những ô trống. Rết có L đốt: B1, B2, . . ., BL, mỗi đốt có kích thước đúng bằng một ô. Vị trí ngủ của nó trong hang được đánh dấu bởi dãy ô (r1,c1), (r2, c2), . . ., (rL, cL), ô (ri, ci) là vị trí của đốt Bi. Đầu rết ở vị trí (r1, c1). Khi bò, đầu rết chuyển động tới ô trống ( không có đá hay một đốt nào đó của rết, đốt thứ i+1 sẽ chuyển tới vị trí của đốt i trước đó. Rết được coi là ra được khỏi hang, khi đầu của nó tới được ô (1, 1).
\includegraphics{http://vn.spoj.com/content/centip1.gif}
\includegraphics{http://vn.spoj.com/content/centip2.gif}

Hình trên  nêu ví dụ vị trí của con rết độ dài 4 đốt khi ngủ và khi chuyển động một bước.

Hãy xác định số bước tối thiểu để rết ra khỏi hang hay cho biết rết không ra được.
\begin{verbatim}


	Dữ liệu: 

	Dòng đầu tiên chứa 3 số nguyên M  N  L ( 0 $<$ M, N ≤ 20, 2 ≤ L ≤ 8),

	Dòng thứ i trong  L dòng tiếp theo chứa 2 số nguyên  xi  yi - toạ độ đốt thứ i của rết,

	Dòng thứ L+2 chứa số nguyên K - số ô có đá,

	K dòng tiếp theo mỗi dòng chứa 2 số nguyên uj  vj - toạ độ ô có đá.



	Kết quả: Đưa ra số bước tối thiểu tìm được hoặc -1 nếu rết không thể ra được khỏi hang.



\textbf{Ví dụ:}

CENTIP.INP		

5  6  4		

4  1

4  2

3  2

3  1

3

2  3

3  3

3  4		



CENTIP.OUT

9



\end{verbatim}