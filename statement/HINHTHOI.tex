



   Hình thoi là hình tứ giác có bốn cạnh bằng nhau. Hình thoi   xuất hiện trong các quân bài Rô trên bộ bài Tây, trên các hoa   văn trang trí, v.v... Hình vuông cũng là một trường hợp đặc biệt của hình thoi.  

   Trong bài này, bạn hãy giải một bài toán về hình thoi: cho   N điểm trên mặt phẳng tọa độ, đếm số hình thoi có 4 đỉnh   thuộc vào tập hợp điểm này.  

\subsubsection{   Dữ liệu  }
\begin{itemize}
	\item     Dòng 1: một số N là số điểm (4 ≤ N ≤ 1500).   
	\item     Dòng thứ i trong N dòng tiếp chứa hai số nguyên   $x_{i}$    , $y_{i}$    là tọa độ của điểm thứ i (-50   ≤ $x_{i}$    , $y_{i}$    ≤ 50).   
\end{itemize}

\subsubsection{   Kết quả  }

   Ghi ra một số nguyên duy nhất là số hình thoi có 4 đỉnh   thuộc tập hợp điểm đã cho.  

\subsubsection{   Ví dụ  }
\begin{verbatim}
Dữ liệu
8
-1 0
0 1
1 0
0 0
1 1
0 -1
1 -1
2 0

Kết quả
4
\end{verbatim}
