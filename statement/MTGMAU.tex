
\begin{verbatim}
E. TAM GIÁC MÀUCho   điểm trên mặt phẳng, không có ba điểm nào thẳng hàng, các điểm được đánh số từ1 đến   . Người ta nối tất cả  các cặp điểm  (     )  bằng sợi dây màu xanh hoặc màu vàng theo nguyên tắc: Nếu         là số  nguyên tố  thì điểm      nối với điểm     bằng sợi dây màu xanh, ngược lại nếu       không phải số nguyên tố thì nối bằng sợi dây màu vàng. Sau đó người ta muốn khảo sát xem có bao nhiêu hình tam giác mà ba đỉnh là 3 điểm trong   điểm được nối với nhau bằng các sợi dây cùng màu.Yêu cầu:  Cho    nguyên dương, hãy đếm số  hình tam giác mà ba đỉnh được nối với nhau bằng các sợi dây cùng màu.Input:  Dòng đầu tiên ghi số nguyên dương T (T ≤ 10) là số lượng bộ dữ liệu.   Tiếp đến là T dòng, mỗi dòng tương ứng với một bộ dữ liệu chứa một số nguyên  (n ≤ 106).Output:  Gồm T dòng, mỗi dòng chứa một số  nguyên là số  tam giác đếm được  tương  ứng với mỗi bộ dữ liệu vào. ExampleInput235Output01
\begin{verbatim}
E. TAM GIÁC MÀU\end{verbatim}
\begin{verbatim}
Cho   điểm trên mặt phẳng, không có ba điểm nào thẳng hàng, các điểm được đánh số từ\end{verbatim}
\begin{verbatim}
1 đến N  . Người ta nối tất cả  các cặp điểm  (  i,j   )  bằng sợi dây màu xanh hoặc màu vàng \end{verbatim}
\begin{verbatim}
theo nguyên tắc: Nếu  i+j        là số  nguyên tố  thì điểm     i nối với điểm     j bằng sợi dây màu \end{verbatim}
\begin{verbatim}
xanh, ngược lại nếu i+j       không phải số nguyên tố thì nối bằng sợi dây màu vàng. Sau đó \end{verbatim}
\begin{verbatim}
người ta muốn khảo sát xem có bao nhiêu hình tam giác mà ba đỉnh là 3 điểm trong   N\end{verbatim}
\begin{verbatim}
điểm được nối với nhau bằng các sợi dây cùng màu.\end{verbatim}
\begin{verbatim}
Yêu cầu:  Cho   N nguyên dương, hãy đếm số  hình tam giác mà ba đỉnh được nối với nhau \end{verbatim}
\begin{verbatim}
bằng các sợi dây cùng màu.\end{verbatim}
\begin{verbatim}
Input:\end{verbatim}
\begin{verbatim}
  Dòng đầu tiên ghi số nguyên dương T (T ≤ 10) là số lượng bộ dữ liệu. \end{verbatim}
\begin{verbatim}
  Tiếp đến là T dòng, mỗi dòng tương ứng với một bộ dữ liệu chứa một số nguyên  \end{verbatim}
\begin{verbatim}
(N ≤ 10^6).\end{verbatim}
\begin{verbatim}
Output:\end{verbatim}
\begin{verbatim}
  Gồm T dòng, mỗi dòng chứa một số  nguyên là số  tam giác đếm được  tương  ứng \end{verbatim}
\begin{verbatim}
với mỗi bộ dữ liệu vào. \end{verbatim}
\begin{verbatim}
Example\end{verbatim}
\begin{verbatim}
Input\end{verbatim}
\begin{verbatim}
2\end{verbatim}
\begin{verbatim}
3\end{verbatim}
\begin{verbatim}
5\end{verbatim}
\begin{verbatim}
Output\end{verbatim}
\begin{verbatim}
0\end{verbatim}
\begin{verbatim}
1\end{verbatim}\end{verbatim}