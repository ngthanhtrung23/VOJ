

 

Một số được gọi là số tree\_num khi nó được là tổng của các lũy thừa cơ số 3 với số mũ không âm tăng dần.

Ví dụ 30=3\textasciicircum1+3\textasciicircum3; 325=3\textasciicircum0+3\textasciicircum4+3\textasciicircum5 là những số tree\_num.

Yêu cầu :Tìm số tree\_num thứ n

 

\subsubsection{Input}

Dòng đầu chứa số nguyên dương ntest là số test(ntest$<$=30000)

Ntest dòng sau:mỗi dòng chứa số nguyên dương n (0 $<$ n $<$ 2\textasciicircum64)

\subsubsection{Output}

 

Ntest dòng mỗi dòng chứa số tree\_num thứ n

 

\subsubsection{Sample}
\begin{verbatim}
Input:
5
1
2
7
3
6
Output:
1
3
13
4
12
\end{verbatim}