



   Có N viên bi màu được sắp thành một hàng trên mặt đất, các viên bi thuộc 1 trong K màu được đánh số từ 1 đến K.  

   Để tiện phân loại, beo\_chay\_so muốn sắp xếp lại các viên bi này sao cho các viên bi cùng màu thì nằm cạnh nhau, như vậy beo\_chay\_so sẽ thu được các đoạn liên tiếp gồm những viên bi cùng màu, mỗi màu chỉ thuộc đúng 1 đoạn.  

   Mỗi lần beo\_chay\_so chỉ được đổi chỗ 2 viên bi cạnh nhau, hãy giúp beo\_chay\_so sắp xếp lại các viên bi này sao cho số lần phải đổi chỗ các viên bi là ít nhất.  

\subsubsection{   Input  }

   Dòng thứ nhất ghi 2 số N và K là số viên bi và số màu. ( 2 ≤ N ≤ 20000, 1 ≤ K ≤ 10 )  

   Dòng thứ hai ghi N số nguyên dương là màu của N viên bi theo thứ tự.  

\subsubsection{   Output  }

   Ghi ra duy nhất một số nguyên là số phép đổi chỗ ít nhất.  

\subsubsection{   Example  }
\begin{verbatim}
Input:
5 3
3 2 1 3 2

Output:
3

Giải thích:
Đổi chỗ số thứ 3 và 4:
3 2 3 1 2
Đổi chỗ số thứ 4 và 5:
3 2 3 2 1
Đổi chỗ số thứ 2 và 3:
3 3 2 2 1
\end{verbatim}