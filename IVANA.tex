



   Zvonko và Ivana cùng chơi trò chơi sau:  
\begin{itemize}
	\item     Ban đầu, Zvonko đặt N số nguyên dương lên một vòng tròn.   
	\item     Mỗi người chơi chọn một số bên cạnh một số đã được chọn trước đó.   
	\item     Ivana đi trước, trong lượt đầu này Ivana được quyền chọn bất kỳ số nào.   
	\item     Trò chơi kết thúc khi tất cả các số đã được chọn. Ai chọn được nhiều số lẻ hơn sẽ thắng!   
\end{itemize}

   Bạn hãy giúp Ivana tìm những nước đi đầu tiên để sau đó cô có cơ hội thắng (biết rằng Zvonko luôn sử dụng chiến thuật tối ưu).  

\subsubsection{   Input  }
\begin{itemize}
	\item     Dòng 1: n, số số trên vòng tròn. (1$<$=n$<$=100)   
	\item     Dòng 2: n số nguyên dương cách nhau bởi khoảng trắng. Các số trong phạm vi từ 1 đến 1000 và không có hai số nào bằng nhau.   
\end{itemize}

\subsubsection{   Output  }

   Gồm một số nguyên duy nhất, là số nước đi đầu tiên để Ivana có cơ hội thắng  

\subsubsection{   Example  }
\begin{verbatim}
Input:
3
3 1 5

Output:
3

Input:
4
1 2 3 4

Output:
2

Input:
8
4 10 5 2 9 8 1 7

Output:
5
\end{verbatim}