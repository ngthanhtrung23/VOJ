

Cho đồ thị G có hướng N đỉnh M cạnh. Một đỉnh X được gọi là "truy cập" được Y nếu có đường đi từ X đến Y. Đỉnh X được gọi là phổ biến nếu thỏa mãn 1 trong 2 điều kiện : 1. X truy cập được Y. 2. Y truy cập được X. Yêu cầu : Cho đồ thị G, đếm số lượng đỉnh phổ biến. Input : - Dòng đầu gồm 2 số N, M (1 $<$= N $<$= 150000; 1 $<$= M $<$= 300000) - M dòng tiếp theo, mỗi dòng chứa 2 số x và y, thể hiện có cạnh nối từ x đến y. Output : - Dòng đầu in số lượng đỉnh phổ biến. - Dòng tiếp theo in chỉ số của các đỉnh phổ biến theo thứ tự tăng dần. Example : Input : 5 4 1 2 3 2 2 4 4 5 Output : 3 2 4 5

Cho đồ thị G có hướng N đỉnh M cạnh.

Một đỉnh X được gọi là "truy cập" được Y nếu có đường đi từ X đến Y.

Đỉnh X được gọi là phổ biến nếu tất cả các đỉnh Y trong đồ thị thỏa mãn 1 trong 2 điều kiện :
\begin{enumerate}
	\item X truy cập được Y.
	\item Y truy cập được X.
\end{enumerate}

Yêu cầu : Cho đồ thị G, đếm số lượng đỉnh phổ biến.

\textbf{Input}
\begin{itemize}
	\item Dòng đầu gồm 2 số N, M (1 $<$= N $<$= 150000; 1 $<$= M $<$= 300000)
	\item M dòng tiếp theo, mỗi dòng chứa 2 số x và y, thể hiện có cạnh nối từ x đến y.
\end{itemize}

\textbf{Output}
\begin{itemize}
	\item Dòng đầu in số lượng đỉnh phổ biến.
	\item Dòng tiếp theo in chỉ số của các đỉnh phổ biến theo thứ tự tăng dần.
\end{itemize}

\textbf{Example}
\begin{verbatim}
\textbf{Input}
5 4
1 2
3 2
2 4
4 5

\textbf{Output}
3
2 4 5

\end{verbatim}