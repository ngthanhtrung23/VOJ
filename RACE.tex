

 

Trong tất cả các cuộc đua trên thế giới thì cuộc đua Công thức 1 (F1) là cuộc đua được quan tâm đến nhiều nhất. Nó tập hợp được tất cả những tay đua tài giỏi nhất cũng như những kỹ thuật tiên tiến nhất để phục vụ cho một mục đích duy nhất là chiến thắng.

Đường đua xe là một đường vòng mà điểm xuất phát trùng với điểm kết thúc, các tay đua sẽ phải hoàn thành một số vòng đua nhất định tùy thuộc vào từng đường đua. Đường đua xe có n khúc quanh mà ta coi như khúc quanh thứ n+1 là khúc quanh thứ 1. Giữa 2 khúc quanh bất kỳ là một đoạn đường thẳng mà ta sẽ đánh số đoạn đường thẳng thứ i là đoạn đường thẳng ở sau khúc quanh thứ i. Đoạn đường thẳng thứ i có độ dài là Si.

Để cho xe đua khỏi bị văng khỏi đường đua thì các nhà kỹ thuật đã tính toán được rằng tại khúc quanh thứ i thì xe đua không được đi quá vận tốc mi , đồng thời nếu tại khúc quanh thứ i xe đua đi với vận tốc vi thì trên đoạn đường thẳng thứ i (là đoạn đường thẳng ngay sau đó) nó cũng phải đi với vận tốc vi.

Để đảm bảo an toàn trong trường hợp xảy ra tai nạn, xe đua chỉ được phép nạp một lượng xăng nhất định và chỉ được nạp thêm khi về đích, do vậy lượng xăng để hoàn thành một vòng đua sẽ chỉ là một số F0 cố định. Biết rằng khi đi trên đoạn đường thẳng có độ dài Si với vận tốc vi thì xe đua sẽ tiêu tốn một lượng xăng là Si.vi.

Bạn hãy tính vận tốc hợp lí trên mỗi đoạn đường thẳng cho xe đua để xe đua hoàn thành một vòng đua sớm nhất.

\subsubsection{Input}

Dòng đầu tiên ghi 2 số nguyên n và F0 là số khúc quanh và lượng xăng dành cho 1 vòng đua. (N$<$=10 000; F0$<$=10\textasciicircum9)

n dòng tiếp theo ghi n số nguyên m1..mn mà mi là vận tốc lớn nhất của xe đua ở khúc quanh thứ i. (Mi$<$=10\textasciicircum9)

n dòng cuối ghi n số nguyên S1..Sn mà Si là độ dài của đoạn đường thẳng thứ i. (Si$<$=10\textasciicircum9)

\subsubsection{Output}

Gồm n dòng, dòng thứ i ghi số thực vi là vận tốc hợp lí của xe đua trên đoạn đường thẳng thứ i. Số thực được ghi với độ chính xác 5 chữ số sau dấu phẩy.

\subsubsection{Example}
\begin{verbatim}
Input:
2 7
2 
4 
2
1


Output:
2.00000
3.00000
\end{verbatim}