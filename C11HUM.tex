

 

Với một tập K số nguyên tố cho trước S = \{p1, p2, …,pk), xét tập tất cả các số sao cho các thừa số nguyên tố của nó là tập con của S. Ví dụ, tập này có thể chứa: p1, p1*p2, p1*p1, p1*p2*p3 (và nhiều số khác). Tập này được gọi là tập “Số khiêm tốn” của tập S. Chú ý: Số 1 rõ ràng không phải là một số khiêm tốn.

Nhiệm vụ của bạn là tìm số khiêm tốn thứ N với một tập S cho trước. Kết quả không vượt quá số nguyên 32 bit có dấu.

\subsubsection{Input}

Dòng 1: gồm 2 số nguyên: K và N, 1  $\le$  K  $\le$  100 và 1  $\le$  N  $\le$  100000.

Dòng 2: K số nguyên dương miêu tả tập S. ( tất cả đều nhỏ hơn 1000 )

\subsubsection{Output}

Một số duy nhất là số khiêm tốn thứ N.

\subsubsection{Example}
\begin{verbatim}
\textbf{Input:}
4 19
2 3 5 7
\textbf{Output:}
27\end{verbatim}
