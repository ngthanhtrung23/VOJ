- Trong tất cả các test, 1 $<$= N $<$= 10$^5$, 0 $<$= M $<$= 10$^5$, 0 $<$= P$_i$ $<$= 10$^9$ và 1 $<$= C $<$= 10$^9$

- Trong 30\% số test đầu tiên, P$_1$ = P$_2$ = ... = P$_N$ = 0.

- Trong 30\% số test tiếp theo, N $<$= 1000 và C = 1.

- Trong 40\% số test còn lại, không có ràng buộc gì thêm.
\begin{verbatim}
\textbf{Input:}


2 1 1 1 100 1 1 2
10
1 2 1
2 3 2
3 4 2
4 8 1
1 8 7
1 5 3
5 8 4
1 6 2
6 7 2
\\7 8 2 \end{verbatim}
\begin{verbatim}
\textbf{Output:}

6 7

nd{verbatim}
egin{verbatim}
extbf{Input:
}6
1 1 2 3 1 0
7
1 2 2
2 3 3
3 6 4
1 4 4
\\4 3 2
\\4 5 3
\\5 6 2\end{verbatim}
\begin{verbatim}
\textbf{Output:}

5

nd{verbatim}
egin{verbatim}
extbf{Input:
}9
1 1 1 1 1 1 1 1 1
10
1 2 3
2 5 3
1 6 2
6 7 2
7 5 2
\\5 3 1
\\3 4 2
\\4 9 3
\\5 8 2
\\8 9 4\end{verbatim}
\begin{verbatim}
\textbf{Output:}

 7

\end{verbatim}
\begin{verbatim}
\textbf{Input:}

2
\\5 5
\\0\end{verbatim}
egin{verbatim}
extbf{Output:}

impossible

\end{verbatim}
Trong test đầu tiên
\\- Con đường 1 - 2 - 3 - 4 - 8 có độ dài 6 và tổng số bánh là 7. Đây là con đường tối ưu.
\\- Con đường 1 - 6 - 7 - 8 tuy có độ dài 6 nhưng tổng số bánh là 6, nên không tôi ưu
\\- Con đường 1 - 8 tuy có rất nhiều bánh nhưng tổng dộ dài là 7 nên không tôi ưu.