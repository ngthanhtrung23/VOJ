Trong tất cả các test, \textbf{T} ≤ 5.

Subtask 1 (15\% số điểm)
\begin{itemize}
	\item 1 ≤ \textbf{D} = \textbf{N} ≤ 20
\end{itemize}

Subtask 2 (25\% số điểm)
\begin{itemize}
	\item 1 ≤ \textbf{D} = \textbf{N} ≤ 10$^5$
\end{itemize}

Subtask 3 (30\% số điểm)
\begin{itemize}
	\item 1 ≤ \textbf{D} ≤ 20
	\item 10$^5$ $<$ \textbf{N} ≤ 10$^9$
\end{itemize}

Subtask 4 (30\% số điểm)
\begin{itemize}
	\item 1 ≤ \textbf{D} ≤ 10$^5$
	\item 10$^5$ $<$ \textbf{N} ≤ 10$^9$
\end{itemize}
\begin{verbatim}
\textbf{Input:}

3

2 2

1 1

2 2

0 0

2 2

1 0\end{verbatim}
\begin{verbatim}
\textbf{Output:}

0

0

1\end{verbatim}
Bộ dữ liệu đầu tiên

Trong đó có \textbf{hai} viên bi \textbf{đỏ} thì khi Tuấn bốc hai viên bi này lên. Theo quy tắc vì chúng cùng màu nên Tuấn sẽ bỏ 2 viên bi đó đi và \textbf{cho lại vào} hộp viên bi \textbf{màu đen} (màu \textbf{0}).

Bộ dữ liệu thứ hai

Tương tự như bộ dữ liệu đầu tiên nhưng là \textbf{hai} viên bi màu \textbf{đen}. Tuấn cũng theo quy tắc và bỏ hai viên bi đen đó đi và sau đó \textbf{cho lại vào} hộp viên bi \textbf{màu đen} (màu \textbf{0}).

Bộ dữ liệu thứ ba

Tuấn bốc hai viên bi từ trong hộp ra thì trong đó có \textbf{một} viên bi \textbf{đỏ} và \textbf{một} viên bi \textbf{đen}. Tuấn bỏ viên bi đen đi và chỉ \textbf{giữ lại} viên bi \textbf{màu đỏ} (màu \textbf{1}) và cho viên đỏ lại vào hộp.