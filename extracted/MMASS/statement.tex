Hóa chất chỉ gồm các nguyên tố C, H, O có trọng lượng 12,1, 16 tương ứng.  

   Nó được biểu diễn dạng "nén", ví dụ COOHHH là CO2H3  hay CH (CO2H) (CO2H) (CO2H) là CH(CO2H)3. Nếu ở dạng nén thì số lần lặp $>$=2 và  $\le$ 9.  

   Tính khối lượng hóa chất.  



\