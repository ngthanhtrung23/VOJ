Với sự phát triển ngày càng nhanh của hàng không Việt Nam, sân bay Nội Bài đã trở nên quá nhỏ bé. Sân bay chỉ có một đường băng và các máy bay khi bay tới Hà Nội sẽ phải bay lòng vòng phía trên để chờ được hạ cánh.  

   Để đơn giản, ta hãy mô tả sân bay trên mặt phẳng tọa độ Đề Các, mỗi đơn vị độ dài sẽ tương đương 1 Km. Đường băng của sân bay là một đoạn thẳng từ (0, 0) đến (-7, 0). Các máy bay khi đến Hà Nội sẽ phải bay ở khu vực chờ, đó là một hình có dạng hình chữ nhật với bốn góc là các đoạn ¼ đường tròn. Góc trái dưới của hình chữ nhật có tọa độ ($X_{1}$   , 0), góc phải trên là ($X_{2}$   , $Y_{2}$   ). Các góc phần tư hình tròn có bán kính là R. Dưới đây là ví dụ với $X_{1}$   = 2, $X_{2}$   = 11, $Y_{2}$   = 7, R = 1.  
\includegraphics{http://vn.spoj.pl/VO09/content/Airctrl.jpg}

   Các máy bay sẽ bay với cùng vận tốc 10 Km/phút và theo hướng cùng chiều kim đồng hồ. Khi được phép hạ cánh, máy bay phải bay tới vị trí ($X_{1}$   + R, 0) (vị trí được đánh dấu hình tròn màu đỏ trên hình vẽ) rồi từ đó bay thẳng vào đường băng. Tại thời điểm ban đầu, có N máy bay, tại các tọa độ ($X_{U}$   , $Y_{U}$   ). Với mỗi máy bay, ta được biết lượng nhiên liệu còn lại đủ để đi quãng đường là $P_{U}$   (Km). Máy bay được coi là hạ cánh an toàn nếu nó đủ nhiêu liệu để bay đến điểm có tọa độ (0, 0) (đầu đường băng).  

   Bạn hãy sắp xếp thứ tự được hạ cánh của các máy bay sao cho thời gian hạ cánh gần nhất giữa 2 máy bay liên tiếp là lớn nhất có thể được, điều này sẽ tăng độ an toàn của các lần hạ cánh.