\emph{Mưa từng con phố có nhớ bóng dáng em đi cuối thu, đông về hè sang vội vàng quá }

\emph{Mưa từng đêm vắng, mưa ơi cứ rơi phố xa, anh đi tìm yêu đương chiều qua }

\emph{Ai vội vàng đi ngang lòng người mang theo bao yêu đương thoáng qua như là cơn mưa rào }

\emph{Chuyện mưa }

Có bao giờ bạn tự hỏi, những giọt mưa kia rồi sẽ về đâu? Chúng ta hãy cùng nhau trả lời câu hỏi này nhé.

Hình dung thế giới như một mặt phẳng Oxy, tại mỗi thời điểm có 1 trong 3 loại sự kiện xảy ra:
\begin{itemize}
	\item Một đoạn thẳng nối điểm (x $_ 1 $ , y $_ 1 $ ) và (x $_ 2 $ , y $_ 2 $ ) xuất hiện trên mặt phẳng
	\item Một đoạn thẳng được thêm vào trước đó biến mất
	\item Một giọt mưa xuất hiện tại điểm (x, y) và rơi xuống. Giọt mưa rơi theo phương thẳng đứng từ điểm (x, y) về điểm (x, 0).
\end{itemize}

Nhiệm vụ của bạn là xác định xem, giọt mưa rơi xuống mặt đất, hay rơi vào một đoạn thẳng.