Cho một ma trận A kích thước N*N chỉ gồm các giá trị \{-2, -1, 0, 1, 2\}. Các hàng của ma trận A được đánh số từ 1 đến N từ trên xuống dưới. Các cột của ma trận A được đánh số từ 1 đến N từ trái sang phải. Phần tử ở hàng i, cột j được ký hiệu là A(i,j).

Bảng A được gọi là tương thích với dãy T nếu:
\begin{itemize}
	\item Dãy T gồm đúng N phần tử. Các phần tử của dãy T được đánh số từ 1 đến N. Phần tử thứ i được ký hiệu là T(i).
	\item Với mọi i, j mà 1 ≤ i, j ≤ N:
\begin{itemize}
	\item A(i,j) = 0 ↔ T(i) = T(j).
	\item A(i,j) = 1 ↔ T(i) $<$ T(j).
	\item A(i,j) = 2 ↔ T(i) ≤ T(j).
	\item A(i,j) = -1 ↔ T(i) $>$ T(j)
	\item A(i,j) = -2 ↔ T(i) ≥ T(j).
\end{itemize}
\end{itemize}

Yêu cầu: cho trước ma trận A, hãy tìm dãy số nguyên dương T tương thích với A, sao cho giá trị lớn nhất của dãy T là nhỏ nhất có thể. Biết rằng luôn tồn tại một dãy T như vậy.