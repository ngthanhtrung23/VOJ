Cho tập gồm N (1 ≤ N ≤ 26) chữ cái đầu tiên trong bảng chữ cái Latin và số nguyên dương M (1 ≤ M ≤ N). Cặp số (N,M) xác định một tập hợp tất cả các từ gồm M chữ cái khác nhau từ N chữ cái đã cho. Các từ trong tập hợp này được sắp xếp thành dãy theo thứ tự từ điển. Khi đó, ta gọi số thứ tự từ điển của một từ là số thứ tự của nó trong dãy từ được sắp xếp.  

   Ví dụ, cặp (N=3, M=2) xác định tập \{ab, ac, ba, bc, ca, cb\}. Từ 'bc' tương ứng với 4, từ 'ab' tương ứng với 1, từ 'ca' tương ứng với 5,....  

   Yêu cầu: Giả sử biết cặp số nguyên (N,M), khi đó cho một từ bạn cần xác định số thứ tự từ điển của nó, ngược lại cho biết số thứ tự từ điển của một từ bạn cần đưa ra từ đó. Trong cả hai tình huống, bạn phải kiểm tra xem dữ liệu có đúng đắn hay không: trong tình huống thứ nhất, dữ liệu là đúng đắn nếu từ đã cho thuộc tập từ được xét, còn trong tình huống thứ hai, dữ liệu là đúng đắn nếu tìm được từ trong tập từ có số thứ tự đã cho.