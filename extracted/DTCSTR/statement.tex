Sắp có một biểu đình đả đảo những đề bài do       \textit{     pirate    }      viết ra. Lý do đơn giản là vì chúng quá dài và quá sến.   \textit{    pirate   }   rất buồn khi nghe được điều đó. Nếu bắt anh ta thay đổi thì chẳng khác nào giết chết tâm hồn thi ca trong một con người. Nhưng vì tình yêu với mọi người, pirate quyết định đây là đề bài dài và sến cuối cùng mà anh sẽ viết ra.  

   Một ngày nọ, đang nghiên cứu môn stringology, anh chàng nổi hứng chuyển sang nghiên cứu môn philosophy (để chuẩn bị cho những năm tháng sẽ bị nó hành hạ sau này). Sau một ngày hì hục bên bên chồng sách về "Tư tưởng Hồ Chí Minh" và "Chủ nghĩa xã hội khoa học", anh ngẫm ra chân lý của cuộc sống : Mọi sự vật hiện hữu ở hiện tại đều do một sự vật hiện hữu ở quá khứ tạo thành, giống như những mắc xích của sự tiến hóa. Ngay lập tức,   \textit{    pirate   }   áp dụng nó vào các chuỗi.  

   Vấn đề đặt ra là cho một chuỗi S, bạn hãy xác định độ dài của chuỗi A thỏa hai điều kiện sau:  
\begin{itemize}
	\item     Chuỗi S phải phân tích được ra thành nhiều mắc xích. Mỗi mắc xích do một dãy các ký tự liên tiếp của S tạo thành và là một chuỗi A. Mỗi ký tự của chuỗi S phải thuộc vào ít nhất một mắc xích. Ví dụ: S = ababa được tạo thành từ mắc xích là ab         a        và         a        ba (khi ghép hai chuỗi này và để phần in đậm trùng lên nhau thì được chuỗi S).   
	\item     Độ dài chuỗi A phải là nhỏ nhất.   
\end{itemize}

\