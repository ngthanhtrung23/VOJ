Năm 2011, Tí bắt đầu tìm hiểu về thuật toán và mua rất nhiều sách để đọc. Hiện tại, Tí đang tìm hiểu về một thuật toán và muốn tìm các chương nói về thuật toán đó trong sách. Để làm việc này, Tí chỉ cần tìm các lần xuất hiện của tên thuật toán trong sách (tất nhiên, Tí hoàn toàn có thể viết một chương trình tự động). Tuy nhiên, một vấn đề nảy sinh là Tí có thể nhớ không chính xác tên của thuật toán mà bị nhầm một ký tự. Ví dụ, Tí có thể nhớ thuật toán tìm đường đi ngắn nhất là Diikstra thay vì Dijkstra.
\\
\\Tiếc là Tí chưa đủ khả năng để viết một chương trình khắc phục vấn đề này, và Tí nhờ bạn: Hãy viết một chương trình tìm vị trí xuất hiện đầu tiên của một từ trong một văn bản, trong đó sai lệch ở một ký tự duy nhất có thể được bỏ qua. Do các cuốn sách của Tí đều sử dụng tiếng Việt có dấu, trong bài này chúng ta biểu diễn xâu bằng một dãy số, trong đó mỗi số thể hiện mã Unicode của một ký tự. (Mã Unicode gần giống như mã ASCII nhưng sử dụng 16 bit).