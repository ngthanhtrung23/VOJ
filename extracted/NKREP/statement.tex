Trung tâm của bạn vừa nhận được 1 bức ảnh chụp bề mặt sao Hỏa. Bức ảnh được mã hóa theo dạng hình chữ nhật được chia thành các ô vuông, trong đó mỗi ô có thể có màu đen hoặc trắng. Do nhiều lí do trên đường vận chuyển, có tối đa K ô bị mờ đi (đen thành trắng), nhưng do không thể xác định chính xác được ô nào bị mờ, nên bạn buộc phải giả sử rằng tất cả các khả năng mờ đi đều có thể xảy ra.

 

Để xác định các vật thể lạ trên sao Hỏa, trung tâm của bạn căn cứ theo các ô đen được chụp trên tấm ảnh. Hai ô đen A và B được gọi là cùng thuộc một vật thể, nếu từ A, tồn tại một đường đi thỏa mãn chỉ đi qua các ô đen kề cạnh, có thể đến được B. Kích thước vật thể được tính bằng khoảng cách Euclid của 2 ô xa nhau nhất cùng thuộc vật thể đó.

 

Yêu cầu: xác định kích thước lớn nhất có thể có được của 1 vật thể trên tấm ảnh.