Có N điểm trên mặt phẳng với tọa độ là các số tự nhiên. Một   \textit{    đa giác lồi nhiều đỉnh nhất   }   là một đa giác lồi có các đỉnh là       gốc tọa độ      và một số đỉnh trong các điểm đã cho, và có số đỉnh là nhiều nhất. Điểm gốc, nghĩa là điểm có tọa độ (0, 0),       phải      là một trong các đỉnh của   \textit{    đa giác lồi nhiều đỉnh nhất   }   .  

   Viết chương trình xác định số đỉnh của đa giác này.  

   Một đa giác là   \textit{    lồi   }   nếu mọi đoạn thẳng có đầu mút nằm trong đa giác đều nằm hoàn toàn trong đa giác đó.  

   Các cạnh liên tiếp của một đa giác không được phép song song với nhau.  

\