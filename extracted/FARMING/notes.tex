- 1 ≤ M ≤ 50

- 1 ≤ N ≤ 50

- 1 ≤ D, T$_i $≤ 100

- 1 ≤ G, R$_i$, E$_i$ ≤ 1,000

- 1 ≤ F, S$_i$, P$_i $≤ 100,000
\textbf{Dữ liệu}
\begin{verbatim}
3 3 5 10000 5

5 3 3000 5000 2

10 2 7000 10000 3

10 1 6000 8000 2

\end{verbatim}

\textbf{Kết quả 1}
\begin{verbatim}
22000

2

1 1

4 2

2

1 1

4 2

1

1 1

\end{verbatim}

\textbf{Kết quả 2}
\begin{verbatim}
24000

3

1 1

4 3

5 3

3

1 1

4 3

5 3

1

1 1

\end{verbatim}

\textbf{Kết quả 3}
\begin{verbatim}
000

3

1 1

4 3

5 3

2

1 1

4 2

1

1 1\end{verbatim}

\textbf{Giải thích kết quả 3}
\begin{tabular}\hline 
\textbf{Ngày canh tác thứ} & \textbf{Ruộng 1} & \textbf{Ruộng 2} & \textbf{Ruộng 3} & \textbf{Tổng tiền (Đồng)} & \textbf{Tổng kinh nghiệm} \\ 
\hline
Ban đầu & - & - & - & 10,000 & 5 \\ 
\hline
1 & Trồng cây 1 & Trồng cây 1 & Trồng cây 1 & 1,000 & 5 \\ 
\hline
2 & Chăm sóc & Chăm sóc & Chăm sóc & 1,000 & 5 \\ 
\hline
3 & Thu hoạch & Thu hoạch & Thu hoạch & 16,000 & 11 \\ 
\hline
4 & Trồng cây3
\\ Thu hoạch & Trồng cây 2 & Để trống & 11,000 & 13 \\ 
\hline
5 & Trồng cây3 
\\ Thu hoạch & Thu hoạch & Để trống & 23,000 & 18 \\ 
\hline

\end{tabular}
- Với mỗi test, bạn chỉ được tính điểm nếu lịch gieo trồng hợp lí, đúng với số tiền kiếm được.

- Điểm mỗi test đúng đắn được tính dựa trên kết quả tốt nhất của tất cả các thí sinh (và của cả BTC)

  Cụ thể, nếu kết quả tốt nhất là best, và chương trình của bạn đưa ra result thì:

  Điểm = result/best * hệ số điểm. (Trong đó hệ số điểm = 100/tổng số test)

- Tổng điểm của bạn là tổng điểm các test đúng đắn.

- Với test ví dụ, best = 24000. Nếu hệ số điểm = 5, kết quả 1 sẽ được 4.58 điểm, kết quả 2 sẽ được 5.00 điểm, và kết quả 3 sẽ được 4.79 điểm.