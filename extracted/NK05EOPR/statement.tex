Cho một dãy số là một hoán vị của 12 số tự nhiên đầu tiên (từ 0 đến 11). Giả sử số 0 ở vị trí thứ i trong dãy số (vị trí được đánh số từ 0 đến 11, từ trái sang phải) thì bạn có thể đổi chỗ số 0 với số ở vị trí thứ j nếu thỏa mãn cả hai điều kiện sau:  
\begin{itemize}
	\item     | i – j | = d    $_     k    $    , với k=1..3 và (d    $_     1    $    ,d    $_     2    $    ,d    $_     3    $    ,d    $_     4    $    )=(1;3;6;12)    


	\item     [i/d    $_     k+1    $    ]=[j/d    $_     k+1    $    ], với [] là hàm phần nguyên   
\end{itemize}

   Bạn hãy tìm số phép đổi chỗ ít nhất để có thể sắp xếp dãy số theo thứ tự tăng dần  

\