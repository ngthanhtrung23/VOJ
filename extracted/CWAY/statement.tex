Một đồ thị đầy đủ N đỉnh là đồ thị mà giữa mọi cặp đỉnh đều có cạnh nối. Bạn hãy đếm số đường đi giữa 2 đỉnh bất kì của đồ thị. Lưu ý rằng một đường đi không được đi qua một đỉnh quá một lần.

\