Cho dãy số A gồm N phần tử là hoán vị của N số nguyên từ 0 đến N - 1 và được đánh số lần lượt từ 1 đến N. Phép biến đổi Swap(x) sẽ đổi chỗ A[x] và A[x + 1] (1 ≤ x $<$ N). Một hoán vị B gọi là đẹp nếu thỏa mãn 2 điều kiện sau:  
\begin{enumerate}
	\item     Là hoán vị của N – 1 số gồm các số từ 1 đến N – 1.   
	\item     Sau khi thực hiện lần lượt các phép biến đổi Swap(B[1]), Swap(B[2]), ..., Swap(B[N - 1]) trên dãy số A đã cho thì được dãy số mới là dãy tăng dần.   
\end{enumerate}

   Yêu cầu: Hãy đếm số hoán vị đẹp.