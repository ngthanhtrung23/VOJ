Dạo này tivi cũng đang chiếu phim Lục Vân Tiên , sẵn tiện lấy luôn làm tiêu đề .   


   Lục Vân Tiên cũng giống Samurai Jack , bị Quan Thái Sư đẩy vào vòng xoáy thời gian và bị chuyển tới tương lai của những năm 2777 .   


   Ở thời đại này , Tráng sỹ phải là người thông thạo máy tính , gõ bàn phím lia lịa như đấu sỹ thời xưa múa kiếm ấy và phải qua một cuộc thi lập trình mới được phong danh hiệu .   


   Để vượt qua vòng loại , Vân Tiên cần tham gia cuộc thi sát hạch . Ban Giám Khảo cuộc thi sát hạch gồm có N người , họ đều là các cao thủ trong giới IT . Các thành viên trong Ban Giám Khảo được đánh số từ 1 -> N và mỗi người lại có một chỉ số sức mạnh gọi là APM ( Actions Per Minute ) . Các giám khảo sẽ xếp hàng lần lượt từ 1 -> N . Mỗi thí sinh sẽ phải đấu với K vị giám khảo và K vị giám khảo này phải đứng liền thành 1 đoạn ( Tức là i , i+1 , i+2 , ... i+K-1 ) , chỉ cần thắng 1 vị giám khảo thì sẽ vượt qua vòng loại .   


   Tuy nhiên thí sinh kô được chọn xem những giám khảo nào sẽ đấu với mình .   


   Vân Tiên rất lo vì lỡ may đụng độ với những vị giám khảo nào "khó nhằn" thì sẽ tiêu mất . Nên chiến thuật của Vân Tiên là tập trung hạ vị giám khảo có chỉ số APM thấp nhất trong số K vị . Bạn hãy lập trình để giúp Lục Vân Tiên xác định được ở tất cả các phương án thì chỉ số APM của vị giám khảo thấp nhất sẽ là bao nhiêu ( Có tất cả N-k+1 phương án :   


   Phương án 1 : Vân Tiên phải đấu với vị 1 -> vị k   


   Phương án 2 : Vân Tiên phải đấu với vị 2 -> vị k+1   


   …   


   Phương án N-k+1  : Vân Tiên phải đấu với vị N-k+1 -> vị N ) .   





   ( 1  $\le$  N  $\le$  17000 , chỉ số APM của 1 giám khảo $>$= 1 và  $\le$  2 tỉ , 1  $\le$  K  $\le$  N ) .