Công ty Cốc Cốc (nhà tài trợ của cuộc thi) đang tập trung nghiên cứu trong lĩnh vực xử lý ngôn ngữ Tiếng Việt.  

   Pirate cũng đang rất quan tâm vấn đề này vì gần đây anh thường bị chóng mặt và nôn mửa khi phải xử lý một khối lượng bài tập TC (không phải TopCoder mà là TeenCode) quá lớn. Vì TeenCode quá rối rắm nên Pirate dễ thương và tội nghiệp quyết định tìm một vài phút giây thanh thản bên những biểu thức số học.  

   Nhắc đến biểu thức số học, chắc các bạn đã khá quen thuộc với khái niệm   \textbf{    Postfix Notation   }   (PN) (cũng được biết đến với cái tên   \textbf{    Reverse Polish Notation   }   - ký pháp nghịch đảo Ba Lan).  

   Trong ký pháp thông thường, toán tử được viết ở giữa hai toán hạng, ví dụ 3 + 4. Tuy nhiên, trong PN, toán tử được viết sau hai toán hạng, ví dụ 3 4 +. Các bạn sẽ tự hỏi rằng tại sao cách viết như vậy hiệu quả hơn? Một trong số các câu trả lời là vì biểu thức dạng PN có thể dễ dàng được tính toán bằng cấu trúc dữ liệu   \href{http://www.cosc.canterbury.ac.nz/mukundan/dsal/StackAppl.html}{    ngăn xếp (stack)   }   với độ phức tạp tuyến tính.  

   Thuật toán tính giá trị của biểu thức dạng PN sử dụng stack như sau:  

Xử lý biểu thức từ trái sang phải:
\begin{itemize}
	\item Nếu gặp một toán hạng, đẩy (push) nó vào stack.
	\item Nếu gặp một toán tử, lấy (pop) hai toán hạng từ stack ra và tính kết quả của phép tính gồm hai toán hạng trên và toán tử đang xét. Sau đó, đẩy (push) kết quả trở lại vào stack.
\end{itemize}

   Hãy xem xét một ví dụ phức tạp hơn: "4 - 3 * 2". PN của biểu thức trên là "4 3 2 * -". Ta sẽ thử áp dụng thuật toán vào ví dụ:  

Xử lý biểu thức từ trái sang phải:

- Đẩy 4 vào stack.

- Đẩy 3 vào stack.

- Đẩy 2 vào stack.

- Gặp dấu *, ta lấy 3 và 2 ra khỏi stack, tính 3 * 2 = 6. Đẩy 6 vào stack.

- Gặp dấu -, lấy 4 và 6 ra, tính 4 - 6 = -2 (lưu ý rằng số nào được đưa vào stack trước sẽ đứng trước trong phép tính). Đẩy -2 vào stack.

- Kết quả của biểu thức là -2.

   Pirate loay hoay tìm cách mô phỏng thuật toán trên. Anh tìm thấy một cái hộp đựng cầu lông. Tốt quá rồi, cái này có thể mô phỏng được stack. Nhưng khổ một nỗi là hộp đựng cầu này không có cái nắp nào cả nên hai đầu trống trơn. Kế hoạch thất bại, nếu phải quay lại với TeenCode thà chết còn hơn, Pirate nghĩ thầm. Trong lúc đang dùng hộp cầu đập vào đầu để tự vẫn, Thần Cầu hiện ra cười và hỏi:  

\emph{    - Tạj seo kon hú? (Thần Cầu cũng thích TeenCode)    
\\}

- Hú hú hú, con không thể dùng cái hộp này để làm một cái stack vì nó không có đáy.

- Đừng bỏ cuộc con ạ, người nhìn thấy thành công là người nhìn thấy khó khăn và biến nó thành cơ hội. Con hãy nhìn lại xem cái hộp giờ trông giống cái gì?

- Nó giống... một hàng đợi ạ.

   Thần Cầu đã cho Pirate một gợi ý tuyệt vời. Thay vì làm thí nghiệm với stack, Pirate sẽ làm thí nghiệm với   \href{http://www.cosc.canterbury.ac.nz/mukundan/dsal/QueueAppl.html}{    queue (hàng đợi)   }   . Nhưng như thế thì Pirate cần một ký pháp mới để biểu diễn biểu thức. Anh gọi ký pháp mới là   \textbf{    Queue Postfix Notation   }   (QPN). Thuật toán tính giá trị của QPN cũng gần giống với PN thông thường:  

Xử lý biểu thức từ trái sang phải:
\begin{itemize}
	\item Nếu gặp một toán hạng, đẩy (enqueue) nó vào queue.
	\item Nếu gặp một toán tử, lấy (dequeue) hai toán hạng từ queue ra và tính kết  quả của phép tính gồm hai toán hạng trên và toán tử đang xét. Sau đó,  đẩy (enqueue) kết quả trở lại vào queue.
\end{itemize}

   Biểu thức "4 - 3 * 2" có thể được viết lại dưới dạng QPN là "3 2 4 * -" và được tính toán như sau:  

    Xử lý biểu thức từ trái sang phải:   

- Đẩy 3 vào queue.

- Đẩy 2 vào queue.

- Đẩy 4 vào queue.

- Gặp *, lấy 3 và 2 ra khỏi queue, tính 3 * 2 = 6. Đẩy 6 vào queue.

\emph{    - Gặp -, lấy 4 và 6 ra khỏi queue, tính 4 - 6 = -2 (lưu ý rằng số nào vào queue trước sẽ đứng trước trong phép tính). Đẩy -2 vào queue.    
\\}

- Kết quả của biểu thức là -2.