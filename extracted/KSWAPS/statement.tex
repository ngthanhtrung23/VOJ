K đem lòng yêu con gái của nhà phú ông. Ngày nọ, K mang trầu cau sang nhà phú ông xin hỏi cưới. Chẳng biết nghe thiên hạ đồn đại thế nào mà phú ông lại lầm tưởng K là một tên sở khanh và ra một câu đố hóc búa để K tay không trở về. Số là phú ông có N người con gái y hệt như nhau (nhưng tính tình thí khác nhau, vì thế K chỉ thích mỗi một cô thôi). Phú ông cho các cô gái đứng xếp thành một hàng ngang, đánh số từ 1 đến N, rồi bảo sẽ cho đổi chỗ vị trí của các cô M lần. Mỗi lần, hai cô có vị trí cạnh nhau sẽ đổi chỗ cho nhau. "Tưởng gì, định hỏi xem bạn gái ta ở đâu chứ gì", K nghĩ thầm và khá tự tin vì nàng đã ra dấu cho K biết rồi. Nhưng sai rồi, phú ông thách K tính được rằng có bao nhiêu cách sắp xếp khác nhau của N cô gái thu được sau M lần đổi như trên nếu các cô đứng theo thứ tự ban đầu. Mỗi cách sắp xếp là một dãy p gồm N số, trong đó p\_i thể hiện vị trí ban đầu của cô gái thứ đứng ở vị trí i sau khi hoàn tất quá trình đổi chỗ. Không giải được thì K đừng hòng lấy được vợ.