 

Cho dãy số nguyên a $_ 1 $ , a $_ 2 $ , …, a $_ n $ và số nguyên dương k. Ta gọi k-phân đoạn của dãy số đã cho là cách chia dãy số đã cho ra thành k đoạn, mỗi đoạn là một dãy con gồm các phần tử liên tiếp của dãy. Chính xác hơn, một k-phân đoạn được xác định bởi dãy chỉ số

1  $\le$  n $_ 1 $ $<$ n $_ 2 $ $<$ n $_ 3 $ $<$ ... $<$ n $_ k $ = n

Đoạn thứ i là dãy con a $_ n $_ i-1 $ +1 $ , a $_ n $_ i-1 $ +2 $ , ..., a $_ n $_ i $$ , i=1, 2, ..k. Ở đây quy ước n $_ 0 $ =0

Yêu cầu: Hãy xác định số M nhỏ nhất để tồn tại k-phân đoạn sao cho tổng các phần tử trong mỗi đoạn đều không vượt quá M.

\