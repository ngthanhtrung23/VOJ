Consider the following algorithm:
\begin{verbatim}
\texttt{
		1. input n
}


\texttt{		2. print n
}


\texttt{		3. 	if n = 1 then STOP
}


\texttt{		4. 	     if n is odd then  n $<$-- 3n+1 
}


\texttt{		5. 		else  n $<$-- n div 2 
}


\texttt{		6. GOTO 2
}

 \end{verbatim}

Given the input 22, the following sequence of numbers will be printed 22 11  34 17 52 26 13 40 20 10 5 16 8 4 2 1

It is conjectured that the algorithm above will terminate (when a 1 is printed) for any integral input value. Despite the simplicity of the algorithm, it is unknown whether this conjecture is true. It has been verified, however, for all integers n such that 0 $<$ n $<$ 1,000,000 (and, in  fact, for many more numbers than this.)

Given an input n, it is possible to determine the number of numbers  printed (including the 1). For a given n this is called the  cycle-length of n. In the example above, the cycle length of 22  is 16.

For any two numbers i and j you are to determine the maximum  cycle length over all numbers between i and j.