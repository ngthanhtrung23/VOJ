Sau khi thương lượng, cuối cùng Phú Ông quyết định sẽ cắt cho Bờm 2 khu đất trong quỹ đất của mình để đổi lấy chiếc quạt mo. Quỹ đất của phú ông có dạng R hàng ngang và C cột dọt gồm \textbf{R }* \textbf{C} thửa. Mỗi thửa có một giá trị kinh tế khác nhau (có thể âm).

Phú Ông đồng ý chia cho Bờm 2 khu đất, một khu \textbf{hình chữ nhật} kích thước \textbf{A }* \textbf{B} gồm \textbf{A} hàng ngang và \textbf{B }cột thửa đất nhỏ (lưu ý \textbf{A} * \textbf{B} khác \textbf{B} * \textbf{A}); và 1 khu đất \textbf{“hình thoi”} kích thước \textbf{K} sao cho:
\begin{itemize}
	\item Hai khu đất Bờm chọn không chứa thửa đất nào chung (\textbf{không giao nhau}).
	\item Hai khu đất Bờm chọn phải nằm \textbf{hoàn toàn trong} khu đất của Phú Ông.
\end{itemize}

Biết rằng khu đất hình thoi với tâm ở thửa (\textbf{x}, \textbf{y}) và có kích thước \textbf{K} sẽ là tập hợp các thửa (\textbf{x'}, \textbf{y'}) thoả mãn |\textbf{x’} - \textbf{x}| + |\textbf{y’} - \textbf{y}| ≤ \textbf{K}. Phú ông cho Bờm lựa chọn tuỳ ý, Bờm cũng bắt đầu giao động trước đề nghị này. Bạn hãy giúp Bờm xác định 2 khu đất thoả mãn sẽ có giá trị kinh tế \textbf{lớn nhất} là bao nhiêu.