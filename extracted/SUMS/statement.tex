Cho tập số nguyên A gồm n phần tử, A=\{a1, a2,..., an\}. Số k được gọi là phụ thuộc vào tập A, nếu k được tạo thành bằng cách cộng các phần tử của tập A(mỗi phần tử có thể cộng nhiều lần).  

   Ví dụ  cho A=\{2,5,7\}.  Các số như 2, 4(2+2), 12(5+7 hoặc 2+2+2+2+2) được gọi là phụ thuộc vào tập A. Số 0 cũng gọi là phụ thuộc vào tập A.
Cho một dãy B, hãy kiểm tra xem bi có phải là số phụ thuộc vào tập A hay không .