Có n xe buýt chạy trên tuyến là một đường vòng khép kín có độ dài S. Các xe buýt được đánh số từ 1 đến n theo thứ tự nối đuôi nhau trên tuyến đường. Xe số n chạy sau xe số 1.  

   Các xe chạy cùng với vận tốc $V_{0}$   và khoảng cách giữa hai xe liên tiếp là như nhau.  

   Có k xe buýt đồng thời rời khỏi tuyến. Để trở lại khoảng cách đều nhau giữa các xe, cần phải có một khoảng thời gian t và một số xe cần phải thay   đổi tốc độ. Trong khoảng thời gian này, các xe phải chạy với tốc độ không đổi trong khoảng [$V_{min}$   , $V_{max}$   ] theo lệnh của   trung tâm.  

   Hết khoảng thời gian t các xe lại quay về vận tốc $V_{0}$   .  

   Yêu cầu: giúp trung tâm tìm khoảng thời gian bé nhất $T_{min}$   để khôi phục sự cân bằng khoảng cách giữa hai xe liên tiếp trên tuyến và   vận tốc của mỗi xe trong khoảng thời gian ấy. Biết rằng, trong quá trình điều chỉnh, không có xe nào vượt qua xe trước mặt.