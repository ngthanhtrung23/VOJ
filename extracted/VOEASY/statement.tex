Cho bộ N số nguyên không âm ($a_{1}$   , $a_{2}$   , …, $a_{N}$   ) và một danh sách gồm M bộ quy tắc biến đổi.       Ta định nghĩa một quy tắc biến đổi (u, oper, v) là phép biến đổi dãy số A gồm N phần tử (a1, a2, ... , aN) thành một dãy số B gồm N phần tử (b1, b2, ..., bN)       Với dãy b thỏa mãn       bi = au oper av. Oper chỉ gồm \{and, or, xor\}    

   Ta định nghĩa một bộ quy tắc biến đổi gồm N quy tắc biến đổi (u, oper, v). Những quy tắc này sẽ biến đổi dãy số A gồm N phần tử ($a_{1}$   , $a_{2}$   , …, $a_{N}$   ) thành dãy số B cũng gồm N phần tử ($b_{1}$   , $b_{2}$   , ... , $b_{N}$   ). Trong đó, quy tắc thứ i như sau:  

   $b_{i}$   = $a_{u}$   oper $a_{v}$   , với oper là 1 trong 3 phép toán and, or, xor.  


\\   Yêu cầu: Cho bộ ($a_{1}$   , $a_{2}$   , ... , $a_{N}$   ) ban đầu, M bộ quy tắc biến đổi và số K. Ta lần lượt biến đổi dãy A ban đầu thông qua các bộ quy tắc theo thứ tự sau: 1, 2, …, M, 1, 2, …,M, 1, 2, ... Hãy xác định giá trị cuối cùng của ($a_{1}$   , $a_{2}$   , …, $a_{N}$   ) sau K lần biến đổi.