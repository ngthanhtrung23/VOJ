\begin{itemize}
	\item Trong tất cả các test 0 ≤ p ≤ 100
	\item Subtask1: N ≤ 15 (25\% số điểm)
	\item Subtask2: N ≤ 2207 (31.25\% số điểm)
	\item Subtask3: N ≤ 97722, p ≤ 97 (43.75\% số điểm)
\end{itemize}

Sau khi kết thúc kỳ thi, kết quả của bạn là kết quả lần nộp bài cuối cùng
\begin{verbatim}
\textbf{Input:}

4

1 2 50

2 3 50

1 4 50\end{verbatim}
\begin{verbatim}
\textbf{Output:}

1.00\end{verbatim}

Giải thích:

 

Nếu 2 con đường (1,2) và (2,3) đều không bị ngập (xác suất 25\%), độ dài đường đi là 2.

Nếu 2 con đường (1,2) và (1,4) đều bị ngập (xác suất 25\%), độ dài đường đi là 0.

Các trường hợp còn lại (xác suất 50\%), độ dài đường đi là 1.

Vậy độ dài trung bình là (0.25*2+0.25*0+0.5*1)=1.
\begin{verbatim}
\textbf{Input:}

3

1 2 33

1 3 66\end{verbatim}
\begin{verbatim}
\textbf{Output:}

0.7722\end{verbatim}

Giải thích:

 

Nếu 1 trong 2 con đường k bị ngập, độ dài đường đi là 1, ngược lại độ dài đường đi là 0.

Xác suất để cả 2 con đường đều bị ngập là (100\%-33\%)*(100\%-66\%)=22.78\%