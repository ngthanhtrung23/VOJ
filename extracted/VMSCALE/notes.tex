Giới hạn
\begin{itemize}
	\item     Q ≤ 1000000;   
	\item     1 ≤ L ≤ R ≤ 10000;   
	\item     20\% số test có R ≤ 500;   
	\item     40\% số test có R ≤ 5000;   
\end{itemize}
Chấm điểm
Bài của bạn sẽ được chấm trên thang điểm 100. Điểm mà bạn nhận được sẽ tương ứng với \% test mà bạn giải đúng.  

   Trong quá trình thi, bài của bạn sẽ chỉ được chấm với 1 test ví dụ có trong đề bài.  

   Khi vòng thi kết thúc, bài của bạn sẽ được chấm với bộ test đầy đủ.
Example
\begin{verbatim}
\textbf{Input:}
3


1 3


3 13


15 20

\textbf{Output:}
2


5


6\end{verbatim}
Giải thích
Trong trường hợp đầu tiên, con heo nhà Bờm nặng trong khoản [1;3] (Kg). Có nhiều cách để Bờm chỉ tốn tối đa 2 đồng. Và đây là một trong những cách đó:  
\begin{itemize}
	\item     Đặt lên đĩa bên TRÁI: quả cân 1 (Kg);   
	\item     Đặt lên đĩa bên PHẢI: quả cân 3 (Kg);   
	\item     Cuối cùng đặt heo lên đĩa bên TRÁI;   
\end{itemize}

   Nếu:  
\begin{itemize}
	\item     Trái $>$ Phải : heo nặng 3 (Kg);   
	\item     Trái = Phải : heo nặng 2 (Kg);   
	\item     Trái $<$ Phải : heo nậng 1 (Kg);   
\end{itemize}
\begin{itemize}
\end{itemize}

   Cân một lần và sử dụng 2 quả cân, nên tốn đúng 2 đồng.