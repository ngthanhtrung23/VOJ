- Dòng đầu tiên là số nguyên n (3  $\le$  n  $\le$  5000) là số đỉnh của đa giác lồi mô tả cánh đồng của Phú ông.   
\\   - Mỗi dòng trong n dòng tiếp theo chứa cặp tọa độ của một đỉnh của đa giác lồi.   
\\   - Dòng tiếp theo là số nguyên m (2  $\le$  m  $\le$  5000) - số thiên thạch rơi xuống.   
\\   - Mỗi dòng trong số m dòng cuối cùng chứa 2 số là tọa độ điểm rơi của một thiên thạch.   
\\   Các tọa độ là các số nguyên có trị tuyệt đối không quá 10^6.