Hãy tìm tất cả các số nguyên tố trong đoạn [A,B] .  

\