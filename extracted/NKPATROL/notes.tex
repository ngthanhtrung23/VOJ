\begin{verbatim}
\textbf{Input:}
3

\textbf{Output:}
6\textbf{
\\
\\Giải thích:} có 6 cách đi
\\1. (1,1) --> (1,3) --> (2,3) --> (2,2) --> (3,2) --> (3,1) --> (1,1)
\\2. (1,1) --> (1,3) --> (3,3) --> (3,2) --> (2,2) --> (2,1) --> (1,1)
\\3. (1,1) --> (1,2) --> (2,2) --> (2,3) --> (3,3) --> (3,1) --> (1,1)
\\4. (1,2) --> (1,3) --> (3,3) --> (3,1) --> (2,1) --> (2,2) --> (1,2)
\\5. (1,1) --> (1,2) --> (3,2) --> (3,3) --> (2,3) --> (2,1) --> (1,1)
\\6. (1,2) --> (1,3) --> (2,3) --> (2,1) --> (3,1) --> (3,2) --> (1,2)
\\Ta có (3,3) --> (1,3) --> (1,2) --> (2,2) --> (2,1) --> (3,1) --> (3,3) 
\\cũng là một đường đi đúng, nhưng hình vẽ của nó đã trùng với hình của đường số 4.\end{verbatim}

\emph{
\includegraphics{http://www.spoj.pl/content/yellowflash12:nkpatrol.png}}

\textbf{    Sân N=3 và 6 dạng đường đi (xem mỗi ô trên sân là 1 chấm đỏ)   }   .