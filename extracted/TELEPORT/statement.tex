Pháp sư vĩ đại Byter đã phù phép tạo nên 2 hòn đảo trên biển Baltic : đảoBornholm và đảo Gotland . Ở mỗi đảo thì ông cũng tạo nên một vài cổng dịch chuyển tức thời ( CDCTT ) . Mỗi CDCTT sẽ là 1 trong 2 loại hình sau :
\begin{enumerate}
	\item Cổng Đến : Người ta sẽ được di chuyển tới cổng này .
	\item Cổng Đi : Khi bước vào cổng này người ta sẽ được đưa tới 1 Cổng Đến duy nhất xác định nằm ở hòn đảo kia .
\end{enumerate}


\\Một lần Byter đã giao cho các học trò của mình bài toán như sau : Cho biết số lượng CDCTT ở mỗi hòn đảo . Các học trò phải xác định xem cổng nào sẽ là Cổng Đến , cổng nào sẽ là Cổng Đi sao cho thoả mãn yêu cầu sau : Giả sử cổng i được đặt là Cổng Đến thì có ít nhất 1 Cổng Đi sẽ đưa người được dịch chuyển tới cổng i này và ngược lại , cổng i được đặt là Cổng Đi thì cổng mà nó gửi người đến phải được đặt là Cổng Đến .