Có một băng giấy gồm N ô vuồng liên tiếp nhau. K rất thích chơi tô màu với băng giấy này. Mỗi ô vuông K sẽ tô một màu. Nhưng vì nhà quá nghèo, quá nghèo và quá nghèo, nên K chỉ đủ tiền mua 4 màu là 'a', 'b', 'c' và 'd'. Bần cùng sinh đạo tặc, K cũng chôm thêm một cuốn sách tô màu ở thư viện để bù đắp cho sự thiếu óc thẩm mĩ của mình. Sách có rất nhiều mẫu tô màu khác nhau, mỗi mẫu là một dãy N kí tự thể hiện từng màu trong từng ô của băng giấy. Các mẫu được xếp theo thứ tự từ điển ('a' $<$ 'b' $<$ 'c' $<$ 'd'). Sách còn dạy rằng có một số màu không nên tô ở sau một màu khác, như thế sẽ không đẹp. Một ngày đẹp trời, tâm trạng buồn bực, K xé nát trang sách in mẫu thứ T trong sách. Than ôi, hối hận quá, K quyết định khôi phục lại mẫu tô màu đó, nhưng làm sao đây?