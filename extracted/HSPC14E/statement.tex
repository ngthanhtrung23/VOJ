Siêu mũ của số a bởi một số nguyên dương b, kí hiệu là a       a 1 = a,       a (k+1) = a (a       b hoặc là b a, được xác định như sau:       k)       .       Do đó ta có ví dụ: 3       2 = 3 3 = 27, như vậy 3       Tìm 8 chữ số cuối cùng của a       3 = 3 27 = 7625597484987       b.    

   Siêu mũ của số a bởi một số nguyên dương b, kí hiệu là a^^b được xác định như sau:  

   a^^1 = a,  

   a^^(k+1) = a^(a^^k)  

   Do đó ta có ví dụ:  

   3^^2 = 3^3 = 27, như vậy 3^^3 = 3^27 = 7625597484987  

   Tìm 8 chữ số cuối cùng của a.  



\