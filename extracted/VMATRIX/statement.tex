Cho 3 ma trận A, B, C kích thước N*N. (N  $\le$  1000), gồm các số nguyên từ 0 đến 9.

Các hàng của mỗi ma trận được đánh số từ 1 đến N từ trên xuống dưới. Các cột của mỗi ma trận được đánh số từ 1 đến N từ trái sang phải.

Phần tử ở hàng i, cột j của ma trận A được ký hiệu là A(i,j). Tương tự với ma trận B và ma trận C.

 

Nhiệm vụ của bạn là kiểm tra đẳng thức A*B = C đúng hay sai. Các phép tính được thực hiện trên module 10.

Phép A*B ở đây là phép nhân ma trận, được định nghĩa như sau:
\begin{itemize}
	\item Với ma trận A kích thước m*n và ma trận B kích thước n*k, kết quả của phép nhân là ma trận C kích thước m*k, với
	\item C(i,j) = sum( A(i,k) * B(k,j) với k = 1..n)
\end{itemize}