Bảng thông tin điện tử được lắp trên các đường phố để cung cấp ngắn gọn các thông tin quan trọng, các sự kiện, khẩu hiệu ... Công ty điện tử Sáng Sao vừa cho xuất xưởng một bảng thông tin điện tử có dạng một hàng gồm \emph{ n } vị trí, mỗi vị trí hiển thị một ký tự. Các vị trí được đánh số từ 1 đến \emph{ n } từ trái qua phải. Các ký tự chạy từ phải qua trái. Cứ mỗi giây ký tự ở vị trí \emph{ i } chuyển sang vị trí \emph{ i− } 1 ( \emph{ i } = 2, 3, …, \emph{ n } ) và ký tự mới từ xâu dữ liệu vào được lên bảng ở vị trí \emph{ n } . Ban đầu, tất cả các vị trí đều chứa dấu cách.

Trong thời gian thử nghiệm, để kiểm tra chất lượng bảng Công ty Sáng Sao cho phát lên bảng xâu \emph{ S } được tạo thành từ cách viết liên tiếp các số tự nhiên 1, 2, 3, 4, ..., 10 $^ 15 $ . Như vậy, phần đầu của xâu, khi viết đến số 14 sẽ là

1234567891011121314

Nếu \emph{ n } = 5 thì ở giây thứ 3 kể từ lúc bắt đầu phát thử nghiệm trên bảng thông tin sẽ có nội dung

\textbf{\_   \_   1   2   3 }

và ở giây thứ 19 trên bảng thông tin sẽ có nội dung

\textbf{2   1   3   1   4 }

\textbf{Yêu cầu: } Cho \emph{ n $_$} và \emph{ t } , hãy xác định xâu được hiển thị trên bảng tại thời điểm \emph{ t } , giả thiết là thời điểm bắt đầu phát thử nghiệm là 0.

\