Dữ liệu                   Kết quả                   3 4 5 1 4         

           1 1 1          

           2 2 1          

           3 2 2          

           1 1          

           2 1          

           3 1          

           4 2                    3 3                   1 2 2 1 2         

           1 1 1          

           1 2          

           2 2                    1 2         
\\
\begin{itemize}
	\item        30\% số test có    N, M, K  $\le$  10.      
	\item        70\% số test có    N, M, K  $\le$  100.      
	\item        Thời gian cho    mỗi test là 2s.      
\end{itemize}
Ở ví dụ 1, bé 1 và bé  2 không thể cùng nói thật (do nói khác nhau về chuồng 2).  Tương tự bé 2 và bé 3 không thể cùng nói thật. Do có  đúng 1 bé nói dối nên bé 2 là bé nói dối, bé 1 và bé  3 là 2 bé nói thật, con vật trong 4 chuồng lần lượt là:  1, 1, 2, 2. Vì thế, cô giáo luôn đoán đúng 3 câu.     

      Ở ví dụ 2, em bé duy  nhất đã nói dối. Vì vậy ít nhất 1 trong 2 chuồng không  có con vật 1. Vì chỉ có 2 loại con vật nên ít nhất 1  trong 2 chuồng có con vật 2. Cô giáo đoán đúng 1 câu trong  trường hợp tệ nhất và 2 câu trong trường hợp tốt nhất.