Một dãy gồm \emph{ n } số nguyên không âm \emph{ a }$_ 1 $ , \emph{ a }$_ 2 $ ,..., \emph{ a $_ n $} được viết thành một hàng ngang, giữa hai số liên tiếp có một khoảng trắng, như vậy có tất cả ( \emph{ n­ } -1) khoảng trắng. Người ta muốn đặt \emph{ k } dấu cộng và ( \emph{ n- } 1- \emph{ k } ) dấu trừ vào ( \emph{ n­ } -1) khoảng trắng đó để nhận được một biểu thức có giá trị lớn nhất.

Ví dụ, với dãy gồm 5 số nguyên 28, 9, 5, 1, 69 và \emph{ k } = 2 thì cách đặt 28+9-5-1+69 là biểu thức có giá trị lớn nhất.

\textbf{Yêu cầu: } Cho dãy gồm \emph{ n $_$} số nguyên không âm \emph{ a }$_ 1 $ , \emph{ a }$_ 2 $ ,..., \emph{ a $_ n $} và số nguyên dương \emph{ k } , hãy tìm cách đặt \emph{ k } dấu cộng và ( \emph{ n- } 1- \emph{ k } ) dấu trừ vào ( \emph{ n­ } -1) khoảng trắng để nhận được một biểu thức có giá trị lớn nhất.