Một ngày nọ, Tí Chuột tìm đường về hang của mình. Hang của Tí Chuột có rất nhiều cửa hang nằm rải rác trên các ô của một bảng hình chữ nhật kích thước M x N. Chỉ cần đi vào bất kỳ cửa hang nào là Tí Chuột sẽ về được chỗ ở của mình. Trên một số ô có vật cản và Tí Chuột không thể đi vào các ô đó. Mỗi bước Tí Chuột có thể đứng nguyên tại chỗ hoặc đi vào một trong 4 ô kề cạnh xung quanh. Tí Chuột rất thông minh, do đó nó luôn chọn đi theo ô nằm trong đường đi ngắn nhất để về hang. Nếu có nhiều lựa chọn, Tí Chuột sẽ ưu tiên theo thứ tự: đi lên trên, đi sang phải, đi xuống dưới, đi sang trái và đứng yên.

Tuy nhiên Bé Mèo không muốn Tí Chuột về được hang của mình. Bé Mèo cũng thông minh không kém; mỗi bước Bé Mèo sẽ tìm cách đặt một vật cản vào một ô trống, và tất nhiên phải là ô Tí Chuột không đang đứng. Mục tiêu của Bé Mèo là tìm cách đặt vật cản sao cho Tí Chuột không còn đường về hang nữa.

Bạn hãy lập trình giúp Bé Mèo đặt càng ít vật cản càng tốt để đạt được mục tiêu nhé! Biết rằng ở bước đi đầu tiên, Bé Mèo sẽ đặt vật cản trước, sau đó mới đến lượt Tí Chuột đi.