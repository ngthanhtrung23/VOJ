Nam rất thích hoán vị. Một hoán vị N là một cách sắp xếp N số nguyên dương từ 1 đến N, mỗi số chỉ xuất hiện một lần. Ví dụ 1 3 5 2 4 là một hoán vị 5.

Phép nhân 2 hoán vị N (a $_ 1 $ , a $_ 2 $ , a $_ 3 $ , … , a $_ n $ ) và (b $_ 1 $ , b $_ 2 $ , b $_ 3 $ , … ,b $_ n $ ) được định nghĩa như sau
\\(a $_ 1 $ , a $_ 2 $ , a $_ 3 $ , … , a $_ n $ ) x (b $_ 1 $ , b $_ 2 $ , b $_ 3 $ , … ,b $_ n $ ) = (a $_ b1 $ ,a $_ b2 $ , a $_ b3 $ , …, a $_ bn $ )
\\Ví dụ : (2 5 1 4 3) x (3 4 2 5 1) = (1 4 5 3 2)

Phép lũy thừa hoán vị được định nghĩa theo phép nhân hoán vị :
\\(a $_ 1 $ , a $_ 2 $ , a $_ 3 $ , … , a $_ n $ ) $^ 2 $ = (a $_ 1 $ , a $_ 2 $ , a $_ 3 $ , … , a $_ n $ ) x (a $_ 1 $ , a $_ 2 $ , a $_ 3 $ , … , a $_ n $ )
\\(a $_ 1 $ , a $_ 2 $ , a $_ 3 $ , … , a $_ n $ ) $^ k $ = (a $_ 1 $ , a $_ 2 $ , a $_ 3 $ , … , a $_ n $ ) x (a $_ 1 $ , a $_ 2 $ , a $_ 3 $ , … , a $_ n $ ) x … x (a $_ 1 $ , a $_ 2 $ , a $_ 3 $ , … , a $_ n $ )  (k phép nhân hoán vị)
\\
\\Nam nhận thấy có những số nguyên X mà (a $_ 1 $ , a $_ 2 $ , a $_ 3 $ , … , a $_ n $ ) $^ X $ = (a $_ 1 $ , a $_ 2 $ , a $_ 3 $ , … , a $_ n $ ). Khi đó ta gọi X là một chu trình của (a $_ 1 $ , a $_ 2 $ , a $_ 3 $ , … , a $_ n $ ).
\\Với một một hoán vị ban đầu (a $_ 1 $ , a $_ 2 $ , a $_ 3 $ , … , a $_ n $ ). Nam muốn tìm số nguyên dương K nhỏ nhất sao cho K+1 là một chu trình của (a $_ 1 $ , a $_ 2 $ , a $_ 3 $ , … , a $_ n $ ). Hãy giúp Nam.

\