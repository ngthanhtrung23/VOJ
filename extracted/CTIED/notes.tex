\begin{verbatim}
\textbf{Input:}2 10 6 1 \end{verbatim}
\begin{verbatim}
2 3 \end{verbatim}
\begin{verbatim}
2 1 \end{verbatim}
\begin{verbatim}
6 1 \end{verbatim}
\begin{verbatim}
2 4 \end{verbatim}
\begin{verbatim}
1 1 \end{verbatim}
\begin{verbatim}
2 0 \end{verbatim}
\begin{verbatim}
3 1 \end{verbatim}
\begin{verbatim}
1 3 \end{verbatim}
\begin{verbatim}
5 4 \end{verbatim}
\begin{verbatim}
3 0 \end{verbatim}
\begin{verbatim}
0 1 \end{verbatim}
\begin{verbatim}
3 2 \end{verbatim}
\begin{verbatim}
6 1\textbf{Output:}1\end{verbatim}
\begin{verbatim}
There are two posts at (2,3) and (2,1).  Bessie is at (6,1).  The rope goes \end{verbatim}
\begin{verbatim}
from (6,1) to (2,4) to (1,1), and so on, ending finally at (6,1). The shape of \end{verbatim}
\begin{verbatim}
the rope is the same as in the figure above.\end{verbatim}
\begin{verbatim}
Removing either post 1 or post 2 will allow Bessie to escape.\end{verbatim}