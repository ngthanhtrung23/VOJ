

Đếm số cách biểu diễn của 1 số nguyên thành tổng các số nguyên tố liên tiếp. Ví dụ :53 có hai cách là 5 + 7 + 11 + 13 + 17 và 53. 41 có ba cách 2+3+5+7+11+13, 11+13+17, và 41. Số 20 không có cách nào vì các biểu diễn như 7 + 13 và 3 + 5 + 5 + 7 không gồm các số nguyên tố liên tiếp.

\subsubsection{Input}

Một dãy các số nguyên dương  $\le$  11000, kết thúc là số 0 (ko xử lý).

\subsubsection{Output}

Số cách biểu diễn thành tổng các số nguyên tố liên tiếp cho từng số.
\begin{verbatim}
\textbf{SAMPLE INPUT
}2
3
17
41
20
666
12
53
0
\textbf{SAMPLE OUTPUT}
1
1
2
3
0
0
1
2
\end{verbatim}
