



   Bờm sống trên mảnh đất tổ tiên để lại từ xa xưa. Tuy nhiên, trải qua bao đời, mảnh đất của Bờm ngày nay có thể đã bị thay đổi vị trí, thậm chí còn có thể không giao với mảnh đất của tổ tiên! Một ngày nọ,   Bờm phát hiện tấm bản đồ mô tả hình dạng mảnh đất của tổ tiên. Bờm muốn xác định xem mảnh đất ngày nay và mảnh đất tổ tiên có còn giao nhau hay không!  

   Yêu cầu: Biết mảnh đất ngày nay của Bờm và mảnh đất của tổ tiên đều có hình dạng đa giác lồi. Hãy giúp Bờm xác định 2 mảnh đất có giao nhau (nghĩa là có phần diện tích chung) hay không.  

\subsubsection{   Dữ liệu  }

   Dòng đầu tiên chứa số nguyên t, cho biết số lượng test (t ≤ 10). t nhóm dòng tiếp theo mô tả các test, mỗi test có dạng như sau:  
\begin{itemize}
	\item     Dòng đầu tiên chứa số nguyên m, số đỉnh của đa giác lồi miêu tả mảnh đất của Bờm.   
	\item     Dòng thứ 2 chứa 2m số nguyên cho biết tọa độ các đỉnh của mảnh đất của Bờm. Các đỉnh được liệt kê theo chiều kim đồng hồ.   
	\item     Dòng thứ 3 chứa số nguyên n, số đỉnh của đa giác lồi miêu tả mảnh đất của tổ tiên.   
	\item     Dòng thứ 4 chứa 2n số nguyên cho biết tọa độ các đỉnh của mảnh đất của tổ tiên Bờm. Các đỉnh được liệt kê theo chiều kim đồng hồ.   
\end{itemize}

\subsubsection{   Kết qủa  }

   Gồm t dòng, mỗi dòng ghi ra “YES” / “NO” nếu 2 mảnh đất giao nhau / không giao nhau trong test tương ứng.  

\subsubsection{   Giới hạn  }
\begin{itemize}
	\item     3 ≤ m, n ≤ 1000   
	\item     Tọa độ các đỉnh có giá trị tuyệt đối không vượt quá 1000000000.   
	\item     Có 50\% số test mà trong đó các số m,n đều có giá trị không vượt quá 100.   
\end{itemize}

\subsubsection{   Ví dụ  }
\begin{verbatim}
Dữ liệu:
2
3
-6 3 -11 11 -10 6 
3
-4 0 -3 5 -7 3 
3
-3 4 -3 11 -6 9 
3
0 -2 1 0 -2 -1 

Kết qủa
YES
NO
\end{verbatim}