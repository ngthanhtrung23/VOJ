

 

 

Cho N điểm và P hình chữ nhật trên mặt phẳng tọa độ. Tính xem với mỗi hình chữ nhật có bao nhiêu điểm nằm ở trên cạnh của nó trong P điểm đã cho.

\textbf{Input}

Dòng đầu ghi số điểm N (1 ≤ N ≤ 300 000).

N dòng tiếp theo mỗi dòng ghi 2 số X, Y (1 ≤ X, Y ≤ 10\textasciicircum9). Không có hai điểm nào trùng nhau.

Dòng tiếp theo ghi số P, (1 ≤ P ≤ 100 000), số hình chữ nhật. P dòng tiếp theo, mỗi dòng ghi 4 số X1, Y1, X2 , Y2 (1 ≤ X1 $<$ X2 ≤ 10\textasciicircum9, 1 ≤ Y1 $<$ Y2 ≤ 10\textasciicircum9)  là tọa độ góc trái dưới (X1, Y1) và góc phải trên (X2, Y2) của từng hình chữ nhật.

\href{http://tinypic.com}{
\includegraphics{http://i47.tinypic.com/9uq2qx.jpg}}

\textbf{Output}

Ghi ra P số nguyên, mỗi số trên 1 dòng, là số điểm nằm trên cạnh của từng hình chữ nhật.

\textbf{Sample}
\begin{verbatim}
input  
6 
1 2 
3 2 
2 3 
2 5 
4 4 
6 3 
4 
2 2 4 4 
2 2 6 5 
3 3 5 6 
5 1 6 6  
output  
3 
4 
0 
1
\end{verbatim}