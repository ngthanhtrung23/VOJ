

Một thang máy có 4 nút như sau:
\begin{itemize}
	\item Đi lên a tầng.
	\item Đi lên b tầng.
	\item Đi lên c tầng.
	\item Trở về tầng 1.
\end{itemize}

Hiện tại, thang máy đang ở tầng 1. Hành khách có thể ấn các nút để đi lên tầng họ muốn. Nếu họ muốn ấn nút a, hoặc b, hoặc c mà tầng đó không tồn tại (cao quá) thì thang máy đứng yên.

Tính xem có thể đến được bao nhiêu tầng nếu thang máy xuất phát từ tầng đầu tiên.

\subsubsection{Input}

Dòng đầu tiên là h-chiều cao tòa nhà (1 ≤ h ≤ 10\textasciicircum18).

Dòng thứ hai là ba số a, b và c - (1 ≤ a, b, c ≤ 100000)
\begin{verbatim}
SAMPLE INPUT 1
15
4 7 9

SAMPLE INPUT 2
500000
160 96 111

SAMPLE INPUT 3
987654321987654321
99995 99997 99999

\end{verbatim}

\subsubsection{Output}

Số nguyên ghi số tầng có thể đến được từ tầng 1.
\begin{verbatim}
SAMPLE OUTPUT 1
9
SAMPLE OUTPUT 2
498167
SAMPLE OUTPUT 3
987654319487854318
\end{verbatim}

 