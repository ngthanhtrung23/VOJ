



   Cho một đồ thị dạng cây gồm   \textbf{    N   }   đỉnh, đỉnh thứ i có trọng số là w[i]. Cho   \textbf{    K   }   lần cắt cây (bằng cạnh xóa đi 1 cạnh đang có), ta sẽ tạo ra một rừng gồm   \textbf{    K+1   }   cây. Trọng số của một cây được định nghĩa là tổng trọng số các đỉnh trong cây đó.  

   Bài toán yêu cầu bạn hãy tìm cách cắt cây sao cho trọng số to nhất của các cây là nhỏ nhất.  

\subsubsection{   Input  }

   Dòng đầu tiên gồm 2 số N, K (K $<$ N)  

   Dòng tiếp theo gồm N số là trọng số của các đỉnh  

   N-1 dòng tiếp theo, mỗi dòng gồm 2 số u, v mô tả một cạnh của cây  

\subsubsection{   Output  }

   Một số nguyên duy nhất là trọng số bé nhất.  

\subsubsection{   Giới hạn  }
\begin{itemize}
	\item     0  $\le$  K $<$ N  $\le$  10\textasciicircum5   
	\item     0 $<$ w[i]  $\le$  10\textasciicircum4   
	\item     Trong 20\% số test, N  $\le$  20   
	\item     Trong 20\% số test tiếp theo, N  $\le$  200   
	\item     Trong 20\% số test tiếp theo, N  $\le$  1000, trọng số các đỉnh bằng 1.   
\end{itemize}

    Sau khi kết thúc kỳ thi, kết quả của bạn là kết quả lần nộp bài cuối cùng   

\textbf{}

\subsubsection{\textbf{    Example   }}

\paragraph{   Input:  }
\begin{verbatim}
8 2
7 4 3 8 5 7 5 4
2 1
3 1
4 3
5 2
6 1
7 6
8 1\end{verbatim}

\paragraph{   Output:  }
\begin{verbatim}
20\end{verbatim}
