

Xét các số nguyên từ 1 đế N. Các số này được sắp xếp theo thứ tự từ điển. Ví dụ với N=11, ta có dãy số sau khi sắp xếp là 1, 10, 11, 2, 3, 4, 5, 6, 7, 8, 9.

Ký hiệu Q$_N,K $ là vị trí của số K trong dãy được sắp xếp theo cách nói trên. Ví dụ Q$_11,2 $ =4 Cho các số nguyên K và M. Hãy tìm số nguyên N nhỏ nhất thỏa mãn Q$_N,K $ =M

\subsubsection{Dữ liệu vào}

Dòng đầu tiên chứa số nguyên t cho biết số bộ test.

Mỗi bộ test bao gồm 1 dòng duy nhất chứa 2 số nguyên K và M (1 $<$= K, M $<$= 10$^9 $ )

\subsubsection{Kết qủa}

Với mỗi bộ test xuất ra số N, hoặc 0 nếu không tồn tại N

\subsubsection{Ví dụ}
\begin{verbatim}
Dữ liệu mẫu
1
2 4
Kết qủa
11
\end{verbatim}