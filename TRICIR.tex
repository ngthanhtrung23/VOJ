



\subsubsection{   Đề bài  }

   Cho N điểm cách đều nhau trên một vòng tròn được đánh số từ 0 đến N-1 theo chiều kim đồng hồ, trong đó có P điểm được sơn màu đỏ. Hãy đếm số tam giác vuông có 3 đỉnh đều được sơn màu đỏ.  

   Biết rằng P điểm màu đỏ được tạo thành như sau, cho trước 3 số nguyên a, b, c. Với i = 0, 1, 2, 3, ..., P - 1, thực hiện các bước sau:  
\begin{itemize}
	\item     Tính P[i] = (a*i*i + b*i + c) mod N   
	\item     Bắt đầu từ P[i], tìm điểm đầu tiên theo chiều kim đồng hồ mà chưa được sơn đỏ và sơn đỏ điểm đó   
\end{itemize}

\subsubsection{   Dữ liệu  }
\begin{itemize}
	\item     Mỗi test bắt đầu bằng thẻ "[CASE]", các test cách nhau bởi một dòng trắng. Thẻ "[END]" báo hiệu kết thúc file input.   
	\item     Mỗi test gồm 5 dòng chứa các số N, P, a, b, c   
\end{itemize}

\subsubsection{   Kết quả  }
\begin{itemize}
	\item     Với mỗi test in ra số tam giác vuông tìm được.   
\end{itemize}

\subsubsection{   Giới hạn  }
\begin{itemize}
	\item     1 $<$= N $<$= 1 000 000   
	\item     0 $<$= P $<$= 100 000   
	\item     0 $<$= a, b, c $<$= 1 000 000   
\end{itemize}

\subsubsection{   Ví dụ  }
\begin{verbatim}
Dữ liệu
[CASE]
9
3
0
3
0

[CASE]
40
3
5
0
0

[CASE]
4
4
16
24
17
    	
[CASE]
1000000
47000
0
2
5

[CASE]
200000
700
123456
789012
345678

[END]
Kết quả
0
1
4
0
6980
\end{verbatim}