



   Những người-biết-về-bò đã nhận ra cách mà các con bò nhóm lại thành  các “vùng hàng xóm bò”. Họ đã quan sát N (1  $\le$  N  $\le$  100,000) con bò của  nông dân John (để tiện ta sẽ đánh số các con bò từ 1..N) khi chúng  đi ăn cỏ, tọa độ của các con bò là khác nhau và là các số nguyên,  coi đồng cỏ là hình chữ nhật có các tọa độ X và Y trong  khoảng 1..1,000,000,000.  

   Hai con bò là hàng xóm nếu có 1 trong 2 điều kiện sau:  
\begin{enumerate}
	\item     Khoảng cách Manhattan của các con bò không lớn hơn 1 số nguyên       C cho trước (1  $\le$  C  $\le$  1,000,000,000). [Khoảng cách Manhattan      tính như sau: d = |x1-x2| + |y1-y2|.]   
	\item     Nếu bò A là hàng xóm của bò Z và bò B cũng là hàng xóm của bò Z       thì bò A và bò B cũng là hàng xóm.   
\end{enumerate}

   Một vùng hàng xóm bò là một tập các con bò là hàng xóm của nhau  và không là hàng xóm của bất kỳ con bò nào khác ngoài tập này. Cho vị trí của các con bò và khoảng cách C, xác định số vùng hàng  xóm bò và vùng có nhiều bò nhất.  

   Ví dụ như ở hình bên dưới là đồng cỏ. Với C = 4, đồng cỏ này có  4 vùng hàng xóm bò: 1 vùng lớn ở bên trái, 2 vùng nhỏ có  kích thước 1 (các con bò cô đơn), và một vùng rất lớn ở bên  phải với 60 con bò.  
\begin{verbatim}
.....................................*.................
....*...*..*.......................***.................
......*...........................****.................
..*....*..*.......................*...*.******.*.*.....
........................*.............***...***...*....
*..*..*...*..........................*..*...*..*...*...
.....................................*..*...*..*.......
.....................................*..*...*..*.......
...*................*..................................
.*..*............................*.*.*.*.*.*.*.*.*.*.*.
.*.....*..........................*.*.*.*.*.*.*.*.*.*.*
....*..................................................
\end{verbatim}

   Dữ liệu từ tập tin input sẽ mô tả các tọa độ bằng các số  nguyên X, Y với vị trí góc trái dưới là (1,1) và các con bò  ở gần góc này nằm ở vị trí (2,2) và (5,1) trong ví dụ ở trên.  

   Hình trên là test ví dụ số 2, test này sẽ dùng để đánh giá  bài nộp của bạn.  

   Một phần kết quả của một số test ví dụ sẽ được thông báo ở 10  lần nộp đầu tiên.  

\subsubsection{   Dữ liệu  }
\begin{itemize}
	\item     Dòng 1: 2 số nguyên cách nhau bởi dấu cách: N và C   
	\item     Dòng 2..N+1: Dòng i+1 mô tả vị trí của con bò thứ i gồm 2 số nguyên cách nhau bởi dấu cách: X\_i và Y\_i   
\end{itemize}

\subsubsection{   Kết quả  }
\begin{itemize}
	\item     Dòng 1: 2 số nguyên cách nhau bởi dấu cách tương ứng là số   vùng hàng xóm bò và số lượng bò ở vùng lớn nhất.   
\end{itemize}

\subsubsection{   Ví dụ  }
\begin{verbatim}
Dữ liệu
4 2
1 1
3 3
2 2
10 10

Kết quả
2 3
\end{verbatim}

\subsubsection{   Giải thích  }

   Có 2 vùng bò, một vùng là 3 con bò đầu tiên và vùng thứ 2 là các con bò còn lại. Vùng lớn nhất có 3 con bò.  
