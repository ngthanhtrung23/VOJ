

Trên mặt phẳng tọa độ Đề các vuông góc Oxy cho n điểm phân biệt Ai(xi, yi) i = 1, 2, 3, …, n.   Mỗi điểm Ai được tô bởi màu Ci thuộc \{1, 2, 3, 4\}. Ta gọi hình chữ nhật bốn màu là hình chữ nhật thỏa mãn hai điều kiện sau:
\begin{itemize}
	\item Bốn đỉnh của hình chử nhật là bốn điểm trong n điểm đã cho và được tô bởi bốn màu khác nhau.
	\item Các cạnh của hình chử nhật song song với một trong hai trục tọa độ.
\end{itemize}

\textbf{Yêu cầu: } Cho biết tọa độ và màu của n điểm, hãy đếm số lượng hình chữ nhật bốn màu.

\textbf{Dữ liệu: }
\begin{itemize}
	\item Dòng đầu tiên chứa số nguyên dương n (4  $\le$  n  $\le$  10\textasciicircum5) là số lượng điểm trên mặt phẳng.
	\item Dòng thứ i trong n dòng tiếp theo chứa ba số nguyên xi, yi, ci (|xi|, |yi|  $\le$  200)  là thông tin về tọa độ và màu của điểm thứ i (i = 1, 2, 3, .., n).
	\item Các số trên cùng một dòng được ghi cách nhau ít nhất một dấu cách.
\end{itemize}

\textbf{Kết quả: } Ghi ra trên một dòng số lượng hình chữ nhật đếm được.

\textbf{Ví dụ: }

\textbf{Input }
\begin{verbatim}
7
0 0 1
0 1 4
2 1 2
2 -1 3
0 -1 1
-1 -1 4
-1 1 1\end{verbatim}

\textbf{Output}
\begin{verbatim}
2\end{verbatim}


\includegraphics{http://vn.spoj.com/content/voj:COLOREC.png}
\includegraphics{../../../content/voj:COLOREC.png}

\textbf{Ràng buộc: } 50\% số test ứng với 50\% số điểm của bài có 4  $\le$  n  $\le$  100
