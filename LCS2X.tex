

 

Dãy C = c1, c2, .. ck được gọi là dãy con của dãy A = a1, a2, .., an nếu C có thể nhận được bằng cách xóa bớt một số phần tử của dãy A và giữ nguyên thứ tự của các phần tử còn lại, nghĩa là tìm được dãy các chỉ số 1 ≤ l1 $<$ l2 $<$ … $<$ lk ≤ n sao cho c1 = a\_l1, c2 = a\_l2, …, ck = a\_lk. Ta gọi độ dài của dãy là số phần tử của dãy.

Cho hai dãy A = a1, a2, …, am và B = b1, b2, …, bn Dãy C = c1, c2, …, ck được gọi là dãy con chung bội hai của dãy A và B nếu C vừa là dãy con của dãy A, vừa là dãy con của dãy B và thỏa mãn điều kiện 2 × ci ≤ c(i+1) (i = 1, 2, …, k – 1).

\subsubsection{Yêu cầu}

Cho hai dãy A và B. Hãy tìm độ dài dãy con chung bội hai có độ dài lớn nhất của hai dãy A và B.

\subsubsection{Input}

Dòng đầu tiên chứa T là số lượng bộ dữ liệu. Tiếp đến là T nhóm dòng, mỗi nhóm cho thông tin về một bộ dữ liệu theo khuôn dạng sau:
\begin{itemize}
	\item Dòng đầu chứa 2 số nguyên dương m và n.
	\item Dòng thứ hai chứa m số nguyên không âm a1, a2, ..., am mỗi số không vượt quá 10\textasciicircum9.
	\item Dòng thứ ba chứa n số nguyên không âm b1, b2, ..., bn mỗi số không vượt quá 10\textasciicircum9.
	\item Các số trên cùng một dòng được ghi cách nhau ít nhất một dấu cách.
\end{itemize}

\subsubsection{Giới hạn}
\begin{itemize}
	\item 30\% số test có m, n  $\le$  15.
	\item 30\% số test khác có m, n  $\le$  150.
	\item có 40\% số test còn lại có m, n  $\le$  1500.
\end{itemize}

\subsubsection{Output}

Ghi ra T dòng, mỗi dòng ghi một số nguyên là độ dài dãy con chung bội hai dài nhất của dãy A và B tương ứng với bộ dữ liệu vào.

\subsubsection{Example}
\begin{verbatim}
\textbf{Input:}
1
5 5
5 1 6 10 20
1 8 6 10 20

\textbf{Output:}
3
\end{verbatim}
