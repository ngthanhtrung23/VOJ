

 

Cho dãy số nguyên dương a $_ 1 $ , a $_ 2 $ , ..., a $_ n $ .

Dãy số: a $_ i $ , a $_ i+1 $ , ..., a $_ j $ thỏa mãn a $_ i $ ≤ a $_ i+1 $ ≤ ... ≤ a $_ j $ . Với 1 ≤ i ≤ j ≤ n được gọi là dãy con không giảm của dãy số đã cho và khi đó số j-i+1 được gọi là độ dài của dãy con này.

Yêu cầu: Trong số các dãy con không giảm của dãy số đã cho mà các phần tử của nó đều thuộc dãy số \{u $_ k $ \} xác định bởi u $_ 1 $ = 1, u $_ k = u $_ k $ -1 $ + k (k ≥ 2), hãy tìm dãy con có độ dài lớn nhất.

\subsubsection{Input}

Dòng đầu tiên chứa một số nguyên dương n (n ≤ 10 $^ 4 $ ).

Dòng thứ i trong n dòng tiếp theo chứa một số nguyên dương a $_ i $ (a $_ i $ ≤ 10 $^ 8 $ ) là số hạng thứ i của dãy số đã cho, i = 1, 2, ..., n.

\subsubsection{Output}

Gồm 1 dòng duy nhất ghi số nguyên d là độ dài của dãy con không giảm tìm được (quy ước rằng nếu không có dãy con nào thỏa mãn điều kiện đặt ra thì d = 0).

\subsubsection{Example}
\begin{verbatim}
Input:
8
2
2007
6
6
15
16
3
21
Output:
3

\end{verbatim}